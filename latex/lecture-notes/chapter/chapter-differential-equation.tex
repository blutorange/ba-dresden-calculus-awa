\chapter{Differentialgleichungen}

TODO: Einführende Beispiele, Näherungslösung einer DGL

Differentialgleichungen sind eines der wichtigsten Fundamente für die Beschreibung von Zusammenhängen in der Natur. Sie stellen einen Zusammenhang her zwischen einer Größe und ihrer Änderungsrate. In der klassischen Mechanik etwa stellt man sich die Frage, wie sich ein Körper und Einfluss einer Kraft bewegt. Die gesamte Bahn des Körpers anzugeben, ist nicht einfach. Einfacher ist es zu betrachten, was lokal zu einem konkreten Zeitpunkt an einem konkreten Ort passiert. Dann stellt man fest, dass die Änderungsrate der Geschwindigkeit des Körpers zu jedem Zeitpunkt proportional ist zur Kraft, die in diesem Moment auf den Körper wirkt, geteilt durch seine Masse. Dieser Zusammenhang wird durch die Differentialgleichung $F=m\cdot a = m \cdot \ddot{x}(t)$ ausgedrückt. Wenn man jetzt die Änderungsraten der Geschwindigkeiten zu jedem Zeitpunkt aufaddiert, erhält man die gesamte Bahn, auf der sich der Körper bewegt -- man hat die Differentialgleichung gelöst. Oft ist es so, dass man zwar die vollständige Lösung nicht direkt angeben kann, aber mithilfe von Änderungen und Ableitungen einfach beschreiben kann, was lokal zu einem konkreten Zeitpunkt passiert.

\begin{itemize}
    \item Naturgesetze: Viele Gesetze in der Physik werden durch Differentialgleichungen beschrieben, wie etwa die Newtonsche Grundgleichung aus der klassischen Mechanik, die Navier-Stokes-Gleichung aus der Fluiddynamik, die Maxwell-Gleichungen aus der Elektrodynamik oder die Schrödinger- und Dirac-Gleichung aus der Quantenmechanik.
    \item Auch zur Modellierung von Zusammenhängen sind Differentialgleichung nützlich. Etwa lassen sich damit die Rate chemischer Reaktionen oder die Dynamik einer Population beschreiben.
    \item Differentialgleichungen aus der klassischen Mechanik werden auch zur nummerischen Simulation verwendet, etwa um die Eigenschaften und die Stabilität eines geplanten Bauwerks zu berechnen oder um in einer \emph{Physics Engine} die Bewegung von Körpern zu simulieren.
\end{itemize}

\section{Begriff der Differentialgleichung}

Zuerst müssen wir klären, was wir unter einer Differentialgleichung verstehen und was es bedeutet, eine Differentialgleichung zu lösen.

\begin{definition}{Differentialgleichung}{Deq}
    Eine Gleichung, welche eine Funktion sowie einer oder mehrere ihrer Ableitung enthält, nennt man \textbf{Differentialgleichung} (\textbf{DGL}).
\end{definition}

\begin{definition}{Ordnung einer DGL}{DeqOrder}
    Unter der \textbf{Ordnung $n$} einer DGL versteht man die höchste vorkommende Ableitung.
\end{definition}

So sind etwa folgende Gleichungen alle DGL:

\begin{itemize}
    \item $y' = y$ beziehungsweise $\dd{}{x} y = y$. Diese hat die Ordnung $1$, da $y'$ die höchste vorkommende Ableitung ist.
    \item $(y')^2 + 3 y' = \sin(y)$. Diese hat ebenfalls die Ordnung $1$.
    \item $y''+y'+4y=x^2$. Diese hat die Ordnung $2$, da $y''$ die höchste vorkommende Ableitung ist.
    \item $\pdd{f}{x} = \pdd{f}{y}$. Diese hat die Ordnung $1$, da nur erste partielle Ableitung nach $x$ und $y$ vorkommen.
\end{itemize}

Anmerkung zur Schreibweise: Um Schreibarbeit zu sparen, lässt man in oft das Argument der Funktion aus, also etwa $y'=y$ statt $y'(x) = y(x)$.

Im letzten Beispiel kamen partielle Ableitungen vor. Man unterscheidet allgemein zwischen DGL mit einstelligen Funktionen und DGL mit mehrstelligen Funktionen.

\begin{definition}{Ordinäre und partielle DGL}{OrdPartDGL}
    Eine DGL für eine einstellige Funktion nennt man \textbf{ordinäre DGL} (gewöhnliche DGL). Bei einer DGL mit einer zwei- oder mehrstelligen Funktionen spricht man von einer \textbf{partiellen DGL}.
\end{definition}

\begin{example}{Ordinäre und partielle DGL}{}
    \begin{itemize}
        \item $y''+2y=0$ ist eine gewöhnliche DGL.
        \item $\pdd{B_x}{x} + \pdd{B_y}{y} + \pdd{B_z}{z} = 0$ ist eine partielle DGL. Dabei ist $B$ das Magnetfeld. Diese DGL ist eine Teil der Maxwell-Gleichungen aus der Elektrodynamik und beschreibt die Tatsache, dass es keine magnetischen Monopole (Ladungen) gibt.
        \item $\pdd{u}{t} = k \cdot (\pddn{2}{u}{x} + \pddn{2}{u}{y} + \pddn{2}{u}{z})$ ist ebenfalls eine partielle DGL. Dabei ist $u$ die Temperatur an einem Punkt $(x,y,z)$ im Raum zu einem bestimmten Zeitpunkt $t$. Diese sogenannte Wärmeleitungsgleichung beschreibt, wie sich die Temperatur im Laufe der Zeit verändert. $k$ ist eine materialabhängige Konstante, welche ein Maß für die Temperaturleitfähigkeit ist.
    \end{itemize}
\end{example}

In dieser Vorlesung werden wir uns auf DGL mit einstelligen Funktionen beschränken.

\begin{definition}{Lösungen einer DGL}{DeqSol}
    Eine \textbf{Lösung} einer DGL $n$-ter Ordnung ist eine Funktion derart, dass die Gleichung für alle Argumente aus dem Definitionsbereich erfüllt ist. Die Menge aller Lösungen nenn man \textbf{allgemeine Lösung}, welche eine Kurvenschar mit $n$ willkürlichen wählbaren Parametern darstellt. Eine konkrete Lösungsfunktion aus dieser Schar entsteht durch eine konkrete Wahl der $n$ Parameter und wird \textbf{partikuläre Lösung} genannt. Eine Lösungsfunktion, welche nicht aus der allgemeinen Lösung durch Wahl durch eine bestimmte Belegung der Parameter entsteht, nennt man \textbf{singuläre Lösung}.
\end{definition}

\begin{example}{Arten von Lösungen einer DGL}{DeqSolType}
    \begin{itemize}
        \item Die DGL $y'=y$ etwa beschreibt Funktionen, die identisch mit ihrer Ableitung sind. Eine solche Funktion kennen wir bereits: die Exponentialfunktion $y=e^x$. Setzen wir dieses $y$ in $y'=y$ ein, so erhalten wir $(e^x)'=e^x$. Das ist eine wahre Aussage, also ist $y=e^x$ eine \emph{partikuläre} Lösung der DGL. Gibt es noch mehrere Lösungen? Wir wissen, dass ein konstanter Vorfaktor in der Ableitung einfach mitgezogen wird. Also sind alle Funktionen der Form $y=C\cdot e^x$ mit beliebigen $C\in\R$ die allgemeine Lösung der DGL $y'y$. Eine weitere partikuläre Lösungen erhalten wir, in dem wir beispielsweise $C=-3$ setzen: $y(x) = -3 e^x$.
        \item Die DGL $y'^2 + y^2 = 1$ hat die allgemeine Lösung $y(x) = \pm \sin(C+x)$. Eine partikuläre Lösung ist beispielsweise $y(x) = \sin(3+x)$. Zudem ist aber auch $y(x) = \pm 1$ eine Lösung, was man durch Einsetzen in die DGL nachprüfen kann. Da diese beiden Lösungen nicht durch wahl von $C$ aus  $y(x) = \pm \sin(C+x)$ entstehen, handelt es sich um \emph{singuläre Lösungen}.
    \end{itemize}
\end{example}

Im obigen Beispiel \ref{ex:DeqSolType} haben wir gesehen, dass die allgemeine Lösung eine Funktionsgleichung mit einer frei wählbaren Konstante ist, also eine parametrisierte Kurvenschar darstellt. Für die meisten DGL wird das der Fall sein, die allgemeine Lösung enthält in der Regel eine oder mehrere Konstanten. Diese Konstanten entstehen unter Anderem durch die Integrationskonstante des unbestimmten Integrals. Daher ist es wichtig, diese nicht zu vergessen, da man sonst statt der allgemeinen Lösung nur eine partikuläre Lösung erhält.

Um nun eine Lösung einer DGL zu finden, gibt es gewissen Rechenverfahren, die wir später kennen lernen werden. Zuerst wollen wir uns aber mit einfachen DGL vertraut machen und uns anschauen, wie man diese graphisch lösen kann. Diese graphische Lösung stellt auch die Idee für nummerische Lösungsmethoden dar, die man anwenden kann, wenn sich eine DGL nicht mehr geschlossen lösen lässt.

\begin{definition}{Richtungsfeld einer DGL}{SlopeField}
    Eine DGL der Form $y'(x) = g(y,x)$, in der nur eine einstellige Funktion und deren erste Ableitung vorkommt, kann graphisch durch ein \textbf{Richtungsfeld} dargestellt werden. Das Richtungsfeld weist jedem zweidimensionalen Punkt $(x,y)$ einen \textbf{Richtungspfeil} in Richtung des Anstieg $y'$ und frei wählbaren Betrag zu. Ausgehend von einem Startpunkt lässt sich dann durch Folgen der Pfeile graphisch eine partikuläre Lösung ermitteln.
\end{definition}

\begin{example}{Ermittlung des Richtungsfelds}{CompSlopeField}
    Wir wollen graphisch die Lösung von $-x \cdot y'=y$ bestimmen. Um das Richtungsfeld zeichnen zu können, müssen wir sie zuerst in die Form $y'(x) = g(y,x)$ umformen -- also nach $y'$ umstellen. Für $x \ne 0$ gilt $y' = -y/x$. Das Richtungsfeld ergibt sich also, indem wir in einem zweidimensionalen Koordinatensystem an jedem Punkt $(x,y)$ einen Pfeil in Richtung des Anstieg einzeichnen. Wenn wir etwa $\Delta x$ Einheiten nach rechts gehen, müssen wir $y' \cdot \Delta x = -\frac{y}{x} \cdot \Delta x$ Einheiten nach oben gehen. Etwa für $Delta x = 1$ ergibt sich, dass der Richtungspfeil am Punkt $(x,y)$ lauten muss: $(1,-frac{y}{x})$. Zweckmäßigerweise normiert man alle Richtungspfeilen so, dass sie die gleiche Länge haben und sich gegenseitig nicht überlappen.
    Um die Arbeit beim Zeichnen des Richtungsfelds zu verringern, können wir uns noch überlegen, wo sich Linien gleichen Anstieg befinden. Damit $y'=const.$ gilt, muss der Quotient $y/x$ konstant sein, $y$ muss also ein Vielfaches von $x$ sind. Daher gilt, dass auf allen Geraden durch den Ursprung (etwa $y=2x$ oder $y=-5x$) die Richtungspfeile in die gleiche Richtung zeigen. Solche Linien gleichen Anstieg nennt man auch Isokline, von \emph{ísos} ("gleich") und \emph{inclination} ("Neigung").
\end{example}

\begin{figure}
    \centering
    \includegraphics[width=0.65\textwidth]{./gnuplot/example-slope-field}
    \caption{Richtungsfeld der DGL $y'=-y/x$ und mögliche Anfangswerte der partikulären Lösungen}
    \label{fig:ExSlopField}
\end{figure}

In Abbildung \ref{fig:ExSlopField} ist solch ein Richtungsfeld dargestellt. Wir erkennen in dieser Abbildung auch noch etwa anderes. Um im Richtungsfeld eine partikuläre Lösungen einzuzeichnen, müssen wir den Pfeilen folgen. Doch dazu benötigen wir einen Startpunkt. Je nachdem, welchen Startpunkt wir wählen, erhalten wir eine andere Kurve. Oder anders ausgedrückt: das Richtungsfeld zeigt alle möglichen Lösungen (\emph{allgemeine Lösung}), eine konkrete Lösung (\emph{partikuläre Lösung}) wird dadurch ausgewählt, dass wir einen Punkt $(x,y)$ vorgeben, durch den die Kurve verlaufen soll.

\begin{definition}{Anfangs- und Randbedingungen}{InitBoundCond}
    Um eine partikuläre Lösung einer DGL $n$-ter Ordnung eindeutig zu bestimmen, werden neben der DGL noch zusätzliche Einschränkungen (im allgemeinen $n$ Stück) an die Lösungsfunktion benötigt. Dabei gibt es zwei wesentliche Arten von Bedingungen:
    \begin{itemize}
        \item Bei einem \textbf{Anfangswertproblem} sind \textbf{Anfangsbedingungen} gegeben, also der Wert der Funktion oder ihrer Ableitungen an einer Stelle.
        \item Bei einem \textbf{Randwertproblem} sind \textbf{Randbedingungen} gegeben, also der Wert der Funktion oder ihrer Ableitung am Rande eines betrachteten Gebiets.
    \end{itemize}
\end{definition}

Ein physikalisches Beispiel für ein Anfangswertproblem ist etwa der freie Fall, wenn Anfangsort x und Anfangsgeschwindigkeit $\dot{x} = \dd{x}{t}$ zum Zeitpunkt $t_0$ vorgegeben ist. Randwertprobleme treten meist bei partiellen Differentialgleichungen auf, beispielsweise kann man in der Elektrostatik das elektrische Potential $\varphi$ am Rand eines geerdeten Leiters vorgeben. Ein weiteres Beispiel für ein Randwertproblem ist Seil, welches an zwei Enden eingespannt ist. Damit ist die Position des Seils am Rand fest vorgeben, die Bewegung des Seils zwischen beiden Enden wird dann durch die Schwingungsgleichung beschrieben.

\section{Homogenität und Linearität}

Zwei wichtige Eigenschaften von DGL sind die sogenannten Homogenität und Linearität. Wir werden sehen, dass speziell lineare DGL einfacher lösbar sind als nichtlineare DGL. Die Navier-Stokes-Gleichung beschreibt wie bereits erwähnt Phänomene der Fluiddynamik. Es handelt es sich dabei um eine partielle und nichtlineare DGL. Aufgrund ihrer Nichtlinearität ist diese bisher noch nicht vollständig gelöst. Eines der sieben Millennium-Problem des \href{http://www.claymath.org/millennium-problems}{Clay Mathematics Institute} betrifft diese DGL:

\begin{quotation}
    \textbf{Navier–Stokes Equation}. This is the equation which governs the flow of fluids such as water and air. However, there is no proof for the most basic questions one can ask: do solutions exist, and are they unique? Why ask for a proof? Because a proof gives not only certitude, but also understanding.
\end{quotation}

Um die eben erwähnten Begriffe genau zu definieren, benötigen wir zunächst erstmal die kanonische Darstellungsform einer gewöhnlichen DGL. Kanonisch leitet sich ab vom Lateinischen \emph{canon} (\emph{Norm}, \emph{Regel}) und meint in der Mathematik eine konkrete Darstellungsweise aus vielen möglichen Darstellungsweisen.

\begin{definition}{Formen der Gleichung einer DGL}{DeqForms}
    Die Gleichung einer DGL $n$-ter Ordnung kann in verschiedene Formen umgestellt werden. Eine konkrete Form der Gleichung einer DGL heißt
    \begin{itemize}
        \item \textbf{explizit}, wenn die höchste Ableitung $y^{(n)}$ alleine auf der linken oder rechten Seite der Gleichung steht.
        \item \textbf{implizit}, wenn sie nicht explizit ist.
        \item \textbf{kanonisch}, wenn alle Glieder auf einer Seite der Gleichung stehen, also sich schreiben lässt als $\Phi(y,y',y'',y''',\dots,y^{(n)},x) = 0$
    \end{itemize}
\end{definition}

\begin{example}{DGL in expliziter und impliziter Form}{ExExplImplDeq}
    \begin{itemize}
        \item $\sin(y'+x^2) = y''$ ist explizit, da $y''$ alleine auf der rechten Seite steht.
        \item $\sin(y''+x^2) = y'$ ist implizit, da $y''$ innerhalb des Terms auf der linken Seite vorkommt.
        \item $\sin(y''+x^2) = y''$ ist implizit, da $y''$ auf beiden Seiten der Gleichung vorkommt.
        \item $y''+x=0$ ist implizit, umgeformt zu $y''=-x$ erhält man die explizite Form.
    \end{itemize}
\end{example}

\begin{example}{Kanonische Form einer DGL}{ExCanonDeq}
    \begin{itemize}
        \item Die DGL $y''+2y'-y=2x$ lässt sich umstellen zu $-y+2y'+y''-2x=0$. Das ist eine kanonische Form, sich lässt sich schreiben als $\Phi(y,y',y'',x)=-y+2y'+y''-2x=0$.
        \item Im Allgemeinen lässt sich jede DGL $\dots = \dots$ auf kanonische Form bringen, indem man durch Subtraktion die rechte Seite auf die linke Seite bringt.
    \end{itemize}
\end{example}

Mithilfe der Notation $\Phi(y,y',y'',y''',\dots,y^{(n)},x) = 0$, also dass wir die DGL als einen Term auffassen können, in dem das Argument, die Funktion und ihre Ableitungen vorkommen, können wir nun die nächste Eigenschaft einer DGL formulieren.

\begin{definition}{Homogenität}{HomogenousDeq}
    Eine DGL $n$-ter Ordnung $\Phi(y,y',y'',y''',\dots,y^{(n)},x) = 0$ heißt \textbf{homogen}, falls sie eine homogene Gleichung in der gesuchten Funktion $y$ und ihren Ableitung ist, das heißt, falls es ein $\alpha\in¸N_0$ gibt, sodass gilt:
    $$
        \forall t: \Phi(ty,ty',ty'',ty''',\dots,ty^{(n)},x) = t^\alpha \cdot \Phi(y,y',y'',y''',\dots,y^{(n)},x)
    $$
    $\alpha$ heißt dann \textbf{Grad der Homogenität}.
\end{definition}

Anders formuliert bedeutet die Gleichung in obiger Definition, dass wir, wenn wir alle $y$ mit einer Konstante multiplizieren, diese Konstante aus der gesamten Gleichung ausklammern können. Der Begriff der Homogenität ("Gleichförmigkeit") geht auf die Definition sogenannten \emph{homogener Funktionen} zurück. Eine Funktion heißt beispielsweise homogen, falls sich ihr Funktionswert verdoppelt, wenn man das Argument verdoppelt -- der Funktionswert also gleichmäßig mit dem Argument skaliert. Lineare Gleichungen der Form $y=mx$ sind homogen (vom Grad 1). Ein anderes Beispiel für homogene Funktionen sind Parabeln $y=ax^2$. Wenn wir das Argument verdoppeln, vervierfacht sich der Funktionswert. Allgemeiner können wir sagen, dass wenn wir das Argument mit $t$ multiplizieren (skalieren), der Funktionswert mit $t^2$ multipliziert wird: $f(tx)=t^2f(x)$. Noch allgemeiner heißt eine Funktion $f$ homogen vom Grad $\alpha$, wenn $f(tx) = t^\alpha f(x)$ gilt. Bei einer homogenen Funktionen steht also die Skalierung entlang der Abszisse (x-Achse) in einem Zusammenhang mit der Skalierung entlang der Ordinate (y-Achse). Wenn wir jetzt nochmal an die Definition für homogene DGL zurückdenken, so stellen wir fest, dass sie der Definition einer homogenen Funktion ähnelt, nur mit dem Unterschied, dass wir jetzt mehrere Argumente gleichzeitig skalieren.

\begin{example}{Überprüfung der Homogenität}{ExHomoDeq}
    \begin{itemize}
        \item $y''+xy=0$ ist homogen, denn wenn wir alle $y$ mit $t$ multiplizieren, erhalten wir $ty''+xty$, und hier können wir $t$ vollständig aus allen Summanden ausklammern: $t^1 \cdot (y''+xy)$. Der Grad der Homogenität ist wegen $t^1$ gleich 1.
        \item $y''+xy-3=0$ ist inhomogen. Wenn wir alle $y$ mit $t$ multiplizieren, erhalten wir $ty''+xty-3$. Da im letzten Summanden $-3$ kein $t$ vorkommt, können wir es nicht vollständig ausklammern. Anders betrachtet ist es uns also nicht möglich, in der Gleichung $ty''+xty-3 = t^n \cdot (y''+xy-3)$ einen Wert für $n$ zu finden, sodass daraus eine wahre Aussage wird.
        \item $y''+xy-3x=0$ ist inhomogen. Wenn wir alle $y$ mit $t$ multiplizieren, erhalten wir $ty''+xty-3x$. Da im letzten Summanden $-3x$ kein $t$ vorkommt, können wir es nicht vollständig ausklammern.
        \item $y''+xy^2=0$ ist inhomogen. Nach Multiplikation mit $t$ erhalten wir $ty''+xt^2y^2$. Im ersten Summanden kommt $t$ zur Potenz $1$ vor, im zweiten Summanden zur Potenz $2$. Wir können daher weder $t$ noch $t^2$ auf beiden Seiten ausklammern.
        \item $(y')^2+y^2=0$ ist homogen vom Grad $2$. Wir setzen an $(ty')^2+(ty)^2=t^2\cdot\left[(y')^2+y^2\right]$. Da $t^2$ in beiden Summanden vorkommt, konnten wir es vollständig ausklammern. Wegen $t^2$ ist der Grad der Homogenität $2$.
        \item $\sin(x)y'''+x^2y=0$ ist homogen vom Grad $1$, denn $\sin(x)ty'''+x^2 t y = t^1 \cdot \left(\sin(x)y''+x^2y\right)$.
        \item $\frac{y'}{y}+x=0$ ist homogen vom Grad $0$, denn $\frac{ty'}{ty}+x = t^0 \cdot \left(\frac{y'}{y}+x\right)$.
   \end{itemize}
\end{example}

Anhand der obigen Beispiel können wir einige Beobachtungen machen, welche uns das Überprüfen der Homogenität erleichtern. Kommt in der Funktionsgleichung ein Summand (ungleich 0) vor, in dem kein $y$ oder eine Ableitung von $y$ steht, ist die DGL in der Regeln inhomogen. Die einzige Ausnahme tritt auf, wenn die restlichen Terme Quotienten von $y$ sind, sodass sich ein Grad der Homogenität $0$ ergibt. Im folgenden werden wir hauptsächlich DGL betrachten, wo der Grad der Homogenität $1$ ist, daher ist diese Faustregel sehr hilfreich.  Weiterhin stellen wir noch fest, dass eine homogene DGL mit Grad der Homogenität $1$ vorliegt, wenn $y$ in allen weiteren Summanden ohne weitere Rechenoperation (Quadrieren, Sinusbildung, \dots) vorkommt und höchstens noch einen Vorfaktor hat. Solche Arten von DGL nennt man lineare homogene DGL, welche wir in einem der nächsten Abschnitte noch genauer betrachten werden.

Des Weiteren können wir noch eine Feststellung machen, wenn wir uns die ersten drei Beispiele noch einmal genauer anschauen: $y''+xy=0$ ist homogen, $y''+xy-3=0$ bzw. $y''+xy-3x=0$ jedoch nicht. Der Unterschied besteht im Summanden $-3$ beziehungsweise $-3x$. Während die letzten beiden DGL zwar nicht homogen sind, können wir sie zerlegen in eine Summe aus einer homogenen DGL ($y''+xy=0$) und einem Inhomogenitäts-Term ($-3$ beziehungsweise $-3x$).

\begin{definition}{Homogener und inhomogener Anteil}{HomInhomPart}
    Einige inhomogene DGL $n$-ter Ordnung lassen sich zerlegen in einen homogenen und einen inhomogenen Anteil:
    $$
        \Phi(y,y',y'',y''',\dots,y^{(n)},x) = 0 = \underbrace{H(y,y',y'',y''',\dots,y^{(n)},x)}_{Homogener Anteil} + \underbrace{F(x)}_{\text{Inhomogenität}}
    $$
    Ist dies möglich, so heißt $H$ der \textbf{inhomogene Anteil} der DGL und $H(y,y',y'',y''',\dots,y^{(n)},x) = 0$ die zur DGL \textbf{zugehörige homogene DGL}. Weiterhin heißt $F$ die \textbf{Inhomogenität} der DGL.
\end{definition}

Kleiner Hinweis zur Notation: Oft bringt man den inhomogenen Anteil $F(x)$ auch auf die rechte Seite der Gleichung und nennt dann $-F(x)$ die Inhomogenität:

$$
    H(y,y',y'',y''',\dots,y^{(n)},x) = \underbrace{-F(x)}_{\text{Inhomogenität}}
$$

\begin{example}{Zerlegung einer inhomogenen DGL}{ExHomInhomPart}
    \begin{itemize}
        \item Die zu $y''+xy=3x$ zugehörige homogene DGL lautet $y''+xy=0$, die Inhomogenität ist $3x$.
        \item Die DGL $y''+y = \sin(2 \pi x)$ beschreibt physikalisch ein schwingendes Pendel. Die zugehörige homogene DGL $y''y=0$ stellt eine freischwingendes Pendel dar, die Inhomogenität $\sin(2 \pi x)$ ergibt sich, wenn man das Pendel etwa mit einem Motor sinusförmig mit der Frequenz $1$ antreibt (erzwungene Schwingung).
        \item Die DLG $y''+xy=0$ ist bereits homogen und stellt damit selbst ihre zugehörige homogene DGL dar, die Inhomogenität ist $0$.
        \item Die DGL $\sin(y)=x$ ist inhomogen und lässt sich nicht in einen homogenen und inhomogenen Anteil zerlegen.
    \end{itemize}
\end{example}

Wie erwähnt sind lineare DGL von besonderem Interesse, nicht nur, da sie sich leicht lösen lassen, sondern auch zur Modellierung vieler Phänomene verwendet werden können. Die Linearität ist eine spezielle Form der Homogenität:

\begin{definition}{Lineare DGL}{LinDeq}
    Eine DGL $n$-ter Ordnung nennt man \textbf{linear}, wenn sie einen homogenen Anteil $H(y,y',y'',y''',\dots,y^{(n)},x)$ besitzt und dieser den Grad der Homogenität $1$ besitzt. Andernfalls heißt sie \textbf{nichtlinear}.
\end{definition}

Um eine Homogenität vom Grad $1$ zu erreichen, dürfen die gesuchte Funktion $y$ und alle ihre Ableitungen nur als lineare Summanden mit Vorfaktor (Koeffizient) in der DGL vorkommen. Wir man sich überlegen kann, ist die allgemeine Form solcher linearen DGL $n$-ter Ordnung:

$$
    \sum\limits_{i=0}^{n} a_i(x) y^{(i)}(x) = F(x)
$$

Wobei die Koeffizienten $a_i(x)$ beliebige, nur von $x$ abhängige Funktionen sind und $F(x)$ die Inhomogenität ist. Einige Beispiele für solche linearen DGL sind etwa:

\begin{itemize}
    \item $y'=0$
    \item $y'''-7y''+2y''+y'-5y=\sin(x)\cdot e^x$
    \item $\sin(x)x^2y'''+4e^x y = x$
\end{itemize}

Für nichtlineare DGL sind einige Beispiel etwa:

\begin{itemize}
    \item $\sin(y')=y$
    \item $\frac{y''}{y'} = y^2$
    \item $(y')^2+y=0$
\end{itemize}

\section{Direkte Integration}

In den verbleibenden Abschnitten wollen wir uns noch einige übliche Lösungsmethoden für DGL anschauen. Aufgrund der Knappheit der Zeit müssen wir uns dabei auf einige grundlegende Verfahren beschränken. Auf die Theorie der Lösung von Differentialgleichungssystem mit mehr als einer Funktion $y$ können wir hier gar nicht eingehen.

Die vermutlich einfachste Lösungsmethode einer DGL besteht in ihrer direkten Integration. Falls die DGL $n$-ter Ordnung in der Form $y^{(n)} = f(x)$ vorliegt, so müssen wir nur $n$-mal das unbestimmte Integral bilden und erhalten so die allgemeine Lösung der DGL mit $n$ Konstanten.

\begin{example}{Freier Fall}{FreeFall}
    Das Newtonsche Grundgesetz lautet $\ddot{x} = \frac{1}{m}\cdot F(x,t)$. Dabei ist $\ddot{x} = \ddn{2}{x}{t}$ die zweite Ableitung des Ortes eines Körpers $x$ nach der Zeit $t$, $F$ die auf den Körper wirkende Kraft und $m$ seine Masse. Durch Lösen dieser DGL erhält man die Bahnkurve des Körpers. Für einen frei fallenden Körper nahe der Erdoberfläche gilt $F(x,t)=m \cdot g$ mit der Fallbeschleunigung $g\approx 9.8\umss$. Damit lautet die Bewegungsgleichung für einen frei fallenden Körper $\ddot{x} = g$. Diese DGL können wir durch zweifache direkte Integration lösen. Die erste Integration liefert:
    $$
        v(t) = \dot{x}(t) = \int g \diff{x} = gt +v_0
    $$
    Dabei ist $v$ die Geschwindigkeit, welche per Definition die erste Ableitung des Ortes nach der Zeit ist. Die Integrationskonstante $v_0$ hat eine physikalische Bedeutung, nämlich die Anfangsgeschwindigkeit des Körpers zum Zeitpunkt $t=0$. Erneute Integration liefert:
    $$
        x(t) = \int gt + v_0 \diff{x} = \frac{1}{2} g t^2 + v_0 t + s_0
    $$
    Dabei ist $s_0$ der Anfangsort des Körpers zum Zeitpunkt $t=0$. Die DGL hat die Ordnung $2$, insgesamt haben wir wie erwartet die $2$ Integrationskonstanten $v_0$ und $s_0$ gewonnen. Durch Lösen der Bewegungsgleichung für den freien Fall haben wir somit die bekannte Formel für den freien Fall erhalten.
\end{example}

\section{Trennung der Variablen}

  Trennung der Variablen

\section{Variation der Konstanten}

  Variation der Konstanten

  y'+P(x)y=Q(x)

\section{Lineare Differentialgleichungen}

\subsection{Lösungsstruktur}

Linearkombination von n linear unabhängigen Fundamentallösungen  + 1 partikuläre Lösung

\subsection{Lineare DGL mit konstanten Koeffizienten}

Ansatz e^ax für homogene
Multiplikation mit x wenn mehrfache Nst.

partikuläre Lsg. ist Linearkomb. aller lin. unabhängigen Ableitungen der Inhomogenität
e^x,sin(x),cos(x),Polynome

Erwähnung Variation der Konst. für partik. Lsg.

Alternative: (D-1)(D-2)...(D+3) y = F(x) ist jeweils von der Form y'+P(x)y=Q(x)

Beispiel harm. Schwingung