\item 

\begin{enumerate}

\item Zuerst werden die Variablen getrennt:

$$y' = \frac{\d y}{\d x} = (1+y^2)x^3$$
$$\implies \frac{\d y}{1+y^2} = x^3 \d x$$

Anschließend wird integriert und umgestellt:

$$\implies \int \frac{\d y}{1+y^2} = \int x^3 \d x$$
$$\implies \arctan(y) = \frac{x^4}{4} + C$$
$$\implies y = \tan(\frac{x^4}{4} + C)$$

\item Wir setzen die Anfangsbedingung ein und bestimmen die Konstante $C$:

$$y(3) = 2 = \tan(\frac{3^4}{4} + C)$$
$$\implies \arctan(2) + \pi n= \frac{81}{4} + C, n \in \mathbb{Z}$$
$$\implies C = \arctan(2) - \frac{81}{4} + \pi n, n \in \mathbb{Z}$$

Für $n=7$ ergibt sich etwa $C \approx 2.85$ bzw. $C \approx 163^\circ$.

Die partikuläre Lösungsfunktion, welche die Anfangsbedingung erfüllt, lautet:

$$y(x) = \tan(\frac{x^4}{4} + \arctan(2) - \frac{81}{4})$$

(Es gibt nur eine Lösungsfunktion, da der Tangens periodisch mit Periode $p=\pi$ ist.)

\end{enumerate}

