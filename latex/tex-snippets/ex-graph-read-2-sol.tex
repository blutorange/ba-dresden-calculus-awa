\item Im gezeichneten Bereich des Graphen erkennen wir 4 (einfache) Nullstellen. Nach dem Fundamentalsatz der Algebra gibt es damit wenigstens 4 verschiedene Linearfaktoren, der Grad beträgt also wenigstens 4.

Beträgt der Grad genau 4, lässt sich $p(x)$ schreiben als

$$p(x) = a\cdot(x+\frac{3}{2})(x+1)(x+0)(x-\frac{1}{2})$$

mit einem noch zu ermittelnden $a \in \mathbb{R}$. Um $a$ zu finden, lesen wir ab, dass der Graph von $p$ durch den Punkt $(-\frac{1}{2}, -2)$ verläuft. Es muss also gelten:

$$-2 = p(-\frac{1}{2})$$

$$-2 = a\cdot(-\frac{1}{2}+\frac{3}{2})(-\frac{1}{2}+1)(-\frac{1}{2})(-\frac{1}{2}-\frac{1}{2})$$

$$-2 = a\cdot \frac{1}{4}$$

Wir erhalten $a=-8$ und damit

$$p(x) = -8\cdot(x+\frac{3}{2})(x+1)(x+0)(x-\frac{1}{2})$$
