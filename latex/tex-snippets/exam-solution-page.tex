\begin{center}
{\bf {\large Musterlösung}}
\end{center}

\begin{center}
{\bf {\large Klausur \submodule - \examdate}}
\end{center}

Jeder Anführungspunkt entspricht einem erteilten Bewertungspunkt. \\

Bei Aufgaben, welche einen Rechenweg erfordern, wird von der Maximalpunktzahl ausgegangen und für jeden fehlerhaften Rechenschritt (Bruch falsch gekürzt, nichtzutreffende Rechenregel angewandt, falsch zusammengefasst, Term falsch übernommen oder vergessen etc.) ein halber Punkt abgezogen. Das bedeutet auch:

\begin{itemize}
	\item Wird beispielsweise die Rechnung mit einem fehlerhaften Zwischenergebnis konsequent und korrekt fortgesetzt, gibt es keinen erneuten Punktabzug (Folgefehler).
	\item Werden beispielsweise 2 Rechenfehler gemacht, welche sich so aufheben, dass das Endergebnis korrekt ist, wird dennoch zweimal ein halber Punkt abgezogen.
\end{itemize}
