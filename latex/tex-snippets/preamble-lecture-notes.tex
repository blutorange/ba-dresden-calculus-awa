%%% DOCUMENTCLASS
%%%-------------------------------------------------------------------------------

\documentclass[
a4paper, % Stock and paper size.
11pt, % Type size.
% article,
% oneside,
onecolumn, % Only one column of text on a page.
% openright, % Each chapter will start on a recto page.
% openleft, % Each chapter will start on a verso page.
openany, % A chapter may start on either a recto or verso page.
]{memoir}

%%% PACKAGES
%%%------------------------------------------------------------------------------

\usepackage[utf8]{inputenc} % If utf8 encoding
\usepackage[T1]{fontenc}    %
\usepackage[german]{babel} % German if it pleases m'lord
\usepackage[final]{microtype} % Less badboxes
\usepackage{csquotes} % For german quotations
\usepackage{xcolor} % For colors
\usepackage{listings} % For colors

% \usepackage{kpfonts} %Font

\usepackage{amsmath,amssymb,mathtools,polynom} % Math
\usepackage{minted} % Programming
\usepackage[most]{tcolorbox} % For custom example styles


% \usepackage{tikz} % Figures
\usepackage{graphicx} % Include figures

%%% PAGE LAYOUT
%%%------------------------------------------------------------------------------

\setlrmarginsandblock{0.15\paperwidth}{*}{1} % Left and right margin
\setulmarginsandblock{0.2\paperwidth}{*}{1}  % Upper and lower margin
\checkandfixthelayout

%%% SECTIONAL DIVISIONS
%%%------------------------------------------------------------------------------

\maxsecnumdepth{subsection} % Subsections (and higher) are numbered
\setsecnumdepth{subsection}

\makeatletter %
\makechapterstyle{standard}{
  \setlength{\beforechapskip}{0\baselineskip}
  \setlength{\midchapskip}{1\baselineskip}
  \setlength{\afterchapskip}{8\baselineskip}
  \renewcommand{\chapterheadstart}{\vspace*{\beforechapskip}}
  \renewcommand{\chapnamefont}{\centering\normalfont\Large}
  \renewcommand{\printchaptername}{\chapnamefont \@chapapp}
  \renewcommand{\chapternamenum}{\space}
  \renewcommand{\chapnumfont}{\normalfont\Large}
  \renewcommand{\printchapternum}{\chapnumfont \thechapter}
  \renewcommand{\afterchapternum}{\par\nobreak\vskip \midchapskip}
  \renewcommand{\printchapternonum}{\vspace*{\midchapskip}\vspace*{5mm}}
  \renewcommand{\chaptitlefont}{\centering\bfseries\LARGE}
  \renewcommand{\printchaptertitle}[1]{\chaptitlefont ##1}
  \renewcommand{\afterchaptertitle}{\par\nobreak\vskip \afterchapskip}
}
\makeatother

\chapterstyle{standard}

\setsecheadstyle{\normalfont\large\bfseries}
\setsubsecheadstyle{\normalfont\normalsize\bfseries}
\setparaheadstyle{\normalfont\normalsize\bfseries}
\setparaindent{0pt}\setafterparaskip{0pt}

%%% FLOATS AND CAPTIONS
%%%------------------------------------------------------------------------------

\makeatletter                  % You do not need to write [htpb] all the time
\renewcommand\fps@figure{htbp} %
\renewcommand\fps@table{htbp}  %
\makeatother                   %

\captiondelim{\space } % A space between caption name and text
\captionnamefont{\small\bfseries} % Font of the caption name
\captiontitlefont{\small\normalfont} % Font of the caption text

\changecaptionwidth          % Change the width of the caption
\captionwidth{1\textwidth} %

%%% ABSTRACT
%%%------------------------------------------------------------------------------

\renewcommand{\abstractnamefont}{\normalfont\small\bfseries} % Font of abstract title
\setlength{\absleftindent}{0.1\textwidth} % Width of abstract
\setlength{\absrightindent}{\absleftindent}

%%% HEADER AND FOOTER
%%%------------------------------------------------------------------------------

\makepagestyle{standard} % Make standard pagestyle

\makeatletter                 % Define standard pagestyle
\makeevenfoot{standard}{}{}{} %
\makeoddfoot{standard}{}{}{}  %
\makeevenhead{standard}{\bfseries\thepage\normalfont\qquad\small\leftmark}{}{}
\makeoddhead{standard}{}{}{\small\rightmark\qquad\bfseries\thepage}
% \makeheadrule{standard}{\textwidth}{\normalrulethickness}
\makeatother                  %

\makeatletter
\makepsmarks{standard}{
\createmark{chapter}{both}{shownumber}{\@chapapp\ }{ \quad }
\createmark{section}{right}{shownumber}{}{ \quad }
\createplainmark{toc}{both}{\contentsname}
\createplainmark{lof}{both}{\listfigurename}
\createplainmark{lot}{both}{\listtablename}
\createplainmark{bib}{both}{\bibname}
\createplainmark{index}{both}{\indexname}
\createplainmark{glossary}{both}{\glossaryname}
}
\makeatother                               %

\makepagestyle{chap} % Make new chapter pagestyle

\makeatletter
\makeevenfoot{chap}{}{\small\bfseries\thepage}{} % Define new chapter pagestyle
\makeoddfoot{chap}{}{\small\bfseries\thepage}{}  %
\makeevenhead{chap}{}{}{}   %
\makeoddhead{chap}{}{}{}    %
% \makeheadrule{chap}{\textwidth}{\normalrulethickness}
\makeatother

\nouppercaseheads
\pagestyle{standard}               % Choosing pagestyle and chapter pagestyle
\aliaspagestyle{chapter}{chap} %

%%% NEW COMMANDS
%%%------------------------------------------------------------------------------

% Polynomial long division
\polyset{%
	style=C,
	delims={\big(}{\big)},
	div=:
}

% Differential operator
\newcommand{\diff}[1]{\:\mathrm{d}{#1}}
\newcommand{\pdd}[2]{\frac{\partial #1}{\partial #2}}
\newcommand{\pddn}[3]{\frac{\partial^{#1} #2}{\partial #3^{#1}}}
\newcommand{\dd}[2]{\frac{\mathrm{d}{#1}}{\mathrm{d}{#2}}}
\newcommand{\ddn}[3]{\frac{\mathrm{d}^{#1}{#2}}{\mathrm{d}{#3^{#1}}}}

% N-th root
% \nroot{3}{27}
\newcommand*{\nroot}[2]{\sqrt[\leftroot{-1}\uproot{2}#1]{#2}}
\newcommand*{\ncroot}[4]{\sqrt[\leftroot{#1}\uproot{#2}#3]{#4}}

% 2 component vector
% \tvect{1}{-1}
% \tvec{1}{-1}
\newcommand{\tvect}[2]{%
   \ensuremath{\Bigl(\negthinspace\begin{smallmatrix}#1\\#2\end{smallmatrix}\Bigr)}}
\newcommand{\tvec}[2]{%
    \ensuremath{\left(\negthinspace\begin{matrix}#1\\#2\end{matrix}\right)}}

% 3 component vector
% \rvect{1}{-1}{0}
% \rvec{1}{-1}{0}
\newcommand{\rvect}[3]{%
   \ensuremath{\Bigl(\negthinspace\begin{smallmatrix}#1\\#2\\#3\end{smallmatrix}\Bigr)}}
\newcommand{\rvec}[3]{%
    \ensuremath{\left(\negthinspace\begin{matrix}#1\\#2\\#3\end{matrix}\right)}}

% Long vector arrow
% \xshlongvec{ABC}

% German-style quotation marks %
\MakeOuterQuote{"}

% Number sets
\newcommand{\N}{\mathbb{N}}
\newcommand{\Z}{\mathbb{Z}}
\newcommand{\Q}{\mathbb{Q}}
\newcommand{\R}{\mathbb{R}}
\newcommand{\C}{\mathbb{C}}

\newcommand{\setzero}{\varnothing}

% Mention (small caps)
\newcommand{\mention}[1]{\textsc{#1}}

% Functions
\newcommand{\asin}[0]{\text{asin}}
\newcommand{\acos}[0]{\text{acos}}
\newcommand{\atan}[0]{\text{atan}}
\newcommand{\sgn}[0]{\text{sgn}}
\newcommand{\grad}[0]{\text{grad}}

% Scale
% Usage in math mode: \Scale[1.5]{...equation...} %
\newcommand*{\Scale}[2][4]{\scalebox{#1}{$#2$}}%

% Units
\newcommand{\um}{\text{m}}
\newcommand{\us}{\text{s}}
\newcommand{\ukm}{\text{km}}
\newcommand{\ukg}{\text{kg}}
\newcommand{\uh}{\text{h}}
\newcommand{\ukmh}{\frac{\ukm}{\uh}}
\newcommand{\umpers}{\frac{\um}{\us}}
\newcommand{\umss}{\frac{\ukm}{\us^2}}
\newcommand{\ukgss}{\frac{\ukg}{\us^2}}
\newcommand{\degrees}[1]{\SI{#1}{\degree}}

% Floor / ceil
\newcommand{\floor}[1]{\left\lfloor #1 \right\rfloor}
\newcommand{\ceil}[1]{\left\lceil #1 \right\rceil}

% Circle characters
\newcommand*\circled[1]{
    \tikz[baseline=(char.base)]{
        \node[shape=circle,draw,inner sep=2pt] (char) {#1};
    }
}


%%%% NEW ENVIRONMENTS

\newenvironment{memo}
{\begin{quote}%
	\large\bfseries}
{\end{quote}}

%%%% NEW THEOREMS

\newtcbtheorem{example}{Beispiel}{
	colback=blue!5!white,
	colbacktitle={blue!25!white},
	coltitle={black},
	fonttitle=\bfseries
}{ex}


\newtcbtheorem{definition}{Definition}{
	colback=yellow!20!white,
	colbacktitle={yellow!40!white},
	coltitle={black},
	fontupper=\slshape,
	fonttitle=\bfseries
}{def}

\newtcbtheorem{statement}{Aussage}{
	colback=red!5!white,
	colbacktitle={red!25!white},
	coltitle={black},
	fontupper=\slshape,
	fonttitle=\bfseries
}{stmt}

%%% TABLE OF CONTENTS
%%%------------------------------------------------------------------------------

\maxtocdepth{section} % Only parts, chapters and sections in the table of contents
\settocdepth{section}

\AtEndDocument{\addtocontents{toc}{\par}} % Add a \par to the end of the TOC

%%% INTERNAL HYPERLINKS
%%%-------------------------------------------------------------------------------

\usepackage{hyperref}   % Internal hyperlinks
\hypersetup{
pdfborder={0 0 1},      % Borders around internal hyperlinks
pdfauthor={Andre Wachsmuth} % author
}
\usepackage{memhfixc}   %

%%% Colors
%%%-------------------------------------------------------------------------------
\definecolor{LightGray}{rgb}{0.95,0.95,0.95}


%%% SOURCE CODE
%%%-------------------------------------------------------------------------------

% Default for all source code listings
\setminted{
	framesep=2mm,
	baselinestretch=1.0,
	bgcolor=LightGray,
	fontsize=\footnotesize,
	linenos
}

% specific languages
\newminted{c}{}
\newminted{js}{}
\newminted{text}{}

%%%% Named equations
%%%------------------------------------------------------------------------------
% Define Nequation environment for named equations.
% This is based on
%     http://tex.stackexchange.com/questions/128050/add-equation-name-underneath-equation-number
\usepackage{stackengine}
\def\stackalignment{r}
\def\stacktype{L}
\def\useanchorwidth{T}
% From page 3 of
%     http://tug.ctan.org/macros/latex/contrib/stackengine/stackengine.pdf
\def\Lstackgap{0.666666\baselineskip}
\newlength\eqshift
\setlength\eqshift{\widthof{)}}
\let\savetheequation\theequation
\newenvironment{Nequation}[1]%
{%
	\def\thecurrentname{#1}%
	\let\theequation\savetheequation
	\begin{equation}%
	\renewcommand\theequation
	{%
		\stackunder
		{\savetheequation}%
		{{\thecurrentname}\hspace{-\eqshift}}%
	}%
}%
{%
	\end{equation}%
	\addcontentsline{loe}{equation}{\protect\numberline{\theequation}\thecurrentname}%
	\let\theequation\savetheequation
	\ignorespacesafterend
}

