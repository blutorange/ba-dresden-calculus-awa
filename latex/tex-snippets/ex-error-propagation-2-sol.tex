\item Wir bezeichnen die Kantenlänge des Würfels mit $l$, wobei in dem Fall $l=10$ beträgt.

Die tatsächlichen Messwerte liegen in einem Bereich von $\Delta x = 0.1$ Einheiten um den Wert $10$, also im Bereich $[x-\Delta x, x + \Delta x]$.

Das Volumen eines Würfels in Abhängigkeit der Kantenlänge $l$ berechnet sich mittels der Funktion $V(l) = l^3$.

Für einen Würfel mit Kanten $l=10$ ist das Volumen etwa $1000$. Uns interessiert nun, welche Werte das Volumen annehmen kann,
wenn Messwerte zwischen $[9.9,10.1]$ für die Kantenlänge eingesetzt werden. Gesucht sind also die $V(x+\Delta x)$ und $V(x-\Delta x)$.

Betrachten wir zuerst $V(x+\Delta x)$. Für kleine $\Delta x$ können wir mithilfe der Ableitung schreiben:

$$V(x+\Delta x) \approx V(x) + \Delta x \cdot V'(x)$$

Für $V(x)=x^3$ kennen wir die Ableitung, nämlich $V'(x) = 3x^2$. Setzen wir dies ein, so erhalten wir:

$$V(x+\Delta x) \approx x^3 + \Delta x \cdot 3x^2$$

Somit haben wir eine Näherungsformel für die obere Grenze des Volumen erhalten. Mit den gegebenen Werten $x=10$ und $\Delta x = 0.1$ ergibt sich

$$V(10.1) \approx 10^3 + 0.1 \cdot 3 \cdot 10^2 = 1000 + 30  = 1030$$

Analog erhalten wir für die untere Grenze, indem wir statt Plus ein Minus setzen:

$$V(x-\Delta x) \approx x^3 - \Delta x \cdot 3x^2$$
$$V(9.9) \approx 10^3 - 0.1 \cdot 3 \cdot 10^2 = 1000 - 30  = 970$$

Antwortsatz: Die Grenzen für das Würfelvolumen betragen näherungsweise $[970, 1030]$.

