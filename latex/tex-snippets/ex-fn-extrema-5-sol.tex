\item Analog zur 1. Aufgabe.

\begin{enumerate}

\item Ansatz 1: Polynom 4. Grades mit 2 Nullstellen der Vielfachheit 2.

$f(x) = (x-3)^2(x-5)^2$

Wir überprüfen noch, ob es sich um Minima oder Maxima handelt. Ableiten:

$f'(x) = 2(x-3)(x-5)^2+2(x-3)^2(x-5) = 2(x-3)(x-5)*((x-5)+(x-3)) = 2(x-3)(x-5)(2x-8)$

(Die Ableitung hat also Nullstellen bei $3$ und $5$, zudem auch noch bei $4$).

Nochmal Ableiten und die Kandidaten für die Minima einsetzen:

$f''(x) = 2*((x-5)(2x-8)+(x-3)(2x-8)+(x-3)(x-5))$

$f''(3) = 2*((3-5)(6-8)) = 8 > 0$

$f''(5) = 2*((5-3)(10-8)) = 8 > 0$

$f''(4) = 2*((4-3)(4-5)) = -2 < 0$

Die Funktion $f(x) = (x-3)^2(x-5)^2$ hat also Minima bei $3$ und $5$ sowie ein Maxima bei $4$.

\item Ansatz 2: Cosinus-Funktion

$f(x) = -cos(\frac{\pi}{2}*(x-3))$

\end{enumerate}

