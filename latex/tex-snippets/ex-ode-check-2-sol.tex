\item Zum Einsetzen bilden wir zuerst die ersten beiden Ableitungen:

$$y=ax^2+bx+c$$
$$y'=2ax+b$$
$$y''=2a$$

Dies setzen wir in die DGL ein:

$$y''+3y=x^2+x$$
$$2a+3(ax^2+bx+c)=x^2+x$$
$$2a+3ax^2+3bx+3c=x^2+x$$
$$(3a)\cdot x^2+(3b)\cdot x+(2a+3c)=x^2+x$$

Koeffizientenvergleich bezüglich der Potenzen von $x$ liefert das lineare Gleichungssystem:

$$3a=1$$
$$3b=1$$
$$2a+3c=0$$

Es folgt sofort $a=\frac{1}{3}$ und $b=\frac{1}{3}$. Eingesetzt in die 3. Gleichung ergibt sich $\frac{2}{3}+3c=0$ und damit also $c=-\frac{2}{9}$.

Die Funktion $y(x) = \frac{1}{3}x^2+\frac{1}{3}x-\frac{2}{9}$ ist eine partikuläre Lösung der gegebenen DGL.
