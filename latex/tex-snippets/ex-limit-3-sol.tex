\item Zuerst stellen wir die rekursive Definition der Folge auf und betrachten dann den Grenzwert. Die ersten Glieder der Folge lauten:

$a_0 = \sin(x)$

$a_1 = \sin(x+\sin(x)) = \sin(x+a_0)$

$a_2 = \sin(x+\sin(x+\sin(x))) = \sin(x+a_1)$

Die rekursive Definition lautet also:

$a_0 = \sin(x)$

$a_{n+1} = \sin(x+a_n)$

Für dem Grenzwert $\lim_{n\to\infty} = a$ gilt:

$a = \sin(x+a)$

Zudem soll nach Aufgabenstellung auch gelten $a = 1$. Somit erhalten wir die Gleichung:

$1 = \sin(x+1)$

$\implies \arcsin(1) + 2\pi k = x + 1, k \in \mathbb{Z}$

$\implies x = \frac{\pi}{2} + 2\pi k - 1, k \in \mathbb{Z}$

Eine Lösung ist in Dezimalschreibweise etwa $\frac{\pi}{2}-1 \approx 0,57$.

