\item

Gegeben sind die beiden Funktionen mit ihren Ableitungen:

$f_1(x) = x^2$

$f_2(x) = -x^2+5x-\frac{17}{4}$

$f_1'(x) = 2x$

$f_2'(x) = -2x+5$

Wir betrachten die Tangente $t_1$ an $f_1$ an der Stelle $x_1$; und die Tangente $t_2$ an $f_2$ an der Stelle $x_2$. Wir wollen die 2 Stellen $x_1$ und $x_2$ so bestimmen, dass die Tangenten $t_1$ und $t_2$ gleich sind.

Insbesonders müssen also ihre Anstiege gleich sein:

$f_1'(x_1) = f_2'(x_2)$

$\implies 2x_1 = -2x_2+5$

$\implies x_1 = \frac{5}{2}-x_2$

Diese Beziehung nutzen wir später. Wir bilden die Tangente an $f_1$:

$t_1(x) = f_1(x_1) + f_1'(x_1)(x-x_1) = t_1(x) = x_1^2+2x_1(x-x_1)$

Da die Tangenten gleich sein sollen, muss der Funktionswert der Tangente $t_1$ im Punkt $x_2$ gleich dem Funktionswert der Funktion $f_2$ sein, also $t_1(x_2)$ = $f_2(x_2)$. Wir betrachten zuerst $t_1(x_2)$:

$t_1(x_2) = x_1^2+2x_1(x_2-x_1) =-x_1^2+2 x_1 x_2$

Da nun gilt $x_1 = \frac{5}{2}-x_2$:

$t_1(x_2) = -(\frac{5}{2}-x_2)^2+2(\frac{5}{2}-x_2)x_2 =-3x_2^2+10x_2-\frac{25}{4}$

Wie erwähnt muss gelten $t_1(x_2)$ = $f_2(x_2)$:

$t_1(x_2)$ = $f_2(x_2)$

$\implies -3x_2^2+10x_2-\frac{25}{4} = -x_2^2+5x_2-\frac{17}{4}$

$\implies -2x_2^2+5x_2-2 = 0$

$\implies x_2^2-\frac{5}{2}x_2+1 = 0$

$\implies x_2 = \frac{5}{4} \pm \sqrt {\frac{25}{16}-1} = \frac{5}{4} \pm \sqrt {\frac{9}{16}} = \frac{5}{4} \pm \frac{3}{4}$

$\implies x_2 = \frac{1}{2} \lor x_2 = 2$

Da $x_1 = \frac{5}{2}-x_2$, folgt auch $x_2 = \frac{1}{2} \lor x_2 = 2$. Es gibt also zwei mögliche Tangenten, diese lauten:

$t_1(x) = \frac{1}{4} + (x-\frac{1}{2})$

$t_2(x) = 4 + 4(x-2)$

