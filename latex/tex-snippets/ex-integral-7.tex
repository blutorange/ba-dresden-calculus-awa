\item (*) Die Fakultät $a_z = (z-1)!$ einer natürlichen Zahl  $z$ ist charakterisiert durch die rekursive Eigenschaft $a_{z+1} = z\cdot a_z$. Für nichtganzzahlige Werte $z$ wird sie erweitert zu der sogennanten Gamma-Funktion, diese ist definiert als $\Gamma(z) = \int\limits_0^\infty x^{z-1} e^{-x} \d x$. Zeigen Sie mittels partieller Integration, dass auch die Gammafunktion diese rekursive Eigenschaft erfüllt, d.h. dass $\Gamma(z+1) = z \cdot \Gamma(z)$ gilt!
