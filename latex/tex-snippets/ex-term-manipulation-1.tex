\item Eine Funktion $f(x)$ heißt achsensymmetrisch bzgl. der vertikalen Achse $x=k, k\in\mathbb{R}$, falls die folgende Bedingung erfüllt ist:

$$\forall x \in \mathbb{D}: f(k-x) = f(k+x)$$

\begin{enumerate}

\item (4P) Untersuchen Sie anhand dieser Bedingung, ob $f(x) = x^2+9-6x$ symmetrisch ist bzgl. der Achse $x=3$!

\item  (3P) Es soll untersucht werden, zu welcher Achse $x=k$ die Funktion $f(x) = x^2+4x+4$ achsensymmetrisch ist. Die Auswertung der Bedingung ergab:

$$k^2-2kx+4k+x^2-4x+4 = k^2 + 2kx + 4k + x^2 + 4x +4$$

Welchen Wert hat k? (Hinweis: Zusammenfassen nach Potenzen von x, Koeffizientenvergleich)

\end{enumerate}

