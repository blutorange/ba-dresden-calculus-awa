\item
Lösen Sie die lineare, inhomogene DGL $y'+3y = x$. Geben Sie die partikuläre Lösung für das Anfangswertproblem $y(0)=4$ an. Anleitung:
\begin{enumerate}
\item Bestimmen Sie die allgemeine Lösung $y_h$ der homogenen DGL $y_h'+3y_h=0$.
\item Um eine partikuläre Lösung $y_p$ zu finden, nutzen Sie den Ansatz $y_p = a\cdot x +b$ ($a,b\in\mathbb{R}$). Setzen Sie diesen Ansatz in die inhomogene DGL $y_p'+3y_p-x=0$ ein und bestimmen Sie hieraus die Konstanten $a$ und $b$.
\item Die allgemeine Lösung der inhomogenen DGL ergibt sich dann als die Summe $y_h+y_p$.
\item Um die partikuläre Lösung zu finden, setzen Sie die Anfangsbedingung in die allgemeine Lösung ein, um den Wert der Konstanten zu ermitteln.
\end{enumerate}

Prüfen Sie, ob das Verfahren "Variation der Konstanten" das gleiche Ergebnis liefert!
