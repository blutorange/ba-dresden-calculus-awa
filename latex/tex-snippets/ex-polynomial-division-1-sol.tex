\item Der Nenner $x^2-x$ darf nicht $0$ ergeben. Also sind $0$ und $2$ vom Definitionbereich ausgeschlossen: $\mathbb{D} = \mathbb{R} \setminus \{0,2\}$. Nun Polynomdivision durchführen:

\[\polylongdiv{x^2+x-6}{x^2-2*x}\]

Somit ist $p(x) = 1$ und $r(x) = \frac{3x-6}{x^2-2x}$ und es gilt die Zerlegung:

$$ f(x) = p(x) + r(x) = 1 + \frac{3x-6}{x^2-2x} $$

Nun Partialbruchzerlegung ansetzen:

$$ \frac{3x-6}{x^2-2x} = \frac{3x-6}{x\cdot(x-2)} = \frac{A}{x} + \frac{B}{x-2} $$

$$ \frac{3x-6}{x^2-2x} = \frac{A(x-2)+Bx}{x(x-2)} = \frac{(A+B)x-2A}{x(x-2)} $$

Wir finden $A+B = 3$ und $-2A=-6$, also $A=3$ und $B=0$. Somit lässt sich $f$ schreiben als:

$$ f(x) = 1 + \frac{3}{x} $$

Daraus erkennen wir schließlich, dass bei der Definitionlücke $x=0$ eine Polstelle vorliegt.

Anmerkung: Bei $x=2$ handelt es sich um eine hebbare Definitionslücke:\\
$$\lim\limits_{x\to 2} f(x) = 1+\frac{3}{2} = \frac{5}{2}$$
