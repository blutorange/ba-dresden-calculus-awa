\item Ohne Beschränkung der Allgemeinheit habe das ursprüngliche große Quadrat den Flächeninhalt 1 (sonst entsprechend skalieren). Wir betrachten die Folge $x_n$ ($n=0,1,2,...$) des im Schritt $n+1$ entfernten Flächeninhalts (s. Skizze):

$$x_0 = \frac{1}{9}$$
$$x_{n+1} = \frac{8}{9} a_n$$

Damit haben wir die rekursive Angabe. Es handelt sich um eine geometrische Folge, die explizite Form ist:

$$x_n = \frac{1}{9} \cdot (\frac{8}{9})^n$$

Damit ist der verbleibende Flächeninhalt:

$$A_n = 1 - \sum_{n=0}^{\infty} {x_n} = 1 - \frac{1}{9} \cdot \sum_{k=0}^{n} {(\frac{8}{9})^k}$$

Für das langfristige Verhalten betrachten wir den Grenzwert (Formel für geometrische Reihe):

$$ A_\infty = 1 - \frac{1}{9} \cdot \sum_{n=0}^{\infty} {(\frac{8}{9})^n} = 1 - \frac{1}{9} \cdot \frac{1}{1-\frac{8}{9}} = 0$$

Somit wird also langfristig der gesamte Flächeninhalt ausgeschnitten.

\begin{figure}[ht]
	\centering
	\includegraphics[width=0.8\textwidth]{plot_4_1.png}
	\caption{Skizze zu Aufgabe 1. Farbig hinterlegt sind die entfernten Bereiche.}
	\label{fig1}
\end{figure}

