\item $p(x) = a\cdot \left(x-x_0\right)^n+k$

\begin{enumerate}

\item Der Grad ist $n$. Der höchste Koeffizient ist $a_n = a$.

\item Zuerst Skalierung entlang y-Richtung mit Faktor $a$, dann Verschiebung um $x_0$ nach rechts und um $k$ nach oben.
\item $p(x) = 2(x+1)^3+4$

\item Der Scheitelpunkt von $x^2$ liegt im Ursprung. 1 Einheit rechts davon ist die Funktion um 1 Einheit angestiegen. Wir müssen also zuerst entlang y-Richtung mit $a=4$ skalieren (strecken). Um den Scheitelpunkt auf $(1,3)$ zu bringen, müssen wir um $x_0=1$ nach rechts und um $k=3$ nach oben verschieben:

$$p(x) = 4(x-1)^2+3$$

\item Damit 2 Polynome gleich sein können, müssen sie den gleichen Grad haben, es ist also $n=2$. Zuerst ausmultiplizieren:

$$a(x-x_0)^2+k =(a)\cdot x^2+(-2ax_0)\cdot x + (ax_0^2+k)$$

Koeffizientenvergleich liefert $4=a$, $-7=-2ax_0$ und $3=ax_0^2+k$. Die erste Gleichung in die zweite eingesetzt ergibt $-7=-2\cdot 4 \cdot x_0$, also $x_0=\frac{7}{8}$. Eingesetzt in die dritte Gleichung ergibt sich $k=3-4\cdot\frac{49}{64}=3-\frac{49}{16}=-\frac{1}{16}$.

Wir können somit schreiben:

$$p(x)=4x^2-7x+3 = 4\cdot \left(x-\frac{7}{8}\right)^2-\frac{1}{16}$$

Hinweis: Alternativ kann man auch eine quadratische Ergänzung durchführen.

\end{enumerate}

