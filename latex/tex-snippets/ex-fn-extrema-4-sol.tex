\item Es sind verschiedene Arten von Funktionen möglichen. Im Folgenden zwei Ansätze, eine solche Funktionsgleichung aufzustellen:

\begin{enumerate}

\item Ansatz 1: Kubische Funktion. An Minima und Maxima ist die Ableitung 0. Die Ableitung der kubischen Funktion ist also eine Parabel mit den Nullstellen 1 und 2:

$f'(x)=-(x-1)\cdot(x-2)$

Ausmultiplizieren und ableiten:

$f'(x)=-(x-1)\cdot(x-2)= -(x^2-2x-x+2)=-x^2+3x-2$.

$f''(x) = -2x+3$

Da $f''(1) = -2*1+3 = 1 > 0$ liegt bei $x=1$ ein Minima vor und wegen $f''(2) = -2*2+3 = -1 < 0$ bei $x=2$ ein Maxima.

Nun integrieren, um die kubische Funktion zu erhalten:

$f(x) = -\frac{1}{3}x^3+\frac{3}{2}x^2-2x+C$

Für $C$ kann eine beliebige reele Zahl eingesetzt werden.

\item Ansatz 2: Sinus-Funktion

$f(x)=-\sin(\pi \cdot (x-\frac{1}{2}))$

Die Periode obiger Sinus-Funktion ist $\frac{2\pi}{\pi} = 2$. Sie zudem um $1/2$ nach rechts verschoben und an der x-Achse gespiegelt, hat damit also ein Minimum bei $1$ und ein Maxima bei $2$.

\end{enumerate}

