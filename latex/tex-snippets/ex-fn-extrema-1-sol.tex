\item

Kandidaten für den höchsten und kleinsten Wert sind die Grenzen des Intervalls $[-2;1]$ sowie die lokalen Extrempunkte.

$f'(x) = 3x^2-1 = 0$

$\implies 3x^2 = 1$

$\implies x_{1,2} = \pm \frac{\sqrt{3}}{3}$

Es gilt $1 = \sqrt{1} < \sqrt{3} < \sqrt{4} = 2$, also $\frac{\sqrt{3}}{3} \in [\frac{1}{3}; \frac{2}{3}]$, die Kandidaten für Extrempunkte $x_{1,2}$ liegen also im betrachteten Intervall.

Nun rechnen wir die Funktionswerte an den Intervallgrenzen und an den möglichen Extremstellen aus:

$f(-2) = (-2)^3 + 2 = -6$

$f(1) = 1^3 - 1 = 0$

$f(\frac{\sqrt{3}}{3}) = 3*\frac{\sqrt{3}}{27} - \frac{\sqrt{3}}{3} = -\frac{2}{9}\sqrt{3}$

$f(-\frac{\sqrt{3}}{3}) = -3*\frac{\sqrt{3}}{27} + \frac{\sqrt{3}}{3} = \frac{2}{9}\sqrt{3}$

Nach obiger Abschätzung ergibt sich wieder $\frac{2}{9}\sqrt{3} \in [\frac{2}{9};\frac{4}{9}]$, somit ist der kleinste Wert $-6$ und der größte Wert $\frac{2}{9}\sqrt{3}$.

