\item Prüfen Sie, ob die folgende DGL homogen sind ! Falls nichts, geben Sie den homogenen und den inhomogenen Anteil an!

Bemerkung: Eine gewöhnliche DGL n-ter Ordnung der Form $\phi(y, y', ..., y^{(n)}, x)=0$ heißt homogen vom Grad $k$, falls es ein $k \in \mathbb{R}$ gibt, sodass für alle $t > 0$ gilt:

$$\phi(t \cdot y, t \cdot y', ..., t \cdot y^{(n)},x) = t^k \cdot \phi(y,y',...,y^{(n)},x)$$

$k$ heißt dann Grad der Homogenität.

\begin{enumerate}
\item $y'=y$
\item $y'=y+\sin(x)$
\item $x^2y''=2y'-y$
\item $0=\frac{y''+2y}{y'''-3y}$
\item $\frac{y'}{y}=(x+\frac{1}{y})$
\item $(y'')^2=\ln(x)(yy')$
\item $\sqrt{y'+y}=0$
\end{enumerate}

