\item Monotonie, Symmetrie, Periodizität

\begin{enumerate}

\item Die Funktion ist periodisch (da es sich um eine um zwei Einheiten nach oben verschobene und um den Faktor 3 in x-Richtung gestreckte Funktion handelt). Damit ist sie periodisch, aber nicht (streng) monoton. Sie ist weder gerade noch ungerade.

\item Funktion ebenfalls periodisch und nicht (streng) monoton; Vereinfachung der Funktion mit der Additionsformel für die Sinusfunktion: 
$\sin(A+B)=\sin A \cos B+\cos A \sin B$. Daraus folgt: $\sin(3x+\pi)=\sin 3x \cos \pi + \cos x \sin \pi = - \sin 3x $; also ungerade.

\item Funktion nicht (streng) monoton, nicht periodisch, nicht ungerade; aber gerade.

\item Funktion ist gerade, nicht ungerade, nicht (streng) monoton; periodisch (selbe Periodizität wie die Cosinus-Funktion)

\item Funktion gerade, nicht ungerade, nicht (streng) monoton, nicht periodisch

\end{enumerate}

