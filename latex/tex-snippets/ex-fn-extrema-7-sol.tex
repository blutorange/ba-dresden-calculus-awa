\item

\begin{enumerate}

\item Es handelt sich um Paare von reellen Zahlen. Der Definitionsbereich lautet $\mathbf{D} = (0,\infty)\times\mathbb{R} = \lbrace (x,y) \in \mathbb{R}^2 | x > 0 \rbrace$

\item Wir berechnen zuerst die Ableitung:

$$f_x(x,y) = y^2 (\ln(x)+1) - \cos(y)$$
$$f_y(x,y) = 2x\ln(x)y +x\sin(y)$$

Nun können wir die gegebene Stelle einsetzen:

$$f_x(e,\pi) = \pi^2 (\ln(e)+1) - \cos(\pi) = 2\pi^2+1$$

Notwendige Bedingung für eine Extremstelle ist das Verschwinden der ersten Ableitung, aber $f_x(e, \pi) = 2\pi^2+1 \ne 0$, also liegt keine Extremstelle vor. (Alternativ hätte man auch $f_y(e,\pi)=2e\pi \ne 0$ berechnen können).

\end{enumerate}

