\item Der Nenner $x^3-4x$ besitzt die Nullstellen $-2, 0, 2$. Damit kann eine Partialbruchzerlegung angesetzt werden:

$$\frac{x^2+2}{x^3-4x} = \frac{A_1}{x-2} + \frac{A_2}{x} + \frac{A_3}{x+2}$$

Es ergibt sich $A_1$ = $\frac{3}{4}$, $A_2=-\frac{1}{2}$, und $A_3 = \frac{3}{4}$.

Damit lässt sich das Integral sofort angeben (Betragsstriche nicht vergessen!)

$$ \int f(x)\d x = \frac{3}{4} \ln(|x-2|) - \frac{1}{2} \ln(|x|) + \frac{3}{4}\ln(|x+2|)$$
