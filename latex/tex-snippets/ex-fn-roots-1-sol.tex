\item Durch den Fundamentalsatz der Algebra wissen wir, dass die Vielfachheit der Nullstelle $2$ und $3$ jeweils gerade ist. Wir setzen an:

$f(x)=c \cdot (x-2)^2(x-3)^2$

Die Konstante $c \in \mathbb{R}$ ist noch so zu bestimmen, dass $f(0) = 1$ gilt:

$f(0) \stackrel{!}{=} 1 =c \cdot (0-2)^2(0-3)^2$

$\implies 1 = c \cdot 4 \cdot 9$

$\implies c=\frac{1}{36}$

Weitere Lösungen: Andere gerade Zahlen im Exponenten bei den Linearfaktoren zu den Nullstellen $2$ und $3$.

