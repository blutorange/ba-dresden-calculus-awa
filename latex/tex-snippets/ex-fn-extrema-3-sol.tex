\item

Hinweis: 

$\frac{d}{dx}\tan(x) = \frac{1}{\cos^2(x)}$

$\frac{d}{dx}\text{atan}(x) = \frac{1}{x^2+1}$

Unter Kentniss der Winkelfunktionen erhält man:

\begin{itemize}
\item $\sin(x)$: Minima bei $\frac{3}{2}\pi+2\pi{n}$, Maxima bei $\frac{\pi}{2}+2\pi{n}$, Wendepunkte bei $\pi{n}$, $n \in \mathbb{Z}$
\item $\cos(x)$: Minima bei $\pi+2\pi{n}$, Maxima bei $2\pi{n}$, Wendepunkte bei $\frac{\pi}{2}+\pi{n}$, $n \in \mathbb{Z}$
\item $\tan(x)$: keine (lokalen) Minima oder Maxima, Wendepunkte bei $\pi{n}$, $n \in \mathbb{Z}$
\item $\text{atan}(x)$: keine (lokalen) Minima oder Maxima, Wendepunkt bei $x=0$
\end{itemize}

