% Polynomial long division
\polyset{%
	style=C,
	delims={\big(}{\big)},
	div=:
}

% Differential operator
\newcommand{\diff}[1]{\:\mathrm{d}{#1}}
\newcommand{\pdd}[2]{\frac{\partial #1}{\partial #2}}
\newcommand{\pddn}[3]{\frac{\partial^{#1} #2}{\partial #3^{#1}}}
\newcommand{\dd}[2]{\frac{\mathrm{d}{#1}}{\mathrm{d}{#2}}}
\newcommand{\ddn}[3]{\frac{\mathrm{d}^{#1}{#2}}{\mathrm{d}{#3^{#1}}}}

% N-th root
% \nroot{3}{27}
\newcommand*{\nroot}[2]{\sqrt[\leftroot{-1}\uproot{2}#1]{#2}}
\newcommand*{\ncroot}[4]{\sqrt[\leftroot{#1}\uproot{#2}#3]{#4}}

% 2 component vector
% \tvect{1}{-1}
% \tvec{1}{-1}
\newcommand{\tvect}[2]{%
   \ensuremath{\Bigl(\negthinspace\begin{smallmatrix}#1\\#2\end{smallmatrix}\Bigr)}}
\newcommand{\tvec}[2]{%
    \ensuremath{\left(\negthinspace\begin{matrix}#1\\#2\end{matrix}\right)}}

% 3 component vector
% \rvect{1}{-1}{0}
% \rvec{1}{-1}{0}
\newcommand{\rvect}[3]{%
   \ensuremath{\Bigl(\negthinspace\begin{smallmatrix}#1\\#2\\#3\end{smallmatrix}\Bigr)}}
\newcommand{\rvec}[3]{%
    \ensuremath{\left(\negthinspace\begin{matrix}#1\\#2\\#3\end{matrix}\right)}}

% Long vector arrow
% \xshlongvec{ABC}

% German-style quotation marks %
\MakeOuterQuote{"}

% Number sets
\newcommand{\N}{\mathbb{N}}
\newcommand{\Z}{\mathbb{Z}}
\newcommand{\Q}{\mathbb{Q}}
\newcommand{\R}{\mathbb{R}}
\newcommand{\C}{\mathbb{C}}

\newcommand{\setzero}{\varnothing}

% Mention (small caps)
\newcommand{\mention}[1]{\textsc{#1}}

% Functions
\newcommand{\asin}[0]{\text{asin}}
\newcommand{\acos}[0]{\text{acos}}
\newcommand{\atan}[0]{\text{atan}}
\newcommand{\sgn}[0]{\text{sgn}}
\newcommand{\grad}[0]{\text{grad}}

% Scale
% Usage in math mode: \Scale[1.5]{...equation...} %
\newcommand*{\Scale}[2][4]{\scalebox{#1}{$#2$}}%

% Units
\newcommand{\um}{\text{m}}
\newcommand{\us}{\text{s}}
\newcommand{\ukm}{\text{km}}
\newcommand{\ukg}{\text{kg}}
\newcommand{\uh}{\text{h}}
\newcommand{\ukmh}{\frac{\ukm}{\uh}}
\newcommand{\umpers}{\frac{\um}{\us}}
\newcommand{\umss}{\frac{\ukm}{\us^2}}
\newcommand{\ukgss}{\frac{\ukg}{\us^2}}
\newcommand{\degrees}[1]{\SI{#1}{\degree}}

% Floor / ceil
\newcommand{\floor}[1]{\left\lfloor #1 \right\rfloor}
\newcommand{\ceil}[1]{\left\lceil #1 \right\rceil}

% Circle characters
\newcommand*\circled[1]{
    \tikz[baseline=(char.base)]{
        \node[shape=circle,draw,inner sep=2pt] (char) {#1};
    }
}
