\item p-q-Formel für (b) und (c), (d) ist ein biquadratische Gleichung und kann mittels $z = \lambda^2$ auf eine quadratische reduziert werden.

\begin{enumerate}
\item $\lambda + 2 = 0 \implies \lambda_1 = -2$
\item $\lambda^2-3\lambda=0 \implies \lambda_1 = 0, \lambda_2 = 3$
\item $2\lambda^2+\frac{10}{3}\lambda-\frac{4}{3}=0 \implies \lambda_1 = -2, \lambda_2 = \frac{1}{3}$
\item $2\lambda^4+2\lambda^2-12=0 \implies \lambda_1=-\sqrt{2}, \lambda_2=\sqrt{2}, \lambda_3=-\sqrt{3}j, \lambda_4=\sqrt{3}j \rbrace$
\item $\lambda^5-\lambda = 0 \implies \lambda_1 = 0, \lambda_2 = -1, \lambda_3 = 1, \lambda_4 = -j, \lambda_5 = j$
\item Homogene Lösung ist aus Aufgabe (a) bekannt. Partikuläre Lösung mit Ansatz $y_p=1$, $y_p'=0$. Einsetzen in DGL liefert $0+2k=1$, also $k=\frac{1}{2}$. Somit ist die allgemeine Lösung $y(x) = C_1 e^{-2x} + \frac{1}{2}$. 
\end{enumerate}

Die allgemeine Lösung ergibt sich nach dem Schema $y(x) = \sum\limits_{i=1}^{n} \left( C_n \cdot e^{\lambda_n x} \right)$. Also etwa für (b):

$$ y(x) = C_1 + C_2 e^{3x}$$

