\item

\begin{enumerate}

\item Wir klammern zuerst aus: 

$$f(x) = \ln(2x-2) = \ln(2(x-1))$$

Daraus erkennen wir, dass die Funktion zuerst entlang der x-Achse mit dem Faktor 2 skaliert wird (Stauchung). Anschließend wird sie ebenfalls entlang der x-Achse um $1$ Einheit nach rechts verschoben.

Die Nullstelle von $\ln(x)$ befindet sich bei $1$. Durch die Stauchung rückt sie auf $0{,}5$ und durch die anschließende Verschiebung auf $1{,}5$.

\item Der Graph muss aussehen wie eine Logarithmusfunktion, die wie oben beschrieben verschoben und skaliert ist:

\begin{itemize}
\item Graph nähert sich der Vertikalen $x=1$, ohne sie zu schneiden.
\item Für $x\le 1$ ist die Funktion nicht definiert.
\item Graph ist monoton steigend mit Biegung im Uhrzeigersinn
\end{itemize}

Ein konkreter Punkt, durch den die Funktion verläuft, ist etwa $f(1.5) = \ln(2*1.5-2) = \ln(1) = 0$.

\item Die Funktion $f$ ist monoton steigend, genauer: streng monoton steigend.

\end{enumerate}

