\item

\begin{enumerate}

\item Wir vergleichen die gegebenen Terme und erhalten (unter Beachtung der Vorzeichen): $r_1 = -\pi$, $r_2 = 2\pi$, $a = -\pi$, $b = -2\pi^2$. Graphisch sind $r_{1,2}$ die Nullstellen des Polynoms $p_2$, dies erkennt man schnell, indem man $r_1$ bzw. $r_2$ in die faktorisierte Darstellung einsetzt.

\item Wir fassen zuerst nach Potenzen von $x$ zusammen:

\begin{itemize}
\item $p_2(x) = x^2 + \underline{a}x + \underline{b} = (x-r_1)(x-r_2)$
\item $= (x^2+r_1r_2-r_1x-r_2x)$ (Ausmultiplizieren)
\item $= x^2 \underline{-(r_1+r_2)}x+\underline{(r_1r_2)}$ (Zusammenfassen)
\end{itemize}

Ein Koeffzientenvergleich der untersrichenen Koeffizienten ergibt die behaupteten Werte für $a$ und $b$ und beweist damit die Aussage.

\end{enumerate}
