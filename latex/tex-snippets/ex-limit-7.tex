\item Achilles und die Schildkröte veranstalten ein Wettrennen. Die Schildkröte bewegt sich fort mit $5\frac{\text{km}}{\text{h}}$, Achilles mit $25\frac{\text{km}}{\text{h}}$. Die Schildkröte hat einen Vorsprung von $\Delta x_0 = 100\text{km}$. Zum Überwinden dieses Vorsprungs benötigt Achilles die Zeit $\Delta t_0$. Die Schildkröte aber ist weitergelaufen und hat nach Ablauf dieser Zeit einen neuen Vorsprung $\Delta x_1$. Zum Aufholen dieses Vorsprungs benötigt Achilles nun die Zeit $\Delta t_1$. 
\begin{enumerate}
\item Geben Sie tabellenförmig die Werte $\Delta x_n$ und $\Delta t_n$ für $n=0,1,2,3,4$ an!
\item Finden Sie eine allgemeine Formel für $\Delta x_n$ und $\Delta t_n$!
\item Lösen Sie Zenons Paradoxon durch Ermitteln der Grenzwerte $x = \Delta x_0 + \Delta x_1 + \Delta x_2 + ...$ und $t = \Delta t_0 + \Delta t_1 + \Delta t_2 + ...$!
\item Bestätigen Sie das Ergebnis, indem Sie die Bewegungsgleichungen $x_A(t)$ und $x_S(t)$ für beide Läufer aufstellen und den Überholungspunkt durch Gleichsetzen ermitteln!
\end{enumerate}
