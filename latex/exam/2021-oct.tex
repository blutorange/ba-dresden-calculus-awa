\newcommand{\examdate}{13.10.2021}
\newcommand{\lecturer}{Andre Wachsmuth}
\newcommand{\modulecode}{3MI-MATHE-10}
\newcommand{\module}{Algebra/Analysis}
\newcommand{\submodule}{Analysis}
\newcommand{\termnumber}{1}
\newcommand{\allottedtime}{40 min}
\newcommand{\permittedtools}{1 handbeschriebenes A4-Blatt}
\newcommand{\scoretable}{
	\begin{tabularx}{\textwidth}{l|Y|Y|Y|Y|Y|Y}
		Aufgabe        & 1 & 2 & 3 & 4  & 5 & Summe \\ [1ex] \hline
		Soll-Punktzahl & 8 & 8 & 8 & 8  & 8 &       \\ [3ex]
		Ist-Punktzahl  &   &   &   &    &    &      ¸\\ [3ex]
	\end{tabularx}
}

\documentclass[12pt]{article}
\usepackage[a4paper]{geometry}
\usepackage[pdftex]{hyperref}
\usepackage[german]{babel}
\usepackage[utf8]{inputenc}
\usepackage{csquotes}
\usepackage{amssymb}
\usepackage{graphicx}
\usepackage{multicol}
\usepackage{amsmath}
\usepackage{enumitem}
\usepackage{tabularx}
\usepackage{vwcol}
\usepackage{fancyhdr}
\usepackage{indentfirst}
\usepackage{polynom}

\geometry{
	headheight=14px,
	left=2.54cm,
	right=2.54cm,
	bottom=2cm,
	top=2cm
}

% For horizontally centering text in Y column
\newcolumntype{Y}{>{\centering\arraybackslash}X}

\setlength{\parindent}{0cm}

\setlength{\marginparsep}{1 cm}
\setlength{\topmargin}{-0.6in}
\setlength{\textheight}{9.5in}
\pagestyle{fancy}

\fancypagestyle{firstpage}{ %
	\lhead{\bf Staatliche Studienakademie Dresden\\
		Studienrichtung: Informationstechnologie}
	\rhead{\bf Datum: \examdate}
}

% Polynomial long division
\polyset{%
	style=C,
	delims={\big(}{\big)},
	div=:
}

% Differential operator
\newcommand{\diff}[1]{\:\mathrm{d}{#1}}
\newcommand{\pdd}[2]{\frac{\partial #1}{\partial #2}}
\newcommand{\pddn}[3]{\frac{\partial^{#1} #2}{\partial #3^{#1}}}
\newcommand{\dd}[2]{\frac{\mathrm{d}{#1}}{\mathrm{d}{#2}}}
\newcommand{\ddn}[3]{\frac{\mathrm{d}^{#1}{#2}}{\mathrm{d}{#3^{#1}}}}

% N-th root
% \nroot{3}{27}
\newcommand*{\nroot}[2]{\sqrt[\leftroot{-1}\uproot{2}#1]{#2}}
\newcommand*{\ncroot}[4]{\sqrt[\leftroot{#1}\uproot{#2}#3]{#4}}

% 2 component vector
% \tvect{1}{-1}
% \tvec{1}{-1}
\newcommand{\tvect}[2]{%
   \ensuremath{\Bigl(\negthinspace\begin{smallmatrix}#1\\#2\end{smallmatrix}\Bigr)}}
\newcommand{\tvec}[2]{%
    \ensuremath{\left(\negthinspace\begin{matrix}#1\\#2\end{matrix}\right)}}

% 3 component vector
% \rvect{1}{-1}{0}
% \rvec{1}{-1}{0}
\newcommand{\rvect}[3]{%
   \ensuremath{\Bigl(\negthinspace\begin{smallmatrix}#1\\#2\\#3\end{smallmatrix}\Bigr)}}
\newcommand{\rvec}[3]{%
    \ensuremath{\left(\negthinspace\begin{matrix}#1\\#2\\#3\end{matrix}\right)}}

% Long vector arrow
% \xshlongvec{ABC}

% German-style quotation marks %
\MakeOuterQuote{"}

% Number sets
\newcommand{\N}{\mathbb{N}}
\newcommand{\Z}{\mathbb{Z}}
\newcommand{\Q}{\mathbb{Q}}
\newcommand{\R}{\mathbb{R}}
\newcommand{\C}{\mathbb{C}}

\newcommand{\setzero}{\varnothing}

% Mention (small caps)
\newcommand{\mention}[1]{\textsc{#1}}

% Functions
\newcommand{\asin}[0]{\text{asin}}
\newcommand{\acos}[0]{\text{acos}}
\newcommand{\atan}[0]{\text{atan}}
\newcommand{\sgn}[0]{\text{sgn}}
\newcommand{\grad}[0]{\text{grad}}

% Scale
% Usage in math mode: \Scale[1.5]{...equation...} %
\newcommand*{\Scale}[2][4]{\scalebox{#1}{$#2$}}%

% Units
\newcommand{\um}{\text{m}}
\newcommand{\us}{\text{s}}
\newcommand{\ukm}{\text{km}}
\newcommand{\ukg}{\text{kg}}
\newcommand{\uh}{\text{h}}
\newcommand{\ukmh}{\frac{\ukm}{\uh}}
\newcommand{\umpers}{\frac{\um}{\us}}
\newcommand{\umss}{\frac{\ukm}{\us^2}}
\newcommand{\ukgss}{\frac{\ukg}{\us^2}}
\newcommand{\degrees}[1]{\SI{#1}{\degree}}

% Floor / ceil
\newcommand{\floor}[1]{\left\lfloor #1 \right\rfloor}
\newcommand{\ceil}[1]{\left\lceil #1 \right\rceil}

% Circle characters
\newcommand*\circled[1]{
    \tikz[baseline=(char.base)]{
        \node[shape=circle,draw,inner sep=2pt] (char) {#1};
    }
}


\cfoot{\thepage\ of \pageref*{LastTask}}


\begin{document}

\thispagestyle{firstpage}

\begin{flushright}
Anzahl der Klausurblätter: \pageref*{LastTask}
\end{flushright}

Klausur im Lehrgebiet: \module \\

Teilgebiet: \submodule \\

Modulcode: \modulecode \\

Lehrender: \lecturer \\

Semester: \termnumber \\

\begin{vwcol}[widths={0.4,0.4,0.2},sep=0cm, justify=flush,rule=0pt,indent=4em]
Name:\\
Vorname:\\
SG:
\end{vwcol}

\bigskip
\bigskip
\bigskip
Bearbeitungszeit: \allottedtime \\

Zugelassene Hilfsmittel: \permittedtools \\

\textbf{Es dürfen keine eigenen Zusatzblätter abgegeben werden.} \\

\textbf{Verwenden Sie auch die Blattrückseiten für Antworten! Markieren Sie deutlich, zu welcher Frage die auf Rückseiten gegebenen Antworten gehören.} \\

\textbf {Der Rechengang muss eindeutig und vollständig ersichtlich sein!} \\

Punkteverteilung:

\bigskip

\scoretable


\bigskip
\bigskip

\textbf{Zusatzaufgabe:} Nennen Sie die \textsc{Peano}-Axiome!

\newpage

\section* {Aufgabe 1 - Funktionen}

\begin{center}
	\includegraphics[width=0.95\textwidth]{./gnuplot/exam-2020-feb-grid}
\end{center}

\begin{enumerate}[label=(\alph*)]

	\item (2P) In der oberen Skizze ist die Funktion $f(x)$ dargestellt. Es handelt sich um eine transformierte Kosinusfunktion. Lesen Sie für $f(x)$ folgende Eigenschaften ab:
		\begin{itemize}		
			\bigskip
		
			\item Amplitude:
		
			\bigskip
			\bigskip
		
			\item Periode:
		
			\bigskip
			\bigskip
			\bigskip		
		\end{itemize}

	\item (3P) Finden Sie für $f(x)$ eine mögliche Funktionsgleichung! \\
	
		\bigskip
	
		$f(x)=$
	
		\bigskip
		\bigskip
		\bigskip
	
	\item (2P) Tragen Sie in die Skizze die Taylorentwicklung 2. Grads von $f$ an der Stelle $x_0=0{,}5$ ein!

\end{enumerate}


\newpage

\section* {Aufgabe 2 - Partielle Ableitung}

Ein quadratisches Polynom, welches durch drei paarweise verschiedene Punkte $(x_1,y_1)$, $(x_2,y_2)$ und $(x_3,y_3)$ verläuft, lässt sich wie folgt schreiben:

$$
p_2(x) =
y_1\cdot\frac{x-x_2}{x_1-x_2}\cdot\frac{x-x_3}{x_1-x_3} +
y_2\cdot\frac{x-x_1}{x_2-x_1}\cdot \frac{x-x_3}{x_2-x_3} +
y_3\cdot\frac{x-x_1}{x_3-x_1}\cdot\frac{x-x_2}{x_3-x_2}
$$


\begin{enumerate}[label=(\alph*)]

\item (5P) Finden Sie ein quadratisches Polynom der Form $p_2(x)=ax^2+bx+c$, welches durch die folgenden Punkte verläuft!

\begin{itemize}
\item $(0,1)$
\item $(1,3)$
\item $(2,7)$
\end{itemize}

\bigskip
\bigskip
\bigskip
\bigskip
\bigskip
\bigskip
\bigskip
\bigskip
\bigskip
\bigskip
\bigskip
\bigskip

\item (2P) Beweisen Sie durch Einsetzen, dass für alle Punkte $(x_2,y_2)\in\mathbb{R}^2$ dass Polynom durch diesen Punkt verläuft, d.h. dass $p_2(x_2) = y_2$ gilt!

\end{enumerate}


\newpage

\section* {Aufgabe 3 - Grenzwerte}

Gegeben sei die Funktion $f(x) = x\cdot e^x$ und die Entwicklungsstelle $x_0=0$. Es soll die Taylorentwicklung 2. Grads $p_2(x)$ betrachtet werden.

Taylorformel: $\sum\limits_{n=0}^\infty\frac{f^{(n)}(x_0)}{n!}(x-x_0)^n$

\begin{enumerate}[label=(\alph*)]

\item (4P) Berechnen Sie die Taylorentwicklung 2. Grads von $f$!

\bigskip

$f'(x) = $

\bigskip

$f''(x) = $

\bigskip

$p_2(x) = $

\bigskip
\bigskip
\bigskip
\bigskip
\bigskip

\item (1P) Geben Sie einen Näherungswert für $f(0{,}5)$ an!

\bigskip

$p_2(0{,}5) = $

\bigskip
\bigskip

\item (3P) Die 3. Ableitung lautet $f'''(x)=(x+3)\cdot e^x$. Ermitteln Sie damit eine Abschätzung für den Fehler auf den Wert aus Aufgabe (b)! 

(Hinweis: $\sqrt{e} < \sqrt{4} = 2$)

\bigskip

$|R_2(0{,}5)| \le \frac{|0{,}5-0|^3}{(2+1)!}\cdot\max\limits_{\vartheta\in[0;0{,}5]} f'''(\vartheta) =$

\end{enumerate}


\newpage

\section* {Aufgabe 4 - Taylorentwicklung}

Gegeben sei die inhomogene DGL

$$2yy' = \frac{x+6}{x^2-4}$$

Durch Termumformung lässt sich zeigen:

$$\frac{x+6}{x^2-4} = \frac{2}{x-2} - \frac{1}{x+2}$$

\begin{enumerate}[label=(\alph*)]

\item (4P) Beweisen Sie die Termumformung mittels Partialbruchzerlegung!

$\frac{x+6}{x^2-4} = $

\bigskip
\bigskip
\bigskip
\bigskip
\bigskip
\bigskip
\bigskip
\bigskip
\bigskip
\bigskip
\bigskip
\bigskip

\item (4P) Ermitteln Sie die allgemeine Lösung der DGL durch Trennung der Variablen!

\bigskip
\bigskip
\bigskip
\bigskip
\bigskip
\bigskip
\bigskip
\bigskip
\bigskip

\end{enumerate}


\newpage

\section* {Aufgabe 5 - DGL}

\begin{enumerate}[label=(\alph*)]
	
	\item (3P) Erläutern Sie stichpunktartig die Struktur der Lösung einer linearen inhomogenen DGL n.-ter Ordnung mit konstanten Koeffzienten! Nutzen Sie dazu auch die Begriffe "Fundamentallösung", "Linearkombination", "homogene DGL", "inhomogene DGL" , "partikuläre Lösung" und "allgemeine Lösung"!
	
	\bigskip
	\bigskip
	\bigskip
	\bigskip
	\bigskip
	\bigskip
	\bigskip
	\bigskip
	\bigskip
	\bigskip
	\bigskip
	\bigskip
	\bigskip
	\bigskip
	\bigskip
	\bigskip
	\bigskip
	\bigskip
	
	\item (5P) Berechnen Sie die allgemeine Lösung von $y'''-4y'=-24x^2$, wenn eine partikuläre Lösung $2x^3+3x$ lautet!
	
\end{enumerate}


\label{LastTask}

\newpage

\begin{center}
{\bf {\large Musterlösung}}
\end{center}

\begin{center}
{\bf {\large Klausur \submodule - \examdate}}
\end{center}

Jeder Anführungspunkt entspricht einem erteilten Bewertungspunkt. \\

Bei Aufgaben, welche einen Rechenweg erfordern, wird von der Maximalpunktzahl ausgegangen und für jeden fehlerhaften Rechenschritt (Bruch falsch gekürzt, nichtzutreffende Rechenregel angewandt, falsch zusammengefasst, Term falsch übernommen oder vergessen etc.) ein halber Punkt abgezogen. Das bedeutet auch:

\begin{itemize}
	\item Wird beispielsweise die Rechnung mit einem fehlerhaften Zwischenergebnis konsequent und korrekt fortgesetzt, gibt es keinen erneuten Punktabzug (Folgefehler).
	\item Werden beispielsweise 2 Rechenfehler gemacht, welche sich so aufheben, dass das Endergebnis korrekt ist, wird dennoch zweimal ein halber Punkt abgezogen.
\end{itemize}


\begin{enumerate}

\item
	\begin{enumerate}
	
		\item
			\begin{itemize}
				\item Wertebereich $[0,2)$
				\item Periode 1
			\end{itemize}
		
		\item
			\begin{itemize}
				\item Graph hat Form einer Sinus/Kosinuskurve
				\item Ampltiude korrekt (Abfall von 1 auf 0)
				\item Periode des Kosinusbogens korrekt 
				\item Periodische Fortsetzung wurde beachtet (0-1, wiederholend)
				\item Wertebereich und Periode ist kompatibel mit Antwort aus Teilaufgabe (a)
			\end{itemize}
		
			\begin{center}
				\includegraphics[width=0.5\textwidth]{./gnuplot/exam-2021-feb-grid-solution}
			\end{center}
		
		\item 
			\begin{itemize}
				\item An den Unstetigkeitsstellen (Sprungstellen) ist Fourier-Reihe arithmetisches Mittel zwischen links- und rechtsseitigen Grenzwert
				\item $F(0) = (2+0)/2 = 1$
			\end{itemize}
	
	\end{enumerate}


\item
\begin{enumerate}

\item
\begin{itemize}
\item $p_2 =
1\cdot\frac{x-1}{0-1}\cdot\frac{x-2}{0-2} +
3\cdot\frac{x-0}{1-0}\cdot \frac{x-2}{1-2} +
7\cdot\frac{x-0}{2-0}\cdot\frac{x-1}{2-1}$
\item $=\frac{1}{2}(x-1)(x-2)-3x(x-2)+\frac{7}{2}x(x-1)$
\item $=\frac{1}{2}(x^2-3x+2)-3(x^2-2x)+\frac{7}{2}(x^2-x)$
\item $=(\frac{1}{2}-3+\frac{7}{2})x^2+(-\frac{3}{2}+6-\frac{7}{2})x+2\cdot\frac{1}{2}$
\item $=x^2+x+1$
\end{itemize}

\item
\begin{itemize}
\item $p_2(x_2) =
y_1\cdot\underbrace{\frac{x_2-x_2}{x_1-x_2}}_0\cdot\frac{x_2-x_3}{x_1-x_3} +
y_2\cdot\underbrace{\frac{x_2-x_1}{x_2-x_1}}_1\cdot\underbrace{\frac{x_2-x_3}{x_2-x_3}}_1 +
y_3\cdot\frac{x_2-x_1}{x_3-x_1}\cdot\underbrace{\frac{x_2-x_2}{x_3-x_2}}_0$
\item $=y_2$
\end{itemize}

\end{enumerate}


\newpage

\item
\begin{enumerate}

\item
\begin{itemize}
\item $f'(x) = (x+1)e^x$
\item $f''(x) = (x+2)e^x$
\item $f(0) = 0$, $f'(0) = 1$, $f''(0)=2$
\item $p_2(x) \approx 0 + \frac{1}{1!} x + \frac{2}{2!} x^2 = x+x^2$
\end{itemize}

\item
\begin{itemize}
\item $p_2(0{,}5) = 0{,}5+0{,}5^2 = 0{,}75$
\end{itemize}

\item
\begin{itemize}
\item $(x+3)e^x$ ist monoton steigend, Maximum im Intervall $[0;0{,}5]$ ist somit $(0{,}5+3)e^{0{,}5} < 3{,}5\cdot 2 = 7$
\item $|R_2(0{,}5)| \le \frac{0{,}5^3}{6}\cdot 7$
\item $=\frac{7}{48}$
\end{itemize}

\end{enumerate}


\item
	\begin{enumerate}
	
	\item
		\begin{itemize}
			\item $f(0) = \ln(0+1) = 0$
			\item $f'(0) = 1/(0+1) = 1$
			\item $T_1(x) = 0/(0!)x^0+1/(1!)x^1 = x$
		\end{itemize}
	
	\item
		\begin{itemize}
			\item $\frac{|0.5-0|^2}{2!} = 1/4 : 2 = 1/8$
			\item $\frac{-1}{(x+1)^2}$ ist ein an der Abszisse gespiegelte und 1 nach links verschobene Hyperbel
			\item Damit für $x>0$ negativ und monoton steigend, der Betrag also monoton fallend.
			\item Damit $\max\limits_{0\le x \le 0.5} |\frac{-1}{(x+1)^2}| = |\frac{-1}{(x+1)^2}| \bigg\rvert_{x=0} = 1 $
			\item Somit $|f(0.5) - T_1(0.5)| \le 1/8 \cdot 1 = 1 / 8$
		\end{itemize}
		
	\end{enumerate}


\item
	\begin{enumerate}
		
	\item
		\begin{itemize}
			\item Grad 1
			\item Homogen ja
			\item Linear ja
		\end{itemize}
	
	\item
		\begin{itemize}
			\item $u=\sin(x)$, $\dd{u}{x} = \cos(x)$, $\diff{x} = \diff{u} / \cos(x)$
			\item $\int \diff{y} / y = - \int \cos(x) \sin(\sin(x)) \diff{x}$
			\item $\ln(|y|) = - \int \cos(x) \sin(u) \diff{u} / \cos(x) = - \int \sin(u) \diff{u}$
			\item $\ln(|y|) = \cos(u) + C = \cos(\sin(x)) + C$
			\item $y(x) = \hat{C} \cdot e^{\cos(\sin(x))}$
		\end{itemize}
			
	\end{enumerate}


\end{enumerate}

\end{document}
