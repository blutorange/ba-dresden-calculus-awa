\newcommand{\examdate}{13.10.2021}
\newcommand{\lecturer}{Andre Wachsmuth}
\newcommand{\modulecode}{3MI-MATHE-10}
\newcommand{\module}{Algebra/Analysis}
\newcommand{\submodule}{Analysis}
\newcommand{\termnumber}{1}
\newcommand{\allottedtime}{40 min}
\newcommand{\permittedtools}{1 handbeschriebenes A4-Blatt}
\newcommand{\scoretable}{
	\begin{tabularx}{\textwidth}{l|Y|Y|Y|Y|Y|Y}
		Aufgabe        & 1 & 2 & 3 & 4  & 5 & Summe \\ [1ex] \hline
		Soll-Punktzahl & 8 & 8 & 8 & 8  & 8 &       \\ [3ex]
		Ist-Punktzahl  &   &   &   &    &    &      ¸\\ [3ex]
	\end{tabularx}
}

\documentclass[12pt]{article}
\usepackage[a4paper]{geometry}
\usepackage[pdftex]{hyperref}
\usepackage[german]{babel}
\usepackage[utf8]{inputenc}
\usepackage{csquotes}
\usepackage{amssymb}
\usepackage{graphicx}
\usepackage{multicol}
\usepackage{amsmath}
\usepackage{enumitem}
\usepackage{tabularx}
\usepackage{vwcol}
\usepackage{fancyhdr}
\usepackage{indentfirst}
\usepackage{polynom}

\geometry{
	headheight=14px,
	left=2.54cm,
	right=2.54cm,
	bottom=2cm,
	top=2cm
}

% For horizontally centering text in Y column
\newcolumntype{Y}{>{\centering\arraybackslash}X}

\setlength{\parindent}{0cm}

\setlength{\marginparsep}{1 cm}
\setlength{\topmargin}{-0.6in}
\setlength{\textheight}{9.5in}
\pagestyle{fancy}

\fancypagestyle{firstpage}{ %
	\lhead{\bf Staatliche Studienakademie Dresden\\
		Studienrichtung: Informationstechnologie}
	\rhead{\bf Datum: \examdate}
}

% Differential operator
\providecommand\d{}
\renewcommand{\d}[1]{\:\mathrm{d}{#1}}
\newcommand{\pdd}[2]{\frac{\partial #1}{\partial #2}}

% N-th root
% \nroot{3}{27}
\newcommand*{\nroot}[2]{\sqrt[\leftroot{-1}\uproot{2}#1]{#2}}
\newcommand*{\ncroot}[4]{\sqrt[\leftroot{#1}\uproot{#2}#3]{#4}}

% 2 component vector
% \tvect{1}{-1}
% \tvec{1}{-1}
\newcommand{\tvect}[2]{%
   \ensuremath{\Bigl(\negthinspace\begin{smallmatrix}#1\\#2\end{smallmatrix}\Bigr)}}
\newcommand{\tvec}[2]{%
    \ensuremath{\left(\negthinspace\begin{matrix}#1\\#2\end{matrix}\right)}}

% 3 component vector
% \rvect{1}{-1}{0}
% \rvec{1}{-1}{0}
\newcommand{\rvect}[3]{%
   \ensuremath{\Bigl(\negthinspace\begin{smallmatrix}#1\\#2\\#3\end{smallmatrix}\Bigr)}}
\newcommand{\rvec}[3]{%
    \ensuremath{\left(\negthinspace\begin{matrix}#1\\#2\\#3\end{matrix}\right)}}

% German-style quotation marks %
\MakeOuterQuote{"}

% Number sets
\newcommand{\N}{\mathbb{N}}
\newcommand{\Z}{\mathbb{Z}}
\newcommand{\Q}{\mathbb{Q}}
\newcommand{\R}{\mathbb{R}}
\newcommand{\C}{\mathbb{C}}

\newcommand{\setzero}{\varnothing}

% Mention (small caps)
\newcommand{\mention}[1]{\textsc{#1}}

% Units
\newcommand{\ukm}{\text{km}}
\newcommand{\uh}{\text{h}}
\newcommand{\ukmh}{\frac{\ukm}{\uh}}
\newcommand{\degrees}[1]{\SI{#1}{\degree}}

% Floor / ceil
\newcommand{\floor}[1]{\left\lfloor #1 \right\rfloor}
\newcommand{\ceil}[1]{\left\lceil #1 \right\rceil}

\newcommand{\term}[2]{$\stackrel{\hbox{\tiny \textnormal{#2}}}{\hbox{#1}}$}
% \newcommand{\term}[2]{#1}



\cfoot{\thepage\ of \pageref*{LastTask}}


\begin{document}

\thispagestyle{firstpage}

\begin{flushright}
Anzahl der Klausurblätter: \pageref*{LastTask}
\end{flushright}

Klausur im Lehrgebiet: \module \\

Teilgebiet: \submodule \\

Modulcode: \modulecode \\

Lehrender: \lecturer \\

Semester: \termnumber \\

\begin{vwcol}[widths={0.4,0.4,0.2},sep=0cm, justify=flush,rule=0pt,indent=4em]
Name:\\
Vorname:\\
SG:
\end{vwcol}

\bigskip
\bigskip
\bigskip
Bearbeitungszeit: \allottedtime \\

Zugelassene Hilfsmittel: \permittedtools \\

\textbf{Es dürfen keine eigenen Zusatzblätter abgegeben werden.} \\

\textbf{Verwenden Sie auch die Blattrückseiten für Antworten! Markieren Sie deutlich, zu welcher Frage die auf Rückseiten gegebenen Antworten gehören.} \\

\textbf {Der Rechengang muss eindeutig und vollständig ersichtlich sein!} \\

Punkteverteilung:

\bigskip

\scoretable


\bigskip
\bigskip

\textbf{Zusatzaufgabe:} Nennen Sie die \textsc{Peano}-Axiome!

\newpage

\section* {Aufgabe 1 - Funktionen}

\begin{center}
	\includegraphics[width=0.95\textwidth]{./gnuplot/exam-2020-feb-grid}
\end{center}

\begin{enumerate}[label=(\alph*)]

	\item (2P) In der oberen Skizze ist die Funktion $f(x)$ dargestellt. Es handelt sich um eine transformierte Kosinusfunktion. Lesen Sie für $f(x)$ folgende Eigenschaften ab:
		\begin{itemize}		
			\bigskip
		
			\item Amplitude:
		
			\bigskip
			\bigskip
		
			\item Periode:
		
			\bigskip
			\bigskip
			\bigskip		
		\end{itemize}

	\item (3P) Finden Sie für $f(x)$ eine mögliche Funktionsgleichung! \\
	
		\bigskip
	
		$f(x)=$
	
		\bigskip
		\bigskip
		\bigskip
	
	\item (2P) Tragen Sie in die Skizze die Taylorentwicklung 2. Grads von $f$ an der Stelle $x_0=0{,}5$ ein!

\end{enumerate}


\newpage

\section* {Aufgabe 2 - Partielle Ableitung}

Eine auf ganz $\mathbb{R}$ definierte Funktion $f$ lässt sich zerlegen in die Summe aus einem achsensymmetrischen Anteil $f_a$ und einem punktsymmetrischen Anteil $f_p$. Es gilt:

\begin{align*}
&f_a(x) = \frac{f(x)+f(-x)}{2}\\
&f_p(x)= \frac{f(x)-f(-x)}{2}\\
\end{align*}

\begin{enumerate}[label=(\alph*)]

\item (2P) Geben Sie eine Funktionsgleichung für den "Cosinus Hyperbolicus" an, definiert als der achsensymmetrische Anteil der Exponentialfunktion $e^x$!

\bigskip
\bigskip
\bigskip
\bigskip
\bigskip
\bigskip
\bigskip
\bigskip

\item (3P) Ermitteln Sie von $f(x) = (x+1)^2$ den punktsymmetrischen Anteil!

\bigskip
\bigskip
\bigskip
\bigskip
\bigskip
\bigskip
\bigskip
\bigskip
\bigskip
\bigskip
\bigskip
\bigskip

\item (3P) Zeigen Sie, dass $f_p$ für jede Funktion $f$ tatsächlich punktsymmetrisch ist, d.h. dass $f_p(-x) = -f_p(x)$ gilt!


\end{enumerate}


\newpage

\section* {Aufgabe 3 - Grenzwerte}

Wir betrachten die Taylorreihe der Funktion $f(x) = \ln(x+1)$ an der Stelle $x_0=0$.

\begin{enumerate}[label=(\alph*)]

\item (2P) Geben Sie die 1. und 2. Ableitung von $f$ an!

\bigskip

$f'(x) = $ \\
\bigskip \\
$f''(x) = $
\bigskip
\bigskip

\item (3P) Geben Sie die Taylorentwicklung 2. Grads von $f$ an!

\bigskip
\bigskip

$\ln(x+1) \approx \hspace{2cm}   + \hspace{2cm} \cdot \hspace{1mm} x + \hspace{2cm} \cdot \hspace{1mm} x^2$
\bigskip
\bigskip
\bigskip
\bigskip


\item (3P) Die Folge der Koeffizienten $a_n$ der Taylorreihe lautet $a_n=\frac{(-1)^{n+1}}{n}$. Ermitteln Sie den Konvergenzradius R!

\bigskip

$R=1/\lim_{n\to \infty} |\frac{a_{n+1}}{a_n}| = $

\end{enumerate}


\newpage

\section* {Aufgabe 4 - Taylorentwicklung}

Gegeben sei die inhomogene DGL

$$2yy' = \frac{x+6}{x^2-4}$$

Durch Termumformung lässt sich zeigen:

$$\frac{x+6}{x^2-4} = \frac{2}{x-2} - \frac{1}{x+2}$$

\begin{enumerate}[label=(\alph*)]

\item (4P) Beweisen Sie die Termumformung mittels Partialbruchzerlegung!

$\frac{x+6}{x^2-4} = $

\bigskip
\bigskip
\bigskip
\bigskip
\bigskip
\bigskip
\bigskip
\bigskip
\bigskip
\bigskip
\bigskip
\bigskip

\item (4P) Ermitteln Sie die allgemeine Lösung der DGL durch Trennung der Variablen!

\bigskip
\bigskip
\bigskip
\bigskip
\bigskip
\bigskip
\bigskip
\bigskip
\bigskip

\end{enumerate}


\newpage

\section* {Aufgabe 5 - DGL}

Gegeben sei die DGL $y' + y \cos(x) \sin(\sin(x)) = 0$

\begin{enumerate}[label=(\alph*)]

	\item (3P) Klassifizieren Sie diese DGL (ohne Begründung)!

	\bigskip

		\begin{itemize}
			\item Grad:	\bigskip	
			\item Homogen? \bigskip	
			\item Linear? \bigskip	
		\end{itemize}

	\bigskip

	\item (5P) Berechnen Sie die allgemeine Lösung mittels Trennung der Variablen! Das Integral können Sie mit der Substitution $u=\sin(x)$ lösen.

\end{enumerate}


\label{LastTask}

\newpage

\begin{center}
{\bf {\large Musterlösung}}
\end{center}

\begin{center}
{\bf {\large Klausur \submodule - \examdate}}
\end{center}

Jeder Anführungspunkt entspricht einem erteilten Bewertungspunkt. \\

Bei Aufgaben, welche einen Rechenweg erfordern, wird von der Maximalpunktzahl ausgegangen und für jeden fehlerhaften Rechenschritt (Bruch falsch gekürzt, nichtzutreffende Rechenregel angewandt, falsch zusammengefasst, Term falsch übernommen oder vergessen etc.) ein halber Punkt abgezogen. Das bedeutet auch:

\begin{itemize}
	\item Wird beispielsweise die Rechnung mit einem fehlerhaften Zwischenergebnis konsequent und korrekt fortgesetzt, gibt es keinen erneuten Punktabzug (Folgefehler).
	\item Werden beispielsweise 2 Rechenfehler gemacht, welche sich so aufheben, dass das Endergebnis korrekt ist, wird dennoch zweimal ein halber Punkt abgezogen.
\end{itemize}


\begin{enumerate}

\item
	\begin{enumerate}
	
		\item Etwa $f(x) = 3 \sin(\frac{\pi}{4}x)$ oder $f(x) = 3 \cos(\frac{\pi}{4}(x-2))$ oder $f(x) = 3/2\cdot x$ für $-2 < x \le 2$ und periodisch fortgesetzt.
			\begin{itemize}
				\item Funktion mit korrekter Amplitude eingezeichnet
				\item Funktion mit korrekten Maximum eingezeichnet
				\item Funktionsgleichung  stimmt mit Graphen überein
				\item Funktionsgleichung hat korrekte Amplitude
				\item Funktionsgleichung hat korrektes Maximum
			\end{itemize}
				
		\item Nullstellen sind $x \in \{ -4, 2, 3 \}$
			\begin{itemize}
				\item Grad 4
				\item 1/2 Punkt für korrekte Zahl, 1/2 Punkt für korrektes Vorzeichen
				\item Maximal 4 Nullstellen.
			\end{itemize}
	
	\end{enumerate}


\item
\begin{enumerate}

\item
\begin{itemize}
\item $\cosh(x) = \frac{f(x)+f(-x)}{2}$
\item $\cosh(x) = \frac{e^x+e^{-x}}{2}$
\end{itemize}

\item
\begin{itemize}
\item $f_p(x) = \frac{1}{2} ((x+1)^2-(-x+1)^2)$
\item $ = \frac{1}{2} (x^2+2x+1-(x^2-2x+1))$
\item $ = 2x$
\end{itemize}

\item
\begin{itemize}
\item $2f_p(-x) = f(-x)-f(x)$
\item $-2f_p(x) = -(f(x)-f(-x)) = f(-x)-f(x)$
\item Also $f_p(-x) = -f_p(x)$, dies war zu beweisen 
\end{itemize}

\end{enumerate}


\newpage

\item
	\begin{enumerate}
	
	\item Partielle Ableitung sind unabhängig der Reihenfolge, in der nach Variablen abgeleitet wird
		\begin{itemize}
			\item $f_{ab} = f_{ba}$
			\item $f_{ac} = f_{ca}$
			\item $f_{abc} = f_{cba}$
		\end{itemize}
	
	\item
		\begin{itemize}
			\item $ f_a = \frac{3a^2\sin(-3b)}{2c}$
			\item $ f_{ab}= \frac{3a^2\cdot (-3) \cdot \cos(-3b)}{2c}$
			\item $ f_{ab}= \frac{-9a^2\cdot \cos(-3b)}{2c}$
			\item $ f_{abc}= -\frac{-9a^2\cdot \cos(-3b)}{2c^2}$
			\item $ f_{abc}= \frac{9a^2\cdot \cos(-3b)}{2c^2} = \frac{9a^2\cdot \cos(3b)}{2c^2}$
		\end{itemize}
		
	\end{enumerate}


\item 
\begin{enumerate}

\item
\[\polylongdiv{x^3-3x^2-x+3}{x^2-2x-3}\]

\item 
\begin{itemize}
\item $2\lambda^2-6\lambda = 0$
\item $\lambda = 0 \lor \lambda = 3$
\item $y(x) = C_1 + C_2 e^{3x}$
\end{itemize}

\item 
\begin{itemize}
\item $y' = \frac{1}{9}-\frac{1}{6}x, y'' = -\frac{1}{6}$
\item $2y'' - 6y' = -\frac{1}{3} - \frac{2}{3} + x = x -1$
\item $\implies$ Die gegebene Funktion ist Lösung der inhomogenen DGL
\end{itemize}

\end{enumerate}


\item
	\begin{enumerate}
		
	\item
		\begin{itemize}
			\item Grad 1
			\item Homogen ja
			\item Linear ja
		\end{itemize}
	
	\item
		\begin{itemize}
			\item $u=\sin(x)$, $\dd{u}{x} = \cos(x)$, $\diff{x} = \diff{u} / \cos(x)$
			\item $\int \diff{y} / y = - \int \cos(x) \sin(\sin(x)) \diff{x}$
			\item $\ln(|y|) = - \int \cos(x) \sin(u) \diff{u} / \cos(x) = - \int \sin(u) \diff{u}$
			\item $\ln(|y|) = \cos(u) + C = \cos(\sin(x)) + C$
			\item $y(x) = \hat{C} \cdot e^{\cos(\sin(x))}$
		\end{itemize}
			
	\end{enumerate}


\end{enumerate}

\end{document}
