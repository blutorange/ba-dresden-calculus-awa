Sei $f(x)=xe^x$. Es gilt die Aussage (*): $\ddn{n}{}{x} f(x) = (x+n)e^x$.
\begin{itemize}
	\item Taylorreihe bei $x_0$: $f(x) = \sum\limits_{n=0}^{\infty} \frac{f^{(n)}(x_0)}{n!} (x-x_0)^n$
	\item Konvergenzradius: $1/R = \lim\limits_{n\to\infty} \left|\frac{a_{n+1}}{a_n}\right|$
\end{itemize}
\begin{enumerate}[label=(\alph*)]
	
	\item (3P) Aussage (*) kann mittels vollständiger Induktion bewiesen werden. Für den Induktionschritt muss gezeigt werden, dass $\dd{}{x} (x+a)e^x = (x+a+1)e^x$ gilt. Zeigen Sie dies!
	
		\bigskip
		\bigskip
		\bigskip
		\bigskip
		\bigskip
		\bigskip
		\bigskip
		\bigskip
		\bigskip
		\bigskip
		\bigskip
		
	\item (5P) Ermitteln Sie die Taylorreihe von $f$ an der Entwicklungsstelle $0$ und berechnen Sie deren Konvergenzradius!
		
\end{enumerate}
