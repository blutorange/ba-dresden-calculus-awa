Bei der sogenannten Heaviside-Distribution $H$ und der Dirac-Distribution $\delta$ handelt es sich um lineare Funktionale. Diese werden durch ihre Wirkung bei der Integration mit Testfunktionen $\varphi$ definiert:

\begin{itemize}
\item $\int\limits_{-\infty}^\infty \varphi(x) H(x) \diff{x} := \int\limits_{0}^\infty \varphi(x) \diff{x}$
\item $\int\limits_{-\infty}^\infty \varphi(x) \delta(x) \diff{x} := \varphi(0)$
\end{itemize}

Jede Testfunktion $\varphi$ hat unter anderem die Eigenschaft, dass sie außerhalb eines Intervalls um den Ursprung identisch $0$ ist, speziell also auch $\lim\limits_{x\to\pm\infty} \varphi(x) = 0$ gilt.

Zeigen Sie durch partielle Integration, dass die Dirac-Distribution $\delta$ die Ableitung $H'$ der Heaviside-Distribution ist, d.h. dass folgende Beziehung gilt:

\bigskip
 
$\int\limits_{-\infty}^\infty \varphi(x) H'(x) \diff{x} = \varphi(0)$
