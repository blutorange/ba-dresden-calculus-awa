Ein quadratisches Polynom, welches durch drei paarweise verschiedene Punkte $(x_1,y_1)$, $(x_2,y_2)$ und $(x_3,y_3)$ verläuft, lässt sich wie folgt schreiben:

$$
p_2(x) =
y_1\cdot\frac{x-x_2}{x_1-x_2}\cdot\frac{x-x_3}{x_1-x_3} +
y_2\cdot\frac{x-x_1}{x_2-x_1}\cdot \frac{x-x_3}{x_2-x_3} +
y_3\cdot\frac{x-x_1}{x_3-x_1}\cdot\frac{x-x_2}{x_3-x_2}
$$


\begin{enumerate}[label=(\alph*)]

\item (5P) Finden Sie ein quadratisches Polynom der Form $p_2(x)=ax^2+bx+c$, welches durch die folgenden Punkte verläuft!

\begin{itemize}
\item $(0,1)$
\item $(1,3)$
\item $(2,7)$
\end{itemize}

\bigskip
\bigskip
\bigskip
\bigskip
\bigskip
\bigskip
\bigskip
\bigskip
\bigskip
\bigskip
\bigskip
\bigskip

\item (2P) Beweisen Sie durch Einsetzen, dass für alle Punkte $(x_2,y_2)\in\mathbb{R}^2$ dass Polynom durch diesen Punkt verläuft, d.h. dass $p_2(x_2) = y_2$ gilt!

\end{enumerate}
