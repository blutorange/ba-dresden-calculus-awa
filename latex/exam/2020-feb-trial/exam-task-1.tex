In der folgenden Skizze ist die Funktion $f(x)=\frac{x}{\sin(x)}$ dargestellt. Weiterhin ist die Funktion $g(x)$ dargestellt, welche aus $f$ durch Verschiebung hervorging.

\begin{center}
	\includegraphics[width=0.95\textwidth]{./gnuplot/exam-2020-feb-trial-grid}
\end{center}

\begin{enumerate}[label=(\alph*)]
\item (2P) Geben Sie die Koordinaten des Minimums von $f$ und von $g$ an!\\

Minimum von $f$:\\
\bigskip\\
Minimum von $g$:\\
\bigskip
\bigskip

\item (2P) Finden Sie für $g(x)$ eine mögliche Funktionsgleichung!

\bigskip
\bigskip

$g(x)=$

\bigskip
\bigskip
\bigskip

\item (2P) Ermitteln Sie grafisch die Lösungen der Gleichung $\frac{x}{\sin(x)}-2=0$!

\bigskip
\bigskip
\bigskip
\bigskip
\bigskip
\bigskip

\item (1P) Tragen Sie in der Skizze die Taylorentwicklung 1. Grads von $g$ an der Stelle $x_0=2$ ein!

\end {enumerate}
