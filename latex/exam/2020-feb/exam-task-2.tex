Eine auf ganz $\mathbb{R}$ definierte Funktion $f$ lässt sich zerlegen in die Summe aus einem achsensymmetrischen Anteil $f_a$ und einem punktsymmetrischen Anteil $f_p$. Es gilt:

\begin{align*}
&f_a(x) = \frac{f(x)+f(-x)}{2}\\
&f_p(x)= \frac{f(x)-f(-x)}{2}\\
\end{align*}

\begin{enumerate}[label=(\alph*)]

\item (2P) Geben Sie eine Funktionsgleichung für den "Cosinus Hyperbolicus" an, definiert als der achsensymmetrische Anteil der Exponentialfunktion $e^x$!

\bigskip
\bigskip
\bigskip
\bigskip
\bigskip
\bigskip
\bigskip
\bigskip

\item (3P) Ermitteln Sie von $f(x) = (x+1)^2$ den punktsymmetrischen Anteil!

\bigskip
\bigskip
\bigskip
\bigskip
\bigskip
\bigskip
\bigskip
\bigskip
\bigskip
\bigskip
\bigskip
\bigskip

\item (3P) Zeigen Sie, dass $f_p$ für jede Funktion $f$ tatsächlich punktsymmetrisch ist, d.h. dass $f_p(-x) = -f_p(x)$ gilt!


\end{enumerate}
