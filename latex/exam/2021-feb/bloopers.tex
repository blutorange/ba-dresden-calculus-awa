Auch wenn das untenstehende schon Bauchschmerzen bereitet, ist das immer noch besser als Studenten, die von Anfang an aufgeben und ein leeres Blatt abgeben. Einige haben trotz solcher grober Schnitzer bestanden. Daher: Mut haben, etwas zu wagen!

\subsection* {Bruchrechnung}

\begin{itemize}	
	\item Der "Klassiker" zuerst: $\frac{4-2}{5-3} = \frac{4}{5} - \frac{2}{3}$
	\item Auch nicht schlecht: $\frac{2}{1/2} = 1$
	\item Partielles Kürzen: $\frac{3\cancel{n^2}-4n^3}{12n^4+\cancel{n^2}} = \frac{3 - 4n^3}{12n^4+1}$
	\item Was reimte sich nochmal auf "Summen"? Wir kürzen $x$ und erhalten $x^{\cancel{2}}-\cancel{x} = x$
	\item Distributivgesetz der Multiplikation bzgl. Division: $4\cdot\frac{1}{4} = \frac{4\cdot 1}{4\cdot 4} = 0{,}25$
\end{itemize}
	
\subsection* {Rechnungen}
	
\begin{itemize}
	\item Jemand anderes erhält aus dem Wurzelkriterium $q=\sqrt{-\frac{1}{3}}$ und da $\sqrt{-\frac{1}{3}} < 1$ konvergiert die Reihe. Diese Ordnungsrelation auf $\mathbb{C}$ hätte ich auch gerne.
	\item Integrieren kann doch so einfach sein: $\int\frac{f(x)}{g(x)} \diff{x} =\frac{\int f(x)\diff{x}}{\int g(x)\diff{x}} $
	\item Das Schweizer Taschenmesser der Analysis: die p-q-Formel. Damit erhält man immer das richtige Ergebnis! (?)
	\begin{itemize}
		\item $\lambda^2-4=0$
		\item $\implies \lambda_{1,2} = 2 \pm \sqrt{4 - 0} = 2 \pm 2$
		\item $\lambda_1 = 0, \lambda_2 = 4$
	\end{itemize}
	\item Die Rechnung wird nicht richtig, auch wenn man zwanghaft versucht, am Ende das von der Aufgabe geforderte Ergebnis zu erhalten. Aus einer Aussage Schlussfolgerungen zu ziehen ist integraler Bestandteil der Mathematik!
	\begin{itemize}
		\item $\frac{C_1}{x-3} + \frac{C_2}{x+3}$
		\item $= \frac{C_1(x+3)}{x-3} + \frac{C_2(x-3)}{x+3}$
		\item $= \frac{C_1(x+3)+C_2(x-3)}{(x-3)(x+3)}$
	\end{itemize}
	\item $\sum\limits_{n=1}^{\infty} n \frac{3n-4n^2}{5n^3+n+7n^3}$ ist auf Konvergenz untersuchen. Was juckt mit die lästige Summe?
	\begin{itemize}
		\item $\sum\limits_{n=1}^{\infty} 1 \frac{3-4}{5+1+7} = -\frac{1}{13}$
		\item $\implies q=\lim\limits_{n \to\infty} \nroot{n}{|-\frac{1}{13}|} = \frac{1}{13}$, also konvergent
	\end{itemize}
\end{itemize}

\subsection* {Weiteres}

\begin{itemize}
	\item Transitive Variante des Satzes von Schwarz: $f_{ab}+f_{bc}=f_{abc}$
	\item Holistische Variante des Satzes von Schwarz: "Bei allen Ableitungen $f_{ab}$, $f_{ac}$, $f_{abc}$, $f_{ba}$, $f_{ca}$, $f_{cba}$ kommt immer das gleiche heraus."
	\item Zu zeigen ist eine mathematische Behauptung (Termumformung). Jemand formt um und erhält den Term ${\frac{4}{0}}$. Das ist nicht definiert, also (so dessen Schlussfolgerung) ist die Behauptung bewiesen.
	\item "Sie können sich doch denken, was gemeint war!" $\lim\limits_{n\to\infty} \frac{n}{n^2} = \frac{n}{n^2} = 0$
	\item Für Querdenker sind alle Wurzeln quadratisch: $\nroot{1}{4} = \sqrt{4} = 2$
	\item Betrachten wir doch mal den Term $\frac{1}{x^2}$. Wenn wir den auf beiden (?) Seiten mit $x^2$ multiplizieren, erhalten wir: $1=x^2$, also $x=1$.
	\item Die Aufgabenstellung lautet: "Zeigen Sie dass A, wenn B gilt!" beziehungsweise "Berechnen  Sie A unter Verwendung der Tatsache B". Nun gibt es Studenten, die beschreiben die gesamte Rückseite mit einer Rechnung, die zeigt, dass B tatsächlich gilt. Und hören dann auf. Hier werden Zeit und Punkte verschenkt.
	\item Aussage eines Studenten: "Die partikuläre Lösung ist eine Funktion, welche in die Differentialgleichung eingesetzt eine \textbf{wahre Funktion} ergibt."
	\item Weitere Aussage eines anderen Studenten: "Die Menge aller partikulären Lösungen ist eine Linearkombination."
	\item Die Anzahl richtig eingezeichneter Kosinusfunktionen unter 70 Klausuren: 8. Und das ist besser als in manchen anderen Jahren.	
\end{itemize}

\subsection* {Eigentlich zu gut, um wahr zu sein}

Zum krönenden Schluss noch etwas, was einem (aus mehreren Gründen) die Tränen in die Augen treibt. (Wortwörtliche) Antworten eines Studenten, der trotz fehlender Kenntnisse den Stift in die Hand nahm:

\begin{displayquote}
	"Man kann mit diesem q auf die Konvergenz / Divergenz der Reihe schließen, da n bis $\infty$ geht und die n-te Wurzel von $a_n$ im Betrag dazu beiträgt, dass man auf die Konvergenz / Divergenz schließen kann. Dies zeigt q sehr gut."	
\end{displayquote}

--- Antwort auf die Frage, wie man mit dem berechneten Wert q aus dem Wurzelkriterium auf Konvergenz der Reihe schließen kann


\begin{displayquote}
	"Die Lösung einer linearen inhomogenen DGL n.-ter Ordnung besteht aus einer Fundamentallösung, aus einer Linearkombination und aus einer homogenen DGL. Außerdem ist ein Teil der allgemeinen Lösung die partikuläre Lösung. Eine inhomogene DGL kommt in diesem Beispiel nicht vor. Dies ist ziemlich bedauerlich."
\end{displayquote}

--- Antwort auf die Aufgabe, die Lösungstruktur einer lin. inhomog. DGL zu erläutern


\begin{displayquote}
	"Der Satz von Schwarz trifft die großzügige Aussage, dass die Elemente von $f_{ac}$, $f_{abc}$, $f_{cba}$, $f_{ba}$, und $f_{ca}$ alle identisch sein müssen."
\end{displayquote}

--- Antwort auf die Frage, welche Aussage der Satz von Schwarz über partielle Ableitungen trifft
