\item
	\begin{enumerate}
	
	\item Alle schlüssigen Begründungen werden akzeptiert.
		\begin{itemize}
			\item Koeffizientenvergleich
			\item $C_1+C_2 = 6, 3(C_1-C_2)=-6$
			\item Einsetzen von $C_1=2, C_2 = 4$
			\item $2+4=6, 3(2-4)=3\cdot(-2)=-6$
		\end{itemize}
	
	\item $\int f(x) \diff{x} = 2 \ln(|x-3|) + 4 \ln(|x+3|) + C$
		\begin{itemize}
			\item Korrekte Form (Natürlicher Logarithmus mit Vorfaktor)
			\item Konstanten $C_1, C_2$ korrekt eingesetzt
			\item Betragsstriche im Logarithmus
			\item Integrationskonstante $C$ nicht vergessen
		\end{itemize}
		
	\end{enumerate}
