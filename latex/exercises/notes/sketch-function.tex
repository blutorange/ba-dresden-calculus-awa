Beim Skizzieren einer Funktionen sollten zuerst einige wesentlichen Stellen und Eigenschaften der Funktion ermittelt werden. Anhander derer wird dann eine qualitative Skizze angefertigt, in der diese Eigenschaften deutlich werden. Es ist beim Skizzieren \emph{nicht erforderlich}, alle Funktionswerte exakt einzuzeichnen. Im Folgenden sind kurz einige wesentliche Eigenschaften zusammengefasst:

\begin{itemize}
  \item \textbf{Stereotypische Funktionsgleichung}. Einige Funktionen kann man sofort skizzieren, wenn man die Form ihrer Rechenvorschrift erkennt. Dazu gehören etwa lineare ($mx+n$) und quadratische ($ax^2+bx+c$) Funktionen sowie verschobene und skalierte Sinus- und Kosinusfunktionen $A\sin(bx+c)$. Allgemein sollte man auch untersuchen, ob die gesamte Funktion aus einer anderen Funktion durch Verschiebung ($f(x-c), f(x)+c$ und Skalierung $f(kx), k\cdot f(x)$ hervorgeht.
  \item \textbf{Definitionsbereich} und falls möglich \textbf{Wertebereich}. Dieser hilft unter anderem bei der Achseneinteilung. Hierzu gehöhren auch \textbf{Unstetigkeitsstellen}, wo die Funktion nicht definiert ist.
  \item \textbf{Schnittpunkte} mit den \textbf{Koordinatenachsen}. Nullstellen erhält man durch Lösen von $f(x)=0$, der Schnittpunkt mit der Ordinate ergibt sich durch Berechnen von $f(0)$.
  \item \textbf{Symmetrie} und \textbf{Periodizität}. Gerade (achsensymmetrische) Funktionen erkennt man durch $f(x) = f(-x)$, ungerade (punktsymmetrische) Funktionen anhand $f(-x) = -f(x)$. Periodische Funktionen erkennt man häufig dadurch, dass sie trigonometrische Funktionen enthalten.
  \item \textbf{Verhalten im Unendlichen}. Dieses erhält man durch Betrachtung der Grenzwerte $\lim\limits_{x\to\pm\infty} f(x)$.
  \item \textbf{Asymptoten}. Horizontale Asymptoten liegen vor, wenn die Funktion für $x\to\infty$ sich einem endlichen Wert annähert. Kandidaten $x_P$ für vertikalen Asymptoten erkennt man etwa bei Quotienten, wenn man den Zähler auf Nullstellen untersucht. An der Polstelle muss $\lim\limits_{x\to\pm x_P} f(x) = \pm \infty$ gelten. Für gebrochenrationale Funktionen erkennt man diagonale (schräge) Asymptoten daran, dass der Grad von des Zählerpolynoms um eins größer ist als der des Nennerpolynoms. Die Funktionsgleichung der Asymptote erhält man dann mittels Polynomdivision.
  \item \textbf{Extremwerte}, \textbf{Sattelstellen}, \textbf{Wendepunkte} und \textbf{Monotoniebereiche}. Durch Lösen von $f'(x)=0$ findet man Kandidaten für lokale Extremstellen, durch Einsetzen dieser in $f''$ prüft man, ob es sich um Maxima oder Minima handelt. Für einen Sattelpunkt $x_S$ muss gelten $f'(x_S)=f''(x_S)=0$ und $f'''(x_S) \ne 0$. Wendepunkte sind lokale Extremstellen der ersten Ableitung. Zwischen lokalen Extremstellen ist die Funktion dann (falls stetig!) monoton.
\end{itemize}

