\documentclass[a4paper,10pt]{article}
\usepackage[a4paper]{geometry}
\usepackage[pdftex]{hyperref}
\usepackage[german]{babel}
\usepackage[utf8]{inputenc}
\usepackage{amssymb}
\usepackage{csquotes}
\usepackage{graphicx}
\usepackage{multicol}
\usepackage{amsmath}
\usepackage{enumitem}
\usepackage{fancyhdr}
\usepackage{indentfirst}
\usepackage{polynom}
\usepackage{multicol}

\geometry{
  headheight=14px,
  left=1.54cm,
  right=1.54cm,
  bottom=2cm,
  top=1cm
}

\setlength{\marginparsep}{1 cm}
\setlength{\topmargin}{-0.6in}
\setlength{\textheight}{9.5in}
\pagestyle{plain}

% Differential operator
\providecommand\d{}
\renewcommand{\d}[1]{\:\mathrm{d}{#1}}
\newcommand{\pdd}[2]{\frac{\partial #1}{\partial #2}}

% N-th root
% \nroot{3}{27}
\newcommand*{\nroot}[2]{\sqrt[\leftroot{-1}\uproot{2}#1]{#2}}
\newcommand*{\ncroot}[4]{\sqrt[\leftroot{#1}\uproot{#2}#3]{#4}}

% 2 component vector
% \tvect{1}{-1}
% \tvec{1}{-1}
\newcommand{\tvect}[2]{%
   \ensuremath{\Bigl(\negthinspace\begin{smallmatrix}#1\\#2\end{smallmatrix}\Bigr)}}
\newcommand{\tvec}[2]{%
    \ensuremath{\left(\negthinspace\begin{matrix}#1\\#2\end{matrix}\right)}}

% 3 component vector
% \rvect{1}{-1}{0}
% \rvec{1}{-1}{0}
\newcommand{\rvect}[3]{%
   \ensuremath{\Bigl(\negthinspace\begin{smallmatrix}#1\\#2\\#3\end{smallmatrix}\Bigr)}}
\newcommand{\rvec}[3]{%
    \ensuremath{\left(\negthinspace\begin{matrix}#1\\#2\\#3\end{matrix}\right)}}

% German-style quotation marks %
\MakeOuterQuote{"}

% Number sets
\newcommand{\N}{\mathbb{N}}
\newcommand{\Z}{\mathbb{Z}}
\newcommand{\Q}{\mathbb{Q}}
\newcommand{\R}{\mathbb{R}}
\newcommand{\C}{\mathbb{C}}

\newcommand{\setzero}{\varnothing}

% Mention (small caps)
\newcommand{\mention}[1]{\textsc{#1}}

% Units
\newcommand{\ukm}{\text{km}}
\newcommand{\uh}{\text{h}}
\newcommand{\ukmh}{\frac{\ukm}{\uh}}
\newcommand{\degrees}[1]{\SI{#1}{\degree}}

% Floor / ceil
\newcommand{\floor}[1]{\left\lfloor #1 \right\rfloor}
\newcommand{\ceil}[1]{\left\lceil #1 \right\rceil}

\newcommand{\term}[2]{$\stackrel{\hbox{\tiny \textnormal{#2}}}{\hbox{#1}}$}
% \newcommand{\term}[2]{#1}



\begin{document}

\begin{center}
{\bf {\huge Wichtige mathematische Zusammenhänge}}
\end{center}

\section {Rechenregeln}

\begin{multicols}{2}

\begin{itemize}
\item Kommutativgesetz $x \circ y = y \circ x$
\item Assoziativgesetz $x \circ (y \circ z) = (x \circ y) \circ z$
\item Distributivgesetz $x \circ (y \star z) = (x \circ y) \star (x \circ z)$
\item Bruch $\frac{a}{b} + \frac{c}{d} = \frac{ad+bc}{bd}$
\item Binom $(x+y)^2=x^2+y^2+2xy$
\item Binom $x^2-y^2=(x-y)(x+y)$
\item Potenz $a^{-x} = \frac{1}{a^x}$
\item Potenz $a^x\cdot a^y = a^{x+y}$
\item Potenz $a^x\cdot b^x = (a\cdot b)^x$
\item Potenz $\left(x^y\right)^z = x^{y\cdot z}$
\item Wurzel $\sqrt{x} \cdot \sqrt{y} = \sqrt{x\cdot y}$
\item Wurzel $\sqrt[\leftroot{-2}\uproot{2}b]{x^a} = x^{a/b}$
\item Wurzel $\sqrt{x^2} = |x|$
\item Logarithmus $\log_c(a\cdot b) = \log_c(a) + \log_c(b)$
\item Logarithmus $\log_c(a ^ b) = b\cdot \log_c(a)$
\item Logarithmus $\log_b(a) = \frac{\log_c(a)}{\log_c(b)}$
\item Trigonometrie $\tan(x) = \frac{\sin(x)}{\cos(x)}$
\item Trigonometrie $1 = \sin^2(x) + \cos^2(x)$
\end{itemize}

\section {Funktionsgraphen}
\begin{itemize}
\item Definition: $f: \mathbb{D}  \to \mathbb{B}$, $f: x \mapsto x^2$
\item Verschiebung: $f(x-c)$, $f(x)+c$
\item Skalierung: $f(a\cdot x)$, $a\cdot f(x)$
\item Achsensymmetrisch: $f(-x) = f(x)$
\item Punktsymmetrisch: $f(-x) = -f(x)$
\item Monotonie $y > x \implies f(y) > f(x)$
\item Periodizität: $f(x+p) = f(x)$
\item Verkettung: $h = f \circ g$, $h(x) = f(g(x))$
\item Nullstelle: $f(x) = 0$
\item Lokale Extrema: $f'(x) = 0$, $f''(x) \ne 0$
\end{itemize}

\section {Gleichungen}
\begin{itemize}
\item Polynom (Grad $n$): $p_n(x)=\sum_{i=0}^n a_i x^i$
\item Linearfaktoren: $p_n(x) = \sum_{i=0}^k (x-x_i)^{a_i}$, $\sum_k a_i = n$
\item $x^2+px+q=0 \implies x_{1,2} = -p/2 \pm \sqrt{p^2/4-q}$
\item Gebrochenrat. Fkt: $f(x) = \frac{p_n(x)}{q_m(x)}$
\item für $n\ge m$ unecht gebrochen, sonst echt gebrochen
\item Polynomdivision: $\frac{x^3-2x^2-4x+8}{x-1} = x^2-x-5 + \frac{3}{x-1}$
\item Partialbruchzerlegung: $\frac{x}{x^2-1} = \frac{1}{2}\cdot\frac{1}{x-1} + \frac{1}{2}\cdot\frac{1}{x+1}$
\end{itemize}

\section {Ableiten und Integrieren}
\begin{itemize}
\item $\int f(x) \diff{x} = F(x) +C$, $\dd{x}{x} F(x) = f(x)$
\item $\int_a^b f(x) = F(b) - F(a)$
\item Summenregel: $(f+g)' = f'+g'$
\item Produktregel: $(f\cdot g)' = f'g+fg'$
\item Kettenregel: $(f_1\circ f_2)' = f_1' \cdot f_2'$
\item Umkehrfunktion: $(f^{-1})' \cdot f' = 1$
\item Partielle Integration: $\int (f \cdot g) \d x = Fg - \int Fg' \d x$
\item Integration mit Substitution: Substitution ableiten und einsetzen
\end{itemize}

\end{multicols}

\newpage
\noindent

\begin{multicols}{2}


\section {Mehrstellige Funktionen}
\begin{itemize}
\item Definitionsbereich ist kartesisches Produkt
\item Mengenschreibweise: $\{ x \in \mathbb{R} | x > 0\}$
\item Mengenschreibweise: $\mathbb{R}\setminus[0,\infty)$
\item Mengenschreibweise: $[0,\infty) \times (-2,3]$
\item partielle Ableitung: $\frac{\partial f}{\partial x}(x,y) = f_x(x,y)$
\item Tangente: $t(x) = f(x_0) + (x-x_0)f'(x_0)$
\item Tangentialebene: $t(x,y) = $ \\
$f(x_0,y_0) + (x-x_0)f_x(x_0,y_0) + (y-y_0)f_y(x_0,y_0)$
\item Satz von Schwarz: $f_{xy} = f_{yx}$
\item Extremwerte (1-stellig) $f' = 0, f'' \ne 0$
\item Extremwerte (2-stellig) $f_x = f_y = 0$
\item Extremwerte (2-stellig) $f_{xx} f_{yy} - f_{xy}f_{yx} > 0$
\end{itemize}

\section {Grenzwerte}
\begin{itemize}
\item Grenzwert: $\lim_{n \to \infty} 1/x = 0$
\item Grenzwert: $\lim_{n \to \infty} q^n = 0$ ($|q| < 1$)
\item Grenzwert: $\lim_{n \to \infty} \sqrt[\leftroot{-2}\uproot{2}n]{q} = 1$ $(q > 0)$
\item Grenzwert: $\lim_{n \to \infty} \sqrt[\leftroot{-2}\uproot{2}n]{n} = 1$
\item GW-Satz: $\lim (a_n+b_n) = \lim (a_n) + \lim (b_n)$
\item GW-Satz: $\lim (a_n\cdot b_n) = \lim(a_n) \cdot \lim(b_n)$
\item GW-Satz: $\lim (f(a_n)) = f(\lim(a_n))$ (f stetig)
\item L'Hôpital: $\lim \frac{f}{g} = \lim \frac{f'}{g'}$
\item Arithmetische Reihe $s_n = 1 + 4 + 7 + 10 + ...$
\item Arithmetische Reihe $s_n = \sum_{i=1}^n{i} = \frac{n(n+1)}{2}$
\item Geometrische Reihe $s_n = 0{,}5 + 0{,}5^2 + 0{,}5^3 + ...$
\item Geometrische Reihe $s_n = \sum_{i=1}^n q^i = q\frac{1-q^n}{1-q}$
\item Quotientenkriterium $\lim_{n\to\infty} |a_{n+1} / a_n| < 1$
\item Wurzelkriterium $\lim_{n\to\infty} \nroot{n}{|a_n|} < 1$
\end{itemize}

\section {Potenzreihen}
\begin{itemize}
\item Potenzreihe: $s_n = \sum_{i=0}^n a_i x^i$
\item Konvergenzradius: $R = 1/\lim_{n \to \infty} |\frac{a_{n+1}}{a_n}|$
\item Taylorentwicklung: $f(x)=\sum_{n=0}^{\infty} \frac{f^{(n)}(x_0)}{n!} (x-x_0)^n$
\item Taylorentwicklung ist Näherung einer Funktion durch ein Polynom n-ten Grads
\item Restgliedabschätzung: $R_n(x) \le \frac{|x-x_0|^{n+1}}{(n+1)!} \max_{\vartheta \in [x_0,x]} f(\vartheta)$
\item Fourierreihe ist Entwicklung in ein Frequenzspektrum
\end{itemize}

\section {Differentialgleichung}
\begin{itemize}
\item Allgemeine Lösung $y_A$ ist Funktionsschar
\item Partikuläre Lösung $y_P$ für Anfangsbedingungen
\item Ordnung: höchste Ableitung
\item Gewöhnliche und partielle DGL
\item Explizit (höchste Ableitung alleine) und implizit
\item Gewöhnliche DGL n-ter Ordnung:\\
$\phi(y,y',...,y^{(n)},x) = 0$
\item Homogenität: Liegt vor, wenn $\phi$ homogene Funktion in den Argumenten $y$, $y'$, $...$, $y^{(n)}$ ist
\item $\phi(ty,ty',...,ty^{(n)},x) = t^\alpha \phi(y,y',...,y^{(n)},x)$
\item Zerlegung in homogenen und inhomogenen Anteil $\phi_H(y,y',...,y^{(n)},x) + \psi(x) = 0$
\item Lineare DGL: Hat homogenen Anteil mit Homogenitätsgrad 1
\end{itemize}

\section {Lösen von DGL}
\begin{itemize}
\item Grafische Lösung mittels Richtungsfeld (für DGL 1. Ordnung)
\item Probe, ob Funktion Lösung einer DGL ist: Einsetzen
\item Lösung mittels direkter Integration
\item Lösung mittels Trennung der Variablen
\item Lösung mittels Variation der Konstanten
\item Lineare DLG n-ter Ordnung: $\sum_{k=0}^n a_k(x) y^{(k)}(x) = q(x)$
\item Homogen für $q(x) = 0$, sonst inhomogen
\item Spezialfall: Lineare DGL mit konstanten Koeffizienten: $a_k = const.$
\item Allgemeine Lösung homogener Anteil: Charakteristische Gleichung
\item Partikuläre Lösung inhomogener DGL: Koeffizientenvergleich in Ansatz
\item Allgemeine Lösung inhomogener DGL: Summe obiger beider Lösungen
\end{itemize}

\end{multicols}


\end{document}

