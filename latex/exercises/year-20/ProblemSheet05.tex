\documentclass[a4paper,12pt]{article}
\usepackage{a4wide}
\usepackage[pdftex]{hyperref}
\usepackage[german]{babel}
\usepackage[utf8]{inputenc}
\usepackage{amssymb}
\usepackage{csquotes}
\usepackage{wrapfig}
\usepackage{graphicx}
\usepackage{multicol}
\usepackage{amsmath}
\usepackage{enumitem}
\usepackage{polynom}
\usepackage{siunitx}

\setlength{\marginparsep}{1 cm}
\setlength{\topmargin}{-0.6in}
\setlength{\textheight}{9.5in}
\pagestyle{plain}

\polyset{%
   style=C,
   delims={\big(}{\big)},
   div=:
}

% Differential operator
\providecommand\d{}
\renewcommand{\d}[1]{\:\mathrm{d}{#1}}
\newcommand{\pdd}[2]{\frac{\partial #1}{\partial #2}}

% N-th root
% \nroot{3}{27}
\newcommand*{\nroot}[2]{\sqrt[\leftroot{-1}\uproot{2}#1]{#2}}
\newcommand*{\ncroot}[4]{\sqrt[\leftroot{#1}\uproot{#2}#3]{#4}}

% 2 component vector
% \tvect{1}{-1}
% \tvec{1}{-1}
\newcommand{\tvect}[2]{%
   \ensuremath{\Bigl(\negthinspace\begin{smallmatrix}#1\\#2\end{smallmatrix}\Bigr)}}
\newcommand{\tvec}[2]{%
    \ensuremath{\left(\negthinspace\begin{matrix}#1\\#2\end{matrix}\right)}}

% 3 component vector
% \rvect{1}{-1}{0}
% \rvec{1}{-1}{0}
\newcommand{\rvect}[3]{%
   \ensuremath{\Bigl(\negthinspace\begin{smallmatrix}#1\\#2\\#3\end{smallmatrix}\Bigr)}}
\newcommand{\rvec}[3]{%
    \ensuremath{\left(\negthinspace\begin{matrix}#1\\#2\\#3\end{matrix}\right)}}

% German-style quotation marks %
\MakeOuterQuote{"}

% Number sets
\newcommand{\N}{\mathbb{N}}
\newcommand{\Z}{\mathbb{Z}}
\newcommand{\Q}{\mathbb{Q}}
\newcommand{\R}{\mathbb{R}}
\newcommand{\C}{\mathbb{C}}

\newcommand{\setzero}{\varnothing}

% Mention (small caps)
\newcommand{\mention}[1]{\textsc{#1}}

% Units
\newcommand{\ukm}{\text{km}}
\newcommand{\uh}{\text{h}}
\newcommand{\ukmh}{\frac{\ukm}{\uh}}
\newcommand{\degrees}[1]{\SI{#1}{\degree}}

% Floor / ceil
\newcommand{\floor}[1]{\left\lfloor #1 \right\rfloor}
\newcommand{\ceil}[1]{\left\lceil #1 \right\rceil}

\newcommand{\term}[2]{$\stackrel{\hbox{\tiny \textnormal{#2}}}{\hbox{#1}}$}
% \newcommand{\term}[2]{#1}




\begin{document}

\begin{center}
  {\bf {\large Aufgabenblatt 5 (MI/IT 2020)}}
\end{center}

% Tasks

\begin{enumerate}

% Grenzfunktion
% Punktweise / gleichmäßige Konvergenz
% Gliedweises Ableiten
% Fourierreihe, Int
% int sin mx sin nx _0^2pi berechnen (=pi für n=m, 0 sonst)
% Fourierreihe/koeff. berechnen

\item Ermitteln Sie für folgende Potenzreihen die Folge $a_i$ der Koeffizienten und bestimmen Sie den Konvergenzradius!

\begin{multicols}{3}
\begin{enumerate}
\item $\sum\limits_{i=0}^\infty i(3x)^i$
\item $\sum\limits_{i=0}^\infty \frac{(ix)^i}{2^i}$
\item $\sum\limits_{i=0}^\infty \left(x\sin(\frac{1}{i})\right)^i$
\end{enumerate}
\end{multicols}



\item
Nähern Sie die Funktion $f(x) = \ln(x)$ durch ein Taylor-Reihe 3. Grads für die Entwicklungsstelle $x_0=3$ !


\item Ermitteln Sie für folgende Funktionen die Taylor-Reihe 2. Grads an der Entwicklungsstelle $x_0=1$!

\begin{multicols}{2}
\begin{enumerate}
\item $f(x) = x^2+x$
\item $f(x) = \frac{1}{1+x}$ 
\end{enumerate}
\end{multicols}



\item 
Berechnen Sie $^3\sqrt{11}$ mittels einer Schmiegeparabel (=Taylor-Reihe 2. Grads). Schätzen Sie den Fehler mittels dem Restglied $R_2(x)$!



\item Bestimmen Sie eine Näherungslösung folgenden Gleichung $x\cdot(2+\sin(x))=\frac{1}{10}$. Dabei soll die linke Seite der Gleichung durch eine Schmiegeparabel angenähert werden. Was ist eine geeignete Entwicklungsstelle?

\item Zeigen Sie, dass die Ableitung von $f(x)=\sin(x)$ gleich $f'(x)=\cos(x)$ ist! Nutzen Sie dazu die Potenzreihendarstellung der Sinusfunktion aus der Vorlesung und bilden Sie gliedweise die Ableitung. Vergleichen Sie anschließend mit der Potenzreihe der Kosinusfunktion.

\item Gesucht ist ein Näherungswert für $\Gamma(2.8)$. Hierzu soll die Taylorentwicklung 2. Grads der Gammafunktion $\Gamma(x)$ an der Entwicklungsstelle $x_0=3$ betrachtet werden. Ermitteln Sie mit der Taylor-Reihe einen Näherungswert für $\Gamma(2.8)$ auf 2 Kommastellen genau! Schätzen Sie den Fehler auf den so ermittelten Wert auf 3 Kommstellen genau ab! Hinweis: Für $x>2$ ist $\Gamma'''(x)$ streng monoton steigend. Folgende Werte der Gammafunktion benötigen Sie möglicherweise:

\begin{multicols}{2}

$$\Gamma(n) = (n-1)! \text{ für } n \in \mathbb{N}^{+}$$

\begin{tabular}{ c | c | c | c | c}
	$\Gamma'(3)$ & $\Gamma''(3)$ & $\Gamma'''(3)$ & $\Gamma''''(3)$ & $\Gamma'''''(3)$ \\
	\hline
	1.85         &  2.50         & 3.45           &  5.52           & 8.85
\end{tabular}

\end{multicols}


\item Bestimmmen Sie das unbestimmte Integral $\int \sin(nx)\cos(mx)\diff{x}$ für ($n,m\in\N^{+}, n \ne m$) durch zweimalige partielle Integration und anschließendem Umstellen nach dem gesuchten Integral! Was erhalten Sie damit für das bestimmte Integral von $0$ bis $2\pi$? 

\item Betrachtet werde ein Sägezahnimpuls, der auf $(0, 3]$ definiert ist zu $f(x) = \frac{1}{3}x$ und für weitere Argumente periodisch fortgesetzt wird. Berechnen Sie die dessen Fourier-Reihe, d.h. ermitteln Sie \emph{alle} Fourier-Koeffizienten $a_0$, $a_n$ und $b_n$! Welchen Wert hat die Grenzfunktion an der Stelle $x=0$? Lassen Sie sich mithilfe eines Computerprogramms einige der ersten Terme der Fourier-Reihe zeichnen! [Zusatz] Berechnen Sie $\sum_{n=1}^\infty \frac{\sin(n)}{n}$, indem Sie die Fourier-Reihe an der Stelle $x=\frac{3}{2\pi}$ auswerten!

\end{enumerate}

\newpage

% Solutions

\begin{center}
{\bf {\large Lösungen}}
\end{center}

\begin{enumerate}
	
\item Ermitteln Sie für folgende Potenzreihen die Folge $a_i$ der Koeffizienten und bestimmen Sie den Konvergenzradius!

\begin{enumerate}
\item $a_i =  i\cdot3^i$. Wir berechnen:

$\lim\limits_{n\to\infty} |\frac{a_{n+1}}{a_n}|$ \\
$ = \lim\limits_{n\to\infty} |\frac{(n+1)\cdot 3^{n+1}}{n\cdot 3^n}|$ \\
$ = 3\lim\limits_{n\to\infty}\frac{n+1}{n} = 3$ \\

Damit ist der Konvergenzradius $R=\frac{1}{3}$.

\item $a_i = \frac{i^i}{2^i}$. Wir berechnen

$\lim\limits_{n\to\infty} \nroot{n}{\frac{n^n}{2^n}}$ \\
$ = \lim\limits_{n\to\infty} {\frac{n}{2}}$ \\
$ = \infty$

Damit ist der Konvergenzradius $R=0$.

\item $a_i = \sin(1/i)^i$. Wir berechnen:

$\lim\limits_{n\to\infty} \nroot{n}{\sin(1/n)^n}$ \\
$ = \lim\limits_{n\to\infty} \sin(1/n) $ \\
$ = \sin(\lim\limits_{n\to\infty}(1/n)) $ \\
$ = \sin(0) = 0$ \\

Damit ist der Konvergenzradius $R=\infty$.

\end{enumerate}



\item $f(x) = \ln(x)$. Ableiten

$$f'(x) = \frac{1}{x}$$
$$f''(x) = -\frac{1}{x^2}$$
$$f'''(x) = \frac{2}{x^3}$$

Für die Entwicklungsstelle $x_0 = 3$:

$$f(3) = \ln(3)$$
$$f'(3) = \frac{1}{3}$$
$$f''(3) = -\frac{1}{9}$$
$$f'''(3) = \frac{2}{27}$$

Damit folgt für die Taylorentwicklung 3. Grades:

$$\ln(x) \approx \ln(3) + \frac{1}{3} (x-3) - \frac{1}{18} (x-3)^2 + \frac{1}{81} (x-3)^3$$

Anmerkung: $f''''(x)=\frac{6}{x^4}$. Für $\Delta x \in [0,3)$ ist $\max\limits_{\vartheta\in[3-\Delta x,3+\Delta x]}|f''''(\vartheta)| = \frac{6}{(3-\Delta x)^4}$. Das Restglied kann dann abgeschätzt werden mit $|R_3(x)| \leq \frac{|x-3|^4}{4!} \frac{6}{(3-|x-3|)^4}$.



\item Ermitteln Sie für folgende Funktionen das Taylorpolynom 2. Grads an der Entwicklungsstelle $x_0=1$!

\begin{enumerate}
\item $f'(x) = 2x+1$, $f''(x) = 2$. Weiterhin $f(1) = 2$, $f'(1) = 3$, $f''(1) = 2$. Damit folgt für das Taylorpolynom $p_2(x) = 2 + 3(x-1) + (x-1)^2$.
\item $f'(x) = -\frac{1}{(1+x)^2}$, $f''(x) = \frac{2}{(1+x)^3}$ Weiterhin $f(1) = \frac{1}{2}$, $f'(1) = -\frac{1}{4}$, $f''(1) = \frac{1}{4}$. Damit folgt für das Taylorpolynom $p_2(x) = \frac{1}{2} - \frac{1}{4}(x-1) + \frac{1}{8}(x-1)^2$.
\end{enumerate}


\item Wir betrachten die Funktion $f(x) = \nroot{3}{x} = x^\frac{1}{3}$. Wir wissen, dass $f(8) = \nroot{3}{8} = 2$ gilt. Also Taylor-Entwicklung 2. Grades um $x_0=2$:

$$f(x) \approx f(8) + f'(8) \cdot (x-8) + \frac{f''(8)}{2} \cdot (x-8)^2$$

Die Ableitungen sind:

$$f'(x) = \frac{1}{3} x^{-\frac{2}{3}}$$
$$f''(x) = -\frac{2}{9} x^{-\frac{5}{3}} $$
$$f^{(3)}(x) = \frac{10}{27} x^{-\frac{8}{3}} $$


An der Stelle $x=8$ somit:

$$f'(8) = \frac{1}{3} * (\nroot{3}{8})^{-2} = \frac{1}{12}$$
$$f''(8) = - \frac{2}{9} * (\nroot{3}{8})^{-5} = -\frac{1}{144}$$

Damit ergibt sich für die die Taylorentwicklung:

$$f(x) \approx 2 +\frac{1}{12}(x-8) - \frac{1}{288} (x-8)^2$$

Und für den gesuchten Wert:

$$f(11) = \nroot{3}{11} \approx 2 + \frac{1}{4} - \frac{1}{32} = \frac{71}{32} \approx 2,219$$

Anmerkung: Mit der Restgliedabschätzung folgt unter Verwendung von $\max\limits_{\vartheta \in [8,11]}|f^{(3)}(\vartheta)| = \frac{5}{3456}$ für den Höchstfehler $|R_2(x)| \leq \frac{(11-8)^3}{3!}\cdot\frac{5}{3456} = \frac{5}{768} \approx 0,007$. Also $\nroot{3}{11} \in [2,212;2,226]$. Der tatsächliche Wert ist $\nroot{3}{11} = 2.2239800905...$


\item Wir leiten ab:

$$f'(x) = 2+\sin(x)+x\cos(x)$$
$$f''(x) = 2\cos(x)-x\sin(x)$$
$$f'''(x) = -3\sin(x)-x\cos(x)$$
$$f''''(x) = -4\cos(x)+x\sin(x)$$

Es ist $f(0) = 0$, was bereits nahe an $\frac{1}{10}$ liegt. Daher empfiehlt sich die Entwicklungsstelle $x_0=0$:

$$f'(0) = 2$$
$$f''(0) = 2$$
$$f'''(0) = 0$$
$$f''''(0) = -4$$

Da die dritte Ableitung $0$ ist, ist die Schmiegeparabel zugleich auch die Taylorentwicklung 3. Grades.

$$f(x) \approx 0 + \frac{2}{1!}(x-0) + \frac{2}{2!}(x-0)^2 = x^2+2x$$

Nun soll der Funktionswert $\frac{1}{10}$ betragen:
$$f(x) \stackrel{!}{=} \frac{1}{10} = x^2+2x$$
$$\implies x_{1,2} = -1 \pm \sqrt{\frac{11}{10}}$$

Da wir positive x-Werte betrachten, kommt nur die positive Lösung $x = -1 + \sqrt{\frac{11}{10}} \approx 0,0488$ in Frage.



\item Für die Sinusfunktion gilt:

$$
	\sin(x) = x - \frac{1}{3!}x^3 + \frac{1}{5!} x^5 \mp \dots
$$

Da Potenzreihen gleichmäßig konvergieren, können wir sie gliedweise ableiten. Wir erhalten

\begin{alignat*}{1}
	\dd{}{x} \sin(x) &= \dd{}{x} \left(x - \frac{1}{3!}x^3 + \frac{1}{5!} x^5 \mp \dots \right) \\
					 &= 1-\frac{3}{3!}x^2 + \frac{5}{5!} x^4 \mp \dots \\
					 &= 1-\frac{1}{2!}x^2 + \frac{1}{4!} x^4 \mp \dots \\
					 &= \cos(x)
\end{alignat*}

Analog lässt sich zeigen, dass $\dd{}{x} \cos(x) = -\sin(x)$ gilt.

\item Wir nutzen die Formel für die Taylor-Reihe mit den gegebenen Werten und rechnen aus:

\begin{alignat*}{1}
	\Gamma(x)     & \approx 2 + \Gamma'(3)(x-3) + \frac{\Gamma''(3)}{2!}(x-3)^2 \\
	              & = 2 + 1.85 \cdot (x-3) + 1.25 \cdot (x-3)^2 \\
	\Gamma(2.8)   & \approx 2.00 - 1.85 / 5 + 1.25\cdot {0.04} \\
	              & = 2.00 - 0.37 + 0.05 \\
	              & = 1.68
\end{alignat*}

Für die Abschätzung des Fehlers nehmen wir die Formel für die Restgliedabschätzung, setzen ein und rechnen aus:

$$
	|R_2(2.8)| \le \frac{0.2^3}{6}\cdot\max\limits_{\vartheta \in [2.8;3.0]} |\Gamma'''(x)|
$$

Da $\Gamma'''$ wie in der Aufgabe angegeben monoton steigend ist, beträgt der Maximalwert $\Gamma'''(3)= 3.45$. Damit folgt:

\begin{alignat*}{1}
	|R_2(2.8)| & \le \frac{0.2^3}{6}\cdot 3.45 \\
	           & = 0.008\cdot 0.575 \\
	           & = 0.0046 \\
	           & \approx 0.005
\end{alignat*}



\item Wir führen zunächst zweimal partielle Integration durch:

\begin{alignat*}{1}
	I &= \int \sin nx \cos mx \diff{x} \\
	  &= -\frac{1}{n} \cos nx \cos mx - \int \frac{m}{n} \cos nx \sin mx \diff{x} \\
	  &= -\frac{1}{n} \cos nx \cos mx - \frac{m}{n} \left\lbrace \frac{1}{n} \sin nx \sin mx - \int \frac{m}{n} \sin nx \cos mx \right\rbrace \\
	  &= -\frac{1}{n} \cos nx \cos mx - \frac{m}{n^2} \sin nx \sin mx + \frac{m^2}{n^2} \int \sin nx \cos mx \diff{x} \\
	  &= -\frac{1}{n} \cos nx \cos mx - \frac{m}{n^2} \sin nx \sin mx + \frac{m^2}{n^2} I
\end{alignat*}

Wir stellen nach dem gesuchten Integral $I$ um:

\begin{alignat*}{1}
	\left( 1-\frac{m^2}{n^2} \right) I &= -\frac{1}{n} \cos nx \cos mx - \frac{m}{n^2} \sin nx \sin mx \\
	\frac{n^2-m^2}{n^2} I              &= -\frac{1}{n} \cos nx \cos mx - \frac{m}{n^2} \sin nx \sin mx \\
									 I &= -\frac{m}{n^2-m^2} \sin nx \sin mx - \frac{n}{n^2-m^2} \cos nx \cos mx \\
		 \int \sin nx \cos mx \diff{x} &= \frac{m \sin nx \sin mx + n \cos nx \cos mx}{m^2-n^2} + C
\end{alignat*}

Wichtig hierbei ist die Voraussetzung, dass $n \ne m$, da wir sonst eine Division durch $0$ im Nenner durchführen würden. Für das bestimmte Integral erhalten wir, da $\sin(2\pi n) = 0$ und $\cos(2\pi n) = 1$ ist:

$$
	\int\limits_0^{2\pi} \sin nx \cos mx \diff{x} = 0 (\text{ für } n \ne m)
$$

Diese Art der Integrale haben wir bei der Herleitung der Fourier-Reihe benötigt.

Anmerkung: Für $n=m$ ergibt sich das Integral $\int \sin(nx)\cos(nx)\diff{x}$. Mit dem aus der Vorlesung bekannten Integral $\int \sin(x)\cos(x)\diff{x} = \frac{1}{2}\sin^2(x)+C$ und linearer Substitution erhalten wir $\int \sin(nx)\cos(nx)\diff{x} = \frac{1}{2n}\sin^2(nx) + C$. Auch hier ist das bestimmte Integral von $0$ bis $2\pi$ also $0$.

\item Die Periode ist $T=3$. Mithilfe der Formel aus der Vorlesung folgt für den Fourier-Koeffizienten $a_0$:

\begin{alignat*}{1}
	a_0 &= \frac{1}{T} \int\limits_0^T f(x) \diff{x} \\
	    &= \frac{1}{3} \int\limits_0^3 \frac{1}{3}x \diff{x} \\
	    &= \frac{1}{3} \cdot \frac{1}{6} x^2 \Biggr|_0^3 \\
	    &= \frac{1}{3} \cdot \frac{9}{6} \\
	    &= \frac{1}{2}
\end{alignat*}

Dies hätten wir uns auch geometrisch überlegen können, da $a_0$ den Durchschnittswert der Funktion darstellt.

Für $a_n$ folgt:

\begin{alignat*}{1}
	a_n &= \frac{2}{T} \int\limits_0^T f(x) \cos(\frac{2\pi}{T}nx) \diff{x} \\
	    &= \frac{2}{3} \cdot \frac{1}{3} \int\limits_0^3 x \cos(\frac{2\pi}{3}nx) \diff{x}
\end{alignat*}

Dieses Integral lösen wir, indem wir zuerst lineare Substitution $z=\frac{2\pi}{3}nx$ anwenden:

$$
	\int\limits_0^3 x \cos(\frac{2\pi}{3}nx) \diff{x} = \left(\frac{3}{2\pi n}\right)^2 \int\limits_0^{2\pi n} z \cos(z) \diff{z}
$$

Und anschließend partielle Integration anwenden:

\begin{alignat*}{1}
	\int\limits_0^{2\pi n} z \cos(z) \diff{z} &= z \sin(z)\biggr|_0^{2\pi n} - \int\limits_0^{2\pi n} \sin(z) \diff{z} \\
	                                          &= 0 + \cos(z)\biggr|_0^{2\pi n} \\
	                                          &= 0
\end{alignat*}

Damit sind alle $a_1,a_2,\dots=0$. Schließlich berechnen wir noch die Koeffizienten $b_n$:

\begin{alignat*}{1}
	b_n &= \frac{2}{T} \int\limits_0^T f(x) \sin(\frac{2\pi}{T}nx) \diff{x} \\
	    &= \frac{2}{3} \cdot \frac{1}{3} \int\limits_0^3 x \sin(\frac{2\pi}{3}nx) \diff{x}
\end{alignat*}

Auch hier wenden wir lineare Substitution an:

$$
	\int\limits_0^3 x \sin(\frac{2\pi}{3}nx) \diff{x} = \left(\frac{3}{2\pi n}\right)^2 \int\limits_0^{2\pi n} z \sin(z) \diff{z}
$$

Und weiter mit partieller Integration:

\begin{alignat*}{1}
	\int\limits_0^{2\pi n} z \sin(z) \diff{x} &= -z \cos(z)\biggr|_0^{2\pi n} + \int\limits_0^{2\pi n} \cos(z) \diff{z} \\
	                                          &= -z \cos(z)\biggr|_0^{2\pi n} + \sin(x)\biggr|_0^{2\pi n} \\
	                                          &= -2\pi n
\end{alignat*}

Somit folgt für das Integral

\begin{alignat*}{1}
	\int\limits_0^3 x \sin(\frac{2\pi}{3}nx) \diff{x} &= \left( \frac{3}{2\pi n}\right)^2 \int\limits_0^{2\pi n} z \sin(z) \diff{z} \\
	                                                  &= \frac{3}{2\pi n}  \cdot \frac{3}{2\pi n} \cdot (-2\pi n) \\
	                                                  &= -\frac{9}{2\pi n}
\end{alignat*}

Nun erhalten wir endlich für $b_n$:

\begin{alignat*}{1}
	b_n &= \frac{2}{3} \cdot \frac{1}{3} \int\limits_0^3 x \sin(\frac{2\pi}{3}nx) \diff{x} \\
	    &= \frac{2}{3} \cdot \frac{1}{3} \cdot \left(-\frac{9}{2\pi n}\right) \\
	    &= -\frac{1}{\pi n}
\end{alignat*}

Die Grenzfunktion $F$ der Fourier-Reihe lautet damit:

$$
	F(x) = \frac{1}{2} - \sum\limits_{n=1}^\infty \frac{\sin(\frac{2\pi}{3}nx)}{\pi n}
$$

An der Stelle $x=0$ liegt eine Unstetigkeitsstelle vor. Während die Funktion dort den Wert $f(0)=1$ besitzt, weist die Grenzfunktion der Taylor-Reihe wegen der Dirichletschen Bedingung den Wert $F(0)=\frac{1}{2}$ auf (dieser Wert ergibt sich auch, wenn man in die Formel oben explizit $x=0$ einsetzt.)

In Abbildung \ref{fig:FourierChainsaw} sind die ersten Glieder dieser Fourier-Reihe dargestellt.

Für $x=\frac{3}{2\pi}$ erhalten wir (da $F(x)=\frac{1}{3}x$ in $(0,3]$):

\begin{alignat*}{1}
	F(\frac{3}{2\pi}) &= \frac{1}{2} - \sum\limits_{n=1}^\infty \frac{\sin(\frac{2\pi}{3}n\frac{3}{2\pi})}{\pi n} \\
	                  &= \frac{1}{2} - \sum\limits_{n=1}^\infty \frac{\sin(n)}{\pi n} \\
	                  &= \frac{1}{3} \cdot \frac{3}{2\pi} \\
	                  &= \frac{1}{2\pi}
\end{alignat*}

Umgestellt nach der Summe erhalten wir so:

$$
	\sum\limits_{n=1}^\infty \frac{\sin(n)}{\pi n} = \frac{1}{2}-\frac{1}{2\pi}
$$

Nach Multiplikation mit $\pi$ ergibt sich schließlich:

$$
	\sum\limits_{n=1}^\infty \frac{\sin(n)}{n} = \frac{\pi-1}{2}
$$

\begin{figure}
	\centering
	\includegraphics[width=0.48\textwidth]{../gnuplot/ex-fourier-2-img-a-1}
	\includegraphics[width=0.48\textwidth]{../gnuplot/ex-fourier-2-img-a-2}
	\includegraphics[width=0.48\textwidth]{../gnuplot/ex-fourier-2-img-a-3}
	\includegraphics[width=0.48\textwidth]{../gnuplot/ex-fourier-2-img-a-4}
	\includegraphics[width=0.48\textwidth]{../gnuplot/ex-fourier-2-img-a-5}
	\includegraphics[width=0.48\textwidth]{../gnuplot/ex-fourier-2-img-a-6}
	\caption{Die ersten $n$ Glieder der Fourier-Reihe einer Sägezahnfunktion}
	\label{fig:FourierChainsaw}
\end{figure}

\end{enumerate}

\end{document}

