\item

Die Wassermenge (in Litern) im Tank sei bezeichnet mit $V$. Die Zuflussrate beträgt $\sigma_z = 10 \frac{l}{s}$ (Liter pro Sekunde). Die Abflussrate ist $\sigma_a = -\frac{1}{1000} V$ (Promille = pro Mille(Tausend) = 1 Tausendstel).

Zu jedem Zeitpunkt setzt sich die Änderung $dV$ der Wassermenge zusammen aus Zufluss- und Abflussrate:

$$\frac{\d V}{\d t} = V'(t) = \sigma_z + \sigma_a = \sigma_z - \frac{1}{1000}V$$
$$\implies V' + \frac{1}{1000} V = \sigma_z = 10 \frac{l}{s}$$

Es handelt sich um eine DGL vom Ordnung 1, gewöhnlich, explizit, inhomogen, linear.

Bei 7 Sekunden enthält der Tank 98 Liter. Die Anfangsbedingung lautet also:

$$V(7s) = 98l$$

