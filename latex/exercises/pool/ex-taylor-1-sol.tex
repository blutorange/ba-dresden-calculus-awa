\item $f(x) = \ln(x)$. Ableiten

$$f'(x) = \frac{1}{x}$$
$$f''(x) = -\frac{1}{x^2}$$
$$f'''(x) = \frac{2}{x^3}$$

Für die Entwicklungsstelle $x_0 = 3$:

$$f(3) = \ln(3)$$
$$f'(3) = \frac{1}{3}$$
$$f''(3) = -\frac{1}{9}$$
$$f'''(3) = \frac{2}{27}$$

Damit folgt für die Taylorentwicklung 3. Grades:

$$\ln(x) \approx \ln(3) + \frac{1}{3} (x-3) - \frac{1}{18} (x-3)^2 + \frac{1}{81} (x-3)^3$$

Anmerkung: $f''''(x)=\frac{6}{x^4}$. Für $\Delta x \in [0,3)$ ist $\max\limits_{\vartheta\in[3-\Delta x,3+\Delta x]}|f''''(\vartheta)| = \frac{6}{(3-\Delta x)^4}$. Das Restglied kann dann abgeschätzt werden mit $|R_3(x)| \leq \frac{|x-3|^4}{4!} \frac{6}{(3-|x-3|)^4}$.

