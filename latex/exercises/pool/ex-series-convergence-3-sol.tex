\item Man berechnet jeweils den Grenzwert $L$ nach Wurzel- oder Quotientenkriterium.


\begin{enumerate}[label=(\arabic*)]
	\item $L=\infty$, also divergent
	\item $L=3/7$, also konvergent
	\item $L=1$, also keine Aussage möglich
	\item $L=12$, also divergent
	\item $L=12$, also divergent
	\item $L=12$, also divergent (es ist egal, wo $n$ anfängt)		
	\item $L=1/4$, also konvergent
	\item $L=0$, also konvergent
	\item $L=\infty$, also divergent
	\item $L=1/e$, also konvergent
	\item $L=0$, also konvergent		
	\item $L=1/e$, also konvergent		
	\item $L=e$, also divergent
	\item $L=1$, also keine Aussage möglich (tatsächlich divergiert die Reihe)
	\item $L=1$, also keine Aussage möglich (tatsächlich konvergiert die Reihe gegen $\pi^2/6$)
	\item $L=0$, also konvergent		
	\item $L=1/5$, also konvergent		
	\item $L=\infty$, also divergent
	\item $L=1/3$, also konvergent
	\item $L=1/4$, also konvergent
\end{enumerate}
