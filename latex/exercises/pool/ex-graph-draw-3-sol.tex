\item

\begin{enumerate}

\item Es handelt sich um eine abschnittsweise (stückweise) definierte Funktion. Wir unterscheiden 3 Bereiche:

\begin{itemize}
\item $x<0$: Lineare Funktion mit negativen Anstieg, welche die Ordinate bei $y=a$ schneidet.
\item $0\le x <2$: Nach oben geöffnete Parabel, deren Scheitelpunkt sich im Koordinatenursprung befindet.
\item $x \ge 2$: Konstante Funktion mit dem Funktionswert $y=1$.
\end{itemize}

\item Damit die Funktion stetig ist, müssen die Funktionswerte an den Grenzstellen zwischen den 3 Bereichen gleich sein (formal: links- und rechtsseitiger Grenzwert müssen übereinstimmen). Wir erhalten:

\begin{itemize}
\item $a-0 = b \cdot x^2$, also $a=0$
\item $b\cdot 2^2 = 1$, also $b=\frac{1}{4}$
\end{itemize}

\end{enumerate}

