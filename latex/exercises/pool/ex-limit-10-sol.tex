\item Zur Erinnerung: Eine Folge $(a_n)$ heißt Konvergent für $n\to\infty$ mit dem Grenzwert $\alpha$, wenn gilt:

$$
\forall \varepsilon > 0 \exists n_0\in\N: \forall n > n_0 : |a_n - \alpha| < \varepsilon
$$

Nun ist $\alpha=1$. Unsere Aufgabe besteht darin, zu zeigen, dass es für jedes vorgegebene $\varepsilon$ ein $N=N(\varepsilon)$ gibt, was obige Bedingung erfüllt. Dazu betrachten wir die Ungleichung:

\begin{alignat*}{1}
	|a_n - 1| &< \varepsilon \\
	\left|\frac{n-1}{n+1} - 1\right| &< \varepsilon \\
	\left|1 - \frac{n-1}{n+1}\right| &< \varepsilon	
\end{alignat*}

Zuerst müssen wir den Betrag auswerten. Für $n \ge 0$ ist der Ausdruck $\frac{n-1}{n+1}$ kleiner $1$, denn

\begin{alignat*}{1}
	\frac{n-1}{n+1} &< 1 \\
	n -1 &< n + 1 \\
	0 &< 2
\end{alignat*}


Weiterhin ist für alle $n>1$ der Ausdruck $\frac{n-1}{n+1}$ positiv (Quotient von positiven Zahlen). Da wir nur große $n$ betrachten, sind die ersten Glieder der Zahlenfolge für Grenzwertbetrachtungen irrelevant. Somit können wir im Folgenden davon ausgehen, dass $1 - \frac{n-1}{n+1} > 0$ gilt und wir erhalten:

$$
	1 - \frac{n-1}{n+1} < \varepsilon
$$

Diese Ungleichung können wir lösen und erhalten:

\begin{alignat*}{1}
	1-\varepsilon &< \frac{n-1}{n+1} \\
	(1-\varepsilon)(n+1) &< n -1 \\
	-\varepsilon n &< \varepsilon - 2 \\
	n &> \frac{2-\varepsilon}{\varepsilon} = \frac{2}{\varepsilon} - 1
\end{alignat*}

Für alle $\varepsilon > 0$ erhalten wir so einen endlichen Wert. Damit ist die Definition des Grenzwerts für $N(\varepsilon) = \frac{2}{\varepsilon} - 1$ erfüllt, $\alpha = 1$ ist also tatsächlich Grenzwert.
