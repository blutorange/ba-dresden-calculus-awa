\item Wir erkennen zuerst, dass in $6e^{-4x}-33e^{-5x}+x$ der erste Term $6e^{-4x}-33e^{-5x}$ der allgemeinen Lösung der homogenen DGL und der zweite Term $x$ der partikulären Lösung der inhomogenen DGL enstpricht.

Die Vorfaktoren vor den Exponentialfunktionen sind irrelevant, auch andere Vorfakten ergeben eine Lösung. Relevant sind die Exponenten $-4x$ und $-5x$. Daraus lesen wir ab, dass $\lambda_1=-4$ und $\lambda_2=-5$ die beiden Lösungen der charakteristischen Gleichung sind. Diese lautet daher 

$$
 (\lambda+4)(\lambda+5) = \lambda^2+9\lambda+20
$$.

Die homogene lineare DGL, aus der sich diese charakteristischen Gleichung ergibt, ist also

$$
	y''+9y'+20y = 0
$$

Der Term $x$ soll eine Lösung der inhomogenen DGL sein. Wie lautet also die Inhomogenität der DGL? Um das herauszufinden, setzen wir die partikuläre Lösung $y=x$ ein:

$$
	y''+9y'+20y = 0+9+20x = 20x+9
$$

Zusammengefasst: $20x+9$ ist die Inhomogenität und eine mögliche DGL, welche die geforderte Lösung besitzt, lautet  $y''+9y'+20y=20x+9$.