\item

Wir berechnen die Gleichung der Tangentengerade in Abhängigkeit vom Anlagepunkt $x_0$.

Der Anstieg $m$ der Tangenten folgt aus der 1. Ableitung:

$f'(x) = -\frac{1}{x^2}$

$\implies f'(x_0) = -\frac{1}{x_0^2}$

Der Funktionswert an der Stelle $x_0$ ergibt sich zu $f(x_0) = \frac{1}{x_0}$. Die Tangentengleichung $t(x)$ lautet somit:

$t(x) = f(x_0) + f'(x_0) (x-x_0) = \frac{1}{x_0} -\frac{1}{x_0^2} (x-x_0)$

Die Tangente schneidet die x-Achse in der Nullstelle $x_z$:

$t(x_z) = 0 = \frac{1}{x_0} -\frac{1}{x_0^2} (x_z-x_0)$

$\implies \frac{1}{x_0} = \frac{1}{x_0^2} (x_z-x_0) $

$\implies x_0 = x_z-x_0 $

$\implies x_z = 2x_0 $

Nach Aufgabestellung soll der Schnittpunkt der Tangente mit der x-Achse bei $7$ liegen, also $x_z = 2x_0 = 7$, somit $x_0 = 3.5$.

