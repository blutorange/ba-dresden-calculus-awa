\item Wir berechnen zuerst die Ableitungen, setzen dann in die Gleichung ein und prüfen, ob diese erfüllt ist.

Ableitung nach $x$:

\begin{alignat*}{1}
	\pdd{}{x} A\sin(x\pm vt)     &= A \cos(x\pm vt) \\
	\pddn{2}{}{x} A\sin(x\pm vt) &= -A\sin(x\pm vt)
\end{alignat*}


Ableitung nach $t$:

\begin{alignat*}{1}
	\pdd{}{t} A\sin(x\pm vt)     &= A \cdot (\pm v) \cdot \cos(x\pm vt) \\
	\pddn{2}{}{t} A\sin(x\pm vt) &= -A \cdot (\pm v)^2 \cdot \sin(x\pm vt) \\
	                             &= -A v^2 \sin(x\pm vt)
\end{alignat*}

Eingesetzt in die DGL:

\begin{alignat*}{1}
	\pddn{2}{y}{x}                 &= \frac{1}{v^2}\pddn{2}{y}{t} \\
	\left( -A\sin(x\pm vt) \right) &= \left( \frac{1}{v^2} \right) \cdot \left( -A v^2 \sin(x\pm vt) \right) \\
	       -A\sin(x\pm vt)         &= -A \sin(x\pm vt)
\end{alignat*}

Beide Seiten der Gleichung sind identisch, also sind $y(x,t)=A\sin(x\pm vt)$ für beliebige Konstanten $A$ Lösungen der Wellengleichung. Anmerkung: Diese DGL beschreibt die Schwingung eines idealen Seils.