\item

Wir betrachten zuerst die homogene DGL $y'+3y=0$. Das charakteristische Polynom ist $\lambda + 3 = 0$, also $\lambda = -3$. Die allgemeine Lösung ist der homogenen DGL ist mithin $y_h(x) = C e^{-3x}$.

Für eine partikuläre Lösung machen den Ansatz $y_p = a\cdot x + b$. Dessen Ableitung ist $y_p' = a$. Wir setzen den Ansatz ein in die inhomogene DGL:

$$y_p'+3y_p-x=0$$ 
$$\implies a + 3 (ax+b) -x = 0$$ 
$$\implies (a + 3b) + x \cdot (3a-1) = 0$$ 

Diese Gleichung muss für alle Werte von $x$ erfüllt sein. Wir führen einen Koeffizientenvergleich durch:

\begin{samepage}
$$(i)\enspace a+3b = 0$$
$$(ii)\enspace 3a-1 = 0$$
\end{samepage}

Aus Gleichung $(ii)$ folgt $a=\frac{1}{3}$. Eingesetzt in Gleichung $(i)$ folgt $\frac{1}{3} + 3b = 0$, somit $b=-\frac{1}{9}$.

Eine partikuläre Lösung der inhomogenen DGL lautet also:

$$y_p(x) = \frac{1}{3} x - \frac{1}{9}$$

Die allgemeine Lösung der inhomogenen DGL ist mithin:

$$y(x) = y_h + y_p = C e^{-3x} + \frac{1}{3} x - \frac{1}{9}$$

Für eine partikuläre Lösung $y$ der inhomogenen DGL, die das Anfangswertproblem $y(0) = 4$ erfüllt, setzen wir ein:

$$y(0) = 4 = C\cdot e^{-3\cdot 0} + \frac{1}{3}\cdot 0 - \frac{1}{9} = C-\frac{1}{9}$$

Es ergibt sich $C=4+\frac{1}{9}=\frac{37}{9}$ und das Ergebnis lautet:

$$y(x) = \frac{37}{9} e^{-3x} + \frac{1}{3} x - \frac{1}{9}$$

