\item Das Quadrat hat eine Gesamtfläche von $a^2$. Im ersten Schritt wird $\frac{1}{4}$ der Gesamtfläche ausgemalt. Im zweiten Schritten wird vom verbleibenden Viertel erneut ein Viertel ausgemalt: $\left(\frac{1}{4}\right)^2$. Im dritten Schritt dann $\left(\frac{1}{4}\right)^3$ etc. Wir erhalten eine geometrische Reihe für den gesamten Flächeninhalt:

\begin{alignat*}{1}
	A &= a^2 \sum\limits_{n=1}^\infty \left(\frac{1}{4}\right)^n \\
	  &= a^2 \left(\sum\limits_{n=0}^\infty \left(\frac{1}{4}\right)^n - 1\right)
\end{alignat*}

Mit der Summenformel für geometrische Reihen ergibt sich damit:

$$
	A = a^2 \left(\frac{1}{1-1/4} - 1\right) = a^2 (4/3-1) = \frac{1}{3}a^2
$$
