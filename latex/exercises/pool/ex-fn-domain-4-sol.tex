\item 
Ein Quotient ist nur für Divisoren ungleich $0$ definiert, es muss für den Nenner des Bruchs also gelten $0 \ne (2-x)(y+1)^2$. Ein Produkt ist $0$, wenn wenigstens ein Faktor $0$ ist. Somit darf $x$ nicht den Wert $2$ und $y$ nicht den Wert $-1$ annehmen. Die Bildmenge lautet mithin:

$$\mathbb{B} = \lbrace (x,y) \in \mathbb{R}^2 | x \ne 2 \land x \ne -1 \rbrace$$

Alternative Schreibweisen mittels Mengenoperatoren:

$$\mathbb{B} = \mathbb{R}^2 \setminus ((\lbrace 2 \rbrace \times \mathbb{R})  \cup (\mathbb{R} \times \lbrace -1 \rbrace))$$

\begin{alignat*}{5}
  \mathbb{B} &=     &&[ \; (-\infty;2)  &&\times  (-\infty;-1) \; &&] \\
             &\cup  &&[ \; (-\infty;2)  &&\times  (-1;\infty)  \; &&] \\
             &\cup  &&[ \; (2;\infty)   &&\times  (-\infty;-1) \; &&] \\
             &\cup  &&[ \; (2;\infty)   &&\times  (-1;\infty)  \; &&]
\end{alignat*} 

