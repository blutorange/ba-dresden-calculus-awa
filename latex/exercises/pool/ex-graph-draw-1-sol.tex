\item Im Folgenden kurz die Überlegungen, die man anstellen sollte, um den jeweiligen Graph skizzieren zu können. Eine Skizze der Graphen befindet sich am Ende.

\begin{enumerate}

\item Umschreiben als: $f(x) = \frac{1}{3}\sin(2(x-\frac{3}{8}\pi))$

Sinusfunktion, die (i) um $\frac{3}{8}{\pi}$ ($67,5^\circ$) nach rechts verschoben ist, (ii) entlang der x-Achse um den Faktor 2 gestaucht und (iii) entlang der y-Achse um den Faktor $3$ gestaucht ist.

\item $f(x) = \sin(x^2)$

Sinusfunktion, deren Argument immer stärker steigt, somit wird die Periode für betragsmäßig große Argument immer kleiner, die Sinusfunktion oszilliert also immer schneller. Wegen $f(x) = f(-x)$ ist die Funktion gerade, also symmetrisch bzgl. der y-Achse. Zudem $f(0) = 0$, der Graph beginnt im Ursprung.

\item $f(x) = e^\frac{1}{x}$

Wir überlegen uns einige wesentliche Eigenschaften der Funktion. Sie hat keine Nullstellen. $f'(x)=-\frac{1}{x^2}e^\frac{1}{x}$, also auch keine Extremstellen. Für $x=0$ ist die Funktion nicht definiert. Nahe dieser Stelle ist $\lim\limits_{x\to 0^{-}} e^\frac{1}{x} = 0$ und $\lim\limits_{x\to 0^{+}} e^\frac{1}{x} = \infty$. Das Verhalten im Unendlichen ist $\lim\limits_{x\to\pm\infty} e^\frac{1}{x} = e^0 = 1$.

Folglich geht der Graph für $x < 0$ im Ursprung los und nähert sich mit kleiner werdenden x-Werten der Gerade $y=1$ an. Für $x > 0$ kommt der Graph aus dem Positiv-Unendlichen und nähert sich $y=1$ an.

\item $f(x)=\sin^2(x)$

Da quadriert wird, werden alle negativen Werte ins Positive gebracht, damit ist die Periode halb so groß, also $\pi$. Alle Werte des Sinus sind betragsmäßig $\le 1$, werden durch das Quadrieren also kleiner, sodass der Graph zugespitzt wird.

\item $f(x)=\frac{1}{\sin(x)}$

Gleiche Periode wie Sinusfunktion, also $2\pi$. Die Nullstellen werden zu Polstellen. Bei der Nullstelle bei $0^\circ$ geht der Graph  links davon ins Negative-Unendliche und kommt rechts davon aus dem Positiv-Unendlichen. Bei der Nullstelle $180^\circ$  ist es umgekehrt: links davon geht der Graph ins Positiv-Unendliche, rechts davon kommt er aus dem Negativ-Unendlichen.

Bei $90^\circ$ gibt es ein lokales Minima mit $f(90^\circ) = \frac{1}{\sin(90^\circ)} = 1$ (da $\frac{1}{\sin(x)}$ am kleinsten wenn $\sin(x)$ am größten) und bei $270^\circ$ ein lokales Maxima mit $f(270^\circ) = -1$.

\end{enumerate}

