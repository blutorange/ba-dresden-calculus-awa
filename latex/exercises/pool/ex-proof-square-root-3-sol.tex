\item Es gelte

$$
\sqrt{3} = p/q
$$

mit $p,q\in\Z$. Dann folgt

\begin{alignat*}{1}
3 &= \frac{p^2}{q^2} \\
3 q^2 &= p^2
\end{alignat*}

Daran erkennt man, $p^2$ ist das Vielfache einer ganzen Zahlen, also durch $3$ teilbar. Dann ist aber auch $p$ durch 3 teilbar, denn in Modulo-Rechnung zum Modul $3$ gilt:

\begin{alignat*}{1}
0^2 & \equiv 0 \mod 3 \\
1^2 & \equiv 1 \mod 3 \\
2^2 & \equiv 1 \mod 3
\end{alignat*}

Ist $p$ durch $3$ teilbar, können wir schreiben

$$
p = 3 n
$$

mit einem $n \in \Z$. Eingesetzt in die anfängliche Gleichung folgt:

\begin{alignat*}{1}
3 q^2 &= p^2 = (3n)^2 = 3\cdot 3 n^2 \\
  q^2 &= 3 n^2
\end{alignat*}

Es ist also auch $q^2$ durch $3$ teilbar und mithin auch $q$. Nun haben wir aber geschlussfolgert, dass sowohl $p$ als auch $q$ durch $3$ teilbar sind. Dann könnte der Bruch $p/q$ mit $3$ gekürzt werden. Das steht im Widerspruch zu Annahme, dass $p/q$ bereits vollständig gekürzt ist. (Anders ausgedrückt haben wir gezeigt, dass wir solch einen Bruch, wenn es ihn gäbe, beliebig oft kürzen könnten, was mit Brüchen aber nicht möglich ist.) Also kann $\sqrt{3}$ keine rationale Zahl sein.