\item 

Auf einfachlogarithmischem Papier ist nur die Ordinate (y-Achse) logarithmisch. Auf doppeltlogarithmischen Papier ist zudem auch die Abzisse (x-Achse) logarithmisch. Rechnerisch ergibt sich:

\begin{enumerate}
\item
$y=x^2$

$\implies \ln(y) = 2*\ln(x)$

Einfach: $f: x \mapsto 2*\ln(x)$, Graph sieht aus wie Logarithmusfunktion

Doppelt: $f: \ln(x) \mapsto 2 \cdot \ln(x)$, Graph sieht aus wie eine Gerade

\item

$y = 2^x$

$\implies \ln(y) = \ln(2^x) = x \ln(2)$

Einfach: $f: x \mapsto \ln(2) \cdot x$, Graph sieht aus wie eine Gerade

Doppelt: $f: \ln(x) \mapsto \ln(2) e^{\ln(x)}$, Graph sieht aus wie eine Exponentialfunktion.
\end{enumerate}

