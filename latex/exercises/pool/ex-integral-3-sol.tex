\item Aus einer Skizze erkennt man, welche Berechnungen vorzunehmen sind. Die Integrationsgrenzen ergeben sich aus dem Schnittpunkt von $f(x)=x^2$ und $g(x)=3$:

$$x^2-3=0$$
$$\implies x_{1,2} = \pm \sqrt{3}$$

Aufgrund der Symmetrie reicht es aus, den positiven Teil $x \geq 0$ zu betrachten. Wir berechnen den Flächeninhalt zwischen x-Achse mit $f(x)$ und ziehen das Ergebnis anschließend vom Flächeninhalt des Rechtecks ab, welches durch den Ursprung und dem positiven Schnittpunkt von $f$ mit $g$ augespannt wird.

Das Rechteck hat den Flächeninhalt $A_\square = 3 \sqrt{3}$. Für den anderen Flächeninhalt ergibt sich:

$$\int_0^{\sqrt{3}} x^2 \d x= \frac{1}{3} \lbrack x^3 \rbrack_0^{\sqrt{3}} = \frac{1}{3} (3\sqrt{3}-0) = \sqrt{3}$$

Somit ist der positive Anteil des Flächeninhalts innerhalb der Parabel $3\sqrt{3}-\sqrt{3}=2\sqrt{3}$ und der gesamte Flächeninhalt das Doppelte davon, also $4\sqrt{3}$.

