\item

\begin{enumerate}
\item Ordnung 1, gewöhnlich, explizit, inhomogen, linear

Direkte Integration liefert die allgemeine Lösung $y=\frac{x^2}{2}+C$. Die Anfangsbedingung ausgewertet ergibt $y(0)=\frac{0^2}{4}+C=4$, also $C=4$. Somit ist die partikuläre Lösung $y=\frac{x^2}{2}+4$.

\item Ordnung 1, gewöhnlich, explizit, homogen, linear

Die allgemeine Lösung lautet $y=C e^x$. Die Anfangsbedingung ergibt $y(0)=C\cdot e^0 = C = 4$. Somit ist die partikuläre Lösung $y=4e^x$.

\item Ordnung 1, gewöhnlich, explizit, homogen, linear

Die allgemeine Lösung lautet $y=C e^{3x}$. Die Anfangsbedingung ergibt $y(0)=C\cdot e^0 = C = 4$. Somit ist die partikuläre Lösung $y = 4 e^{3x}$.

\item Ordnung 2, gewöhnlich, explizit, homogen, linear

$y''-3y=0$. Die zugehörige charakteristische Gleichung lautet $\lambda^2-3 = 0$ und hat die Lösungen $\lambda = \pm \sqrt{3}$. Die allgemeine Lösung lautet mithin

$$y=C_1 e^{\sqrt{3}x} + C_2 e^{-\sqrt{3}x}$$

Für die Anfangsbedingungen benötigen wir noch die 1. Ableitung, diese lautet:

$$y'= C_1 \sqrt{3} e^{\sqrt{3}x} - C_2 \sqrt{3} e^{-\sqrt{3}x}$$

Nun können wir die Anfangsbedingungen einsetzen und erhalten:

$$y(0) = 4 = C_1 + C_2$$
$$y'(0) = 5 = C_1 \sqrt{3} - C_2 \sqrt{3}$$

Dies stellt ein lineares Gleichungssystem dar. Zur Lösung multiplizieren wir die erste Gleichung mit $\sqrt{3}$ und addieren dazu die zweite Gleichung:

$$\sqrt{3}y(0)+y'(0) = 4\sqrt{3} + 5 = 2\sqrt{3}C_1$$
$$\implies C_1 = 2 + \frac{5}{2\sqrt{3}} = 2 + \frac{5}{6}\sqrt{3}$$

Für die andere Konstante folgt somit

$$C_2 = 4 - C_1 = 2 - \frac{5}{6}\sqrt{3}$$

Damit lautet die partikuläre Lösung:

$$y = (2 + \frac{5}{6}\sqrt{3}) e^{\sqrt{3}x} + (2 - \frac{5}{6}\sqrt{3}) e^{-\sqrt{3}x}$$

\item Ordnung 2, gewöhnlich, explizit, homogen, linear

$y''+3y = 0$. Die charakteristische Gleichung lautet $\lambda^2+3 = 0$ und hat die Lösungen $\lambda = \pm \sqrt{3} j$. Die allgemeine Lösung im Reellen lautet damit

$$y = C_1 \cos(\sqrt{3}x) + C_2 \sin(\sqrt{3}x)$$

Die Ableitung lautet:

$$y' = - C_1 \sqrt{3} \sin(\sqrt{3}x) + C_2 \sqrt{3} \cos(\sqrt{3}x)$$

Die Anfangsbedingungen ergeben:

$$y(0) = 4 = C_1$$
$$y'(0) = 5 = C_2 \sqrt{3}$$

Somit ist die partikuläre Lösung:

$$y = 4 \cos(\sqrt{3}x) + \frac{5}{3}\sqrt{3} \sin(\sqrt{3}x)$$

\end{enumerate}

