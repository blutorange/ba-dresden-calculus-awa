\item Das Gravitationspotential einer punktförmigen Masse $m$ im Koordinatenursprung ist gegeben durch $V(x,y,z) = -\frac{G\cdot m}{\sqrt{x^2+y+^2+z^2}}$, wobei $G$ die Gravitationskonstante ist. Die Gravitationskraft, die ein Körper der Masse $M$ an der Position $(x,y,z)$ erfährt, ergibt sich über den Gradienten als $F(x,y,z) = -M \cdot \vec{\nabla} V(x,y,z)$. Berechnen Sie die Gravitationskraft $F$! Ersetzen Sie in dem berechneten Ausdruck die Koordinaten $x,y,z$ durch den Abstand vom Ursprung $r=\sqrt{x^2+y^2+z^2}$. Wie groß ist die Gravitationskraft für einen Körper im Abstand $10\text{m}$? ( $m=2\text{kg}$, $M=4\text{kg}$)