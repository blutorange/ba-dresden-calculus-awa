\item (*) Gemessen wurde die Spannung zu $U=230V \pm 2V$ und die Stromstärke zu $I=20,0A \pm 0.5A$. Es soll der Widerstand nach $R=\frac{U}{I}$ berechnet werden. Wie groß ist dabei (näherungsweise)

\begin{enumerate}
\item der maximale Fehler?
\item der Fehler nach Gauß'scher Fehlerfortpflanzung?
\end{enumerate}

Der maximale Fehler ergibt sich, in dem Sie die größten bzw. kleinsten Spannungen und Ströme in die Formel einsetzen. Der Fehler nach Gauß'scher Fehlerfortpflanzung berechnet sich wie folgt:

$$(\Delta R)^2 = \left(\Delta U \cdot \frac{\partial}{\partial U} R(U,I)\right)^2 + \left(\Delta I \cdot \frac{\partial}{\partial I} R(U,I)\right)^2$$

Dabei ist $\Delta U$ und $\Delta I$ jeweils der Fehler in der gemesenen Spannung beziehungsweise Stromstärke.

