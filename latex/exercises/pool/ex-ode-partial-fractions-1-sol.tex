\item Wir bringen zuerst die Brüche rechts auf einen Nenner:

$$k \cdot x \cdot (1-\frac{x}{M}) = k \cdot x \cdot \frac{M-x}{m}$$

Nun die Variablen trennen, dann integrieren:

$$\frac{\d x}{\d t} = k \cdot x \cdot \frac{M-x}{M}$$
$$\implies \frac{\d x}{x\cdot(M-x)} = \frac{k}{M} \d t$$
$$\implies \int \frac{\d x}{x\cdot(M-x)} = \int \frac{k}{M} \d t$$

Um das Integral auf der linken Seite zu berechnen, zerlegen wir den Integranden in Partialbrüche:

$$\frac{1}{x\cdot(M-x)} = \frac{A_1}{x} + \frac{A_2}{M-x}$$
$$\implies A_1 = \frac{1}{M-0}, A_2 = \frac{1}{M}$$
$$\implies \frac{1}{x\cdot(M-x)} = \frac{1}{M} \left(\frac{1}{x} + \frac{1}{M-x}\right)$$

Weiter mit der Berechnung des Integrals:

$$\int \frac{\d x}{x\cdot(M-x)} = \int \frac{k}{M} \d t$$
$$\implies \frac{1}{M}(\ln(x)-\ln(M-x)) = \frac{k}{M}t + C$$
$$\implies \ln(x)-\ln(M-x) = k\cdot t + M\cdot C$$
$$\implies e^{\ln(x)-\ln(M-x)} = e^{k\cdot t + M\cdot C}$$
$$\implies \frac{x}{M-x} = e^{M\cdot C} e^{k\cdot t}$$
$$\implies x = M \cdot e^{M\cdot C} e^{k\cdot t} - x \cdot e^{M\cdot C} e^{k\cdot t}$$
$$\implies x\cdot\left(1+e^{M\cdot C} e^{k\cdot t}\right) = M \cdot e^{M\cdot C} e^{k\cdot t}$$

Und somit erhalten wir schließlich die Lösung:

$$x(t) = M \cdot \frac{e^{M\cdot C} e^{k\cdot t}}{1+e^{M\cdot C} e^{k\cdot t}}$$

Mögliche Lösungsfunktionen sollten Sie sich graphisch darstellen lassen. Zuerst findet ein exponentielles Wachstum statt. Da es aber eine Obergrenze gibt für das Wachstum, nimmt der Zuwachs wieder ab.

Zum Verständnis der Differentialgleichung $\dot x= k \cdot x \cdot (1-\frac{x}{M})$: Der Anstieg der Größe $x$ ist proportional zum momentanen Wert $x$, also umso größer, je größer der Wert $x$ ist. Dieser Teil beschreibt das exponentielle Wachstum, der Parameter $k$ bestimmt die Geschwindigkeit des Wachstums.

Zudem ist der Anstieg aber auch proportional zu $1-\frac{x}{M}$. Je näher die Größe $x$ der Obergrenze $M$ kommt, desto kleiner wird diese Ausdruck. Somit wird der Anstieg für Werte nahe der Obergrenze wieder kleiner. Dieser Ausdruck sorgt also dafür, dass die Obergrenze nicht überschritten wird.

