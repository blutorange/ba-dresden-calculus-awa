\item In den einzelnen Abschnitten sind die Teilfunktionen stetig. Untersuchen müssen wir die beiden Übergangsstellen. Wir betrachten zuerst $x=-\pi/2$. Es muss gelten: $\lim\lim\limits_{x\to-\pi/2} f(x) = f(-\pi/2)$. Speziell muss also auch der links und rechtsseitige Grenzwert gleich sein. Es ist

$$
  \lim\limits_{x\to{-\pi/2}^{-}} f(x) = -2\sin(-\pi/2) = 2
$$

und

$$
  \lim\limits_{x\to{-\pi/2}^{+}} f(x) = A\sin(-\pi/2)+ B = -A+B
$$

Analog erhalten wir für die Stelle $x=\pi/2$:

$$
  \lim\limits_{x\to{-\pi/2}^{-}} f(x) = A\sin(\pi/2) + B = A+B
$$

und

$$
  \lim\limits_{x\to{-\pi/2}^{+}} f(x) = \cos(\pi/2) = 0
$$

Durch Gleichsetzen gewinnen wir $2=-A+B$ sowie $A+B=0$. Die Lösung dieses Gleichungssystems lautet $A=-1$ und $B=1$.
