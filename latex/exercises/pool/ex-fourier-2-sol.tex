\item Die Periode ist $T=3$. Mithilfe der Formel aus der Vorlesung folgt für den Fourier-Koeffizienten $a_0$:

\begin{alignat*}{1}
	a_0 &= \frac{1}{T} \int\limits_0^T f(x) \diff{x} \\
	    &= \frac{1}{3} \int\limits_0^3 \frac{1}{3}x \diff{x} \\
	    &= \frac{1}{3} \cdot \frac{1}{6} x^2 \Biggr|_0^3 \\
	    &= \frac{1}{3} \cdot \frac{9}{6} \\
	    &= \frac{1}{2}
\end{alignat*}

Dies hätten wir uns auch geometrisch überlegen können, da $a_0$ den Durchschnittswert der Funktion darstellt.

Für $a_n$ folgt:

\begin{alignat*}{1}
	a_n &= \frac{2}{T} \int\limits_0^T f(x) \cos(\frac{2\pi}{T}nx) \diff{x} \\
	    &= \frac{2}{3} \cdot \frac{1}{3} \int\limits_0^3 x \cos(\frac{2\pi}{3}nx) \diff{x}
\end{alignat*}

Dieses Integral lösen wir, indem wir zuerst lineare Substitution $z=\frac{2\pi}{3}nx$ anwenden:

$$
	\int\limits_0^3 x \cos(\frac{2\pi}{3}nx) \diff{x} = \left(\frac{3}{2\pi n}\right)^2 \int\limits_0^{2\pi n} z \cos(z) \diff{z}
$$

Und anschließend partielle Integration anwenden:

\begin{alignat*}{1}
	\int\limits_0^{2\pi n} z \cos(z) \diff{z} &= z \sin(z)\biggr|_0^{2\pi n} - \int\limits_0^{2\pi n} \sin(z) \diff{z} \\
	                                          &= 0 + \cos(z)\biggr|_0^{2\pi n} \\
	                                          &= 0
\end{alignat*}

Damit sind alle $a_1,a_2,\dots=0$. Schließlich berechnen wir noch die Koeffizienten $b_n$:

\begin{alignat*}{1}
	b_n &= \frac{2}{T} \int\limits_0^T f(x) \sin(\frac{2\pi}{T}nx) \diff{x} \\
	    &= \frac{2}{3} \cdot \frac{1}{3} \int\limits_0^3 x \sin(\frac{2\pi}{3}nx) \diff{x}
\end{alignat*}

Auch hier wenden wir lineare Substitution an:

$$
	\int\limits_0^3 x \sin(\frac{2\pi}{3}nx) \diff{x} = \left(\frac{3}{2\pi n}\right)^2 \int\limits_0^{2\pi n} z \sin(z) \diff{z}
$$

Und weiter mit partieller Integration:

\begin{alignat*}{1}
	\int\limits_0^{2\pi n} z \sin(z) \diff{x} &= -z \cos(z)\biggr|_0^{2\pi n} + \int\limits_0^{2\pi n} \cos(z) \diff{z} \\
	                                          &= -z \cos(z)\biggr|_0^{2\pi n} + \sin(x)\biggr|_0^{2\pi n} \\
	                                          &= -2\pi n
\end{alignat*}

Somit folgt für das Integral

\begin{alignat*}{1}
	\int\limits_0^3 x \sin(\frac{2\pi}{3}nx) \diff{x} &= \left( \frac{3}{2\pi n}\right)^2 \int\limits_0^{2\pi n} z \sin(z) \diff{z} \\
	                                                  &= \frac{3}{2\pi n}  \cdot \frac{3}{2\pi n} \cdot (-2\pi n) \\
	                                                  &= -\frac{9}{2\pi n}
\end{alignat*}

Nun erhalten wir endlich für $b_n$:

\begin{alignat*}{1}
	b_n &= \frac{2}{3} \cdot \frac{1}{3} \int\limits_0^3 x \sin(\frac{2\pi}{3}nx) \diff{x} \\
	    &= \frac{2}{3} \cdot \frac{1}{3} \cdot \left(-\frac{9}{2\pi n}\right) \\
	    &= -\frac{1}{\pi n}
\end{alignat*}

Die Grenzfunktion $F$ der Fourier-Reihe lautet damit:

$$
	F(x) = \frac{1}{2} - \sum\limits_{n=1}^\infty \frac{\sin(\frac{2\pi}{3}nx)}{\pi n}
$$

An der Stelle $x=0$ liegt eine Unstetigkeitsstelle vor. Während die Funktion dort den Wert $f(0)=1$ besitzt, weist die Grenzfunktion der Taylor-Reihe wegen der Dirichletschen Bedingung den Wert $F(0)=\frac{1}{2}$ auf (dieser Wert ergibt sich auch, wenn man in die Formel oben explizit $x=0$ einsetzt.)

In Abbildung \ref{fig:FourierChainsaw} sind die ersten Glieder dieser Fourier-Reihe dargestellt.

Für $x=\frac{3}{2\pi}$ erhalten wir (da $F(x)=\frac{1}{3}x$ in $(0,3]$):

\begin{alignat*}{1}
	F(\frac{3}{2\pi}) &= \frac{1}{2} - \sum\limits_{n=1}^\infty \frac{\sin(\frac{2\pi}{3}n\frac{3}{2\pi})}{\pi n} \\
	                  &= \frac{1}{2} - \sum\limits_{n=1}^\infty \frac{\sin(n)}{\pi n} \\
	                  &= \frac{1}{3} \cdot \frac{3}{2\pi} \\
	                  &= \frac{1}{2\pi}
\end{alignat*}

Umgestellt nach der Summe erhalten wir so:

$$
	\sum\limits_{n=1}^\infty \frac{\sin(n)}{\pi n} = \frac{1}{2}-\frac{1}{2\pi}
$$

Nach Multiplikation mit $\pi$ ergibt sich schließlich:

$$
	\sum\limits_{n=1}^\infty \frac{\sin(n)}{n} = \frac{\pi-1}{2}
$$

\begin{figure}
	\centering
	\includegraphics[width=0.48\textwidth]{../gnuplot/ex-fourier-2-img-a-1}
	\includegraphics[width=0.48\textwidth]{../gnuplot/ex-fourier-2-img-a-2}
	\includegraphics[width=0.48\textwidth]{../gnuplot/ex-fourier-2-img-a-3}
	\includegraphics[width=0.48\textwidth]{../gnuplot/ex-fourier-2-img-a-4}
	\includegraphics[width=0.48\textwidth]{../gnuplot/ex-fourier-2-img-a-5}
	\includegraphics[width=0.48\textwidth]{../gnuplot/ex-fourier-2-img-a-6}
	\caption{Die ersten $n$ Glieder der Fourier-Reihe einer Sägezahnfunktion}
	\label{fig:FourierChainsaw}
\end{figure}