\item Ein Körper ruht auf einer Gummimatte an der Stelle $s=0$. Zur Zeit $t=0$ wird er nach oben geworfen. Kurze Zeit später landet er wieder auf der Gummimatte, wobei der Körper leicht nach unten in die Gummimatte sinkt, welche sich anschließend wieder ausdehnt und den Gegenstand zurück in die Höhe $s=0$ der Gummimatte bringt. Dabei ist $s$ die Höhe über der Gummimatte.  Die Orts-Zeit-Kurve für diesen Vorgang laute $s(t) = t^3-15t^2+54t$. Berechnen Sie den Zeitpunkt $t_1$, zu dem der Körper landet und den Zeitpunkt $t_2$, an dem der Körper wieder in Höhe $s=0$ zurückgekehrt ist! Ermitteln Sie ferner eine Formel für die Geschwindigkeit $v(t) = \dd{}{t}s(t)$ und die Beschleunigung $a(t) = \ddn{2}{}{t}s(t)$! Wie groß ist die Minimal- und Maximalgeschwindigkeit des Körpers im Zeitraum $[0,t_2]$? Berechnen Sie schließlich die Durchschnittsgeschwindigkeit des Körpers im Zeitraum $[0,t_1]$. Hinweis: Die Durchschnittsgeschwindigkeit in einem Zeitraum $[t',t'']$ ist gegeben durch $\bar{v} = \frac{1}{t''-t'}\int\limits_{t'}^{t''} |v(t)| \diff{t}$