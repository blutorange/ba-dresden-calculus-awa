\item Anwendung des Wurzelsatzes von Vi\"{e}ta. Dieser sagt aus, dass das Produkt der Nullstellen den Wert $90$ ergeben muss und zudem die Summe der Nullstellen auch gleich $14$ betragen muss. Die Lösungen sollen natürliche Zahlen sein, also ist es nicht erforderlich, beide Vorzeichen auszuprobieren.

Um mögliche Kandidaten zu ermitteln, betrachten wir zuerst die Primfaktorzerlegung der Zahl $90$:

$$90 = 2 \cdot 3 \cdot 3 \cdot 5$$

Es soll drei verschiedene Nullstellen geben. In Frage kommen also $(2,5,3\cdot 3 = 9)$, $(3, 5, 2\cdot 3 = 6)$ und $(2,3,3\cdot 5 = 15)$.

Nun muss die Summe der Nullstellen betragsmäßig 14 ergeben. Dies ist nur erfüllt für $(3,5,6)$, nicht aber für $(2,5,9)$ (Summe 16) und $(2,3,15)$ (Summe 20).

