\item Wir leiten ab:

$$f'(x) = 2+\sin(x)+x\cos(x)$$
$$f''(x) = 2\cos(x)-x\sin(x)$$
$$f'''(x) = -3\sin(x)-x\cos(x)$$
$$f''''(x) = -4\cos(x)+x\sin(x)$$

Es ist $f(0) = 0$, was bereits nahe an $\frac{1}{10}$ liegt. Daher empfiehlt sich die Entwicklungsstelle $x_0=0$:

$$f'(0) = 2$$
$$f''(0) = 2$$
$$f'''(0) = 0$$
$$f''''(0) = -4$$

Da die dritte Ableitung $0$ ist, ist die Schmiegeparabel zugleich auch die Taylorentwicklung 3. Grades.

$$f(x) \approx 0 + \frac{2}{1!}(x-0) + \frac{2}{2!}(x-0)^2 = x^2+2x$$

Nun soll der Funktionswert $\frac{1}{10}$ betragen:
$$f(x) \stackrel{!}{=} \frac{1}{10} = x^2+2x$$
$$\implies x_{1,2} = -1 \pm \sqrt{\frac{11}{10}}$$

Da wir positive x-Werte betrachten, kommt nur die positive Lösung $x = -1 + \sqrt{\frac{11}{10}} \approx 0,0488$ in Frage.

