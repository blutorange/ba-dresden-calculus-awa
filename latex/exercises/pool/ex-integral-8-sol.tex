\item
\begin{enumerate}

\item Für das Argument des Logarithmus dürfen nur positive Zahlen eingesetzt werden. Damit ist $0$ als Argument ausgeschlossen und der Definitionsbereich lautet $\mathbb{D} = \left\lbrace x \in \R | x \ne 0 \right\rbrace = \R \setminus \lbrace 0 \rbrace = (-\infty,0) \cup (0, \infty)$.
\item Mit der Produkt- und Kettenregel gilt: $h'(x) = 1\cdot\ln(x^2) + x \frac{1}{x^2}\cdot 2x = \ln(x^2)+2$
\item Für lokale Extremstellen muss dort die Ableitung identisch $0$ sein. Aus $h'(x) = \ln(x^2)+2 = 0$ folgt $x^2=e^{-2}$ und damit $x = \pm \frac{1}{e}$. Die 2. Ableitung ist $h''(x) = \frac{2}{x}$, für die Kandidatenstelle ergibt sich $h''(\pm \frac{1}{e}) = \pm 2e$. Für $x=-1/e$ ist die 2. Ableitung negativ und es liegt ein Maxima (Hochpunkt) vor, für $x = 1/e$ ein Minima (Tiefpunkt). Die Koordinaten lauten für den Hochpunkt $(-\frac{1}{e}, h(-\frac{1}{e})) =(\frac{1}{e}, \frac{2}{e})$ und für den Tiefpunkt $(\frac{1}{e}, -\frac{2}{e})$
\item Der Graph von $h$ verläuft im betrachteten Intervall im 1. Quadranten(rechts oben). Das bestimmte Integral kann graphisch interpretiert werden als der gerichtete Flächeninhalt zwischen dem Graphen und der Abszisse (x-Achse). Durch das Minuszeichen (Multiplikation der Funktionen $h$ mit $-1$) wird die Funktion an der Abszisse gespiegelt. Damit verläuft der Graph von $-h(x)$ im 4. Quadranten (rechts unten), das Integral liefert damit einen negativen Wert.
\end{enumerate}

