\item

Zuerst müssen wir wissen, wo die Funktion unterhalb bzw. oberhalb der x-Achse liegt. Dazu berechnen wir die Nullstellen $f(x) = x^2-5x+6 = 0$ mittels p-q-Formel:

$$x_1 = 2, x_2 = 3$$

Also müssen wir die Flächen im Bereich $[1,2]$, $[2,3]$ und $[3,5]$ separat berechnen und die Werte anschließend betragsmäßig aufaddieren:

$$A = | \int\limits_1^2 f(x) \diff{x} | + | \int\limits_2^3 f(x) \diff{x} | + | \int\limits_3^5 f(x) \diff{x} | $$

Das unbestimmte Integral von $f$ lautet

$${\int f(x) \diff{x}} = \frac{1}{3} x^3 - \frac{5}{2} x^2 + 6x + C$$

Für den Flächeninhalt erhalten wir damit den Wert $A=5\frac{2}{3} = \frac{17}{3}$.

