\item Für Polynome 2. Grads der Form $p_2(x)=x^2+ax+b = (x-r_1)(x-r_2)$ trifft der sogenannte "Satz von Vieta" die folgende \textbf{Aussage}:

\begin{align*}
&a = -(r_1+r_2)\\
&b= r_1r_2\\
\end{align*}

\begin{enumerate}[label=(\alph*)]

\item Welche (graphische) Bedeutung haben $r_1$ und $r_2$? Wie lauten $a$, $b$ und $r_1$ und $r_2$ für das Polynom $p_2(x) = (\pi+x)(x-2\pi)$?

\item Zeigen Sie, dass der Satz von Vieta für beliebige Polynome $p_2(x)$ stimmt, indem Sie $p_2(x)=(x-r_1)(x-r_2)$ ausmultiplizieren
und einen Koeffizientenvergleich durchführen!

\end{enumerate}
