\item Die DGL für eine gedämpfte Schwingung lautet $\ddot{x} + \frac{r}{m} \dot{x} + \frac{k}{m} x = 0$ (z.B. Federpendel mit Reibung). Dabei ist $m$ die Masse, $k$ die Federkonstante und $r$ der Reibungskoeffizient.
\begin{enumerate}
	\item Zeigen Sie, dass $\lambda_{1,2} = -\omega_0 \gamma \pm \omega_0 \sqrt{\gamma^2-1}$ Lösungen der charakteristischen Gleichung sind ! Dabei ist $\omega_0 = \sqrt{\frac{k}{m}}$ und $\gamma = \frac{r}{2\sqrt{mk}}$
	\item Schreiben Sie die allgemeine Lösung auf! Wann kann man diese mittels Sinus-/Kosinusfunktionen ausdrücken? Begründen Sie, warum man die 3 Fälle $\gamma < 1$, $\gamma = 1$ und $\gamma > 1$ unterscheidet! In welchen Fällen tritt eine Schwingung auf?
	\item Zeigen Sie, dass das Pendel in jedem Fall asymptotisch in die Ruhelage zurückkehrt, also $\lim\limits_{t\to\infty} x(t) = 0$ gilt!
	\item In welchem Fall (für welches $\gamma$) klingt die Schwingung am Schnellsten ab?
\end{enumerate}