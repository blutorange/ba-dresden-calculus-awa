\item Angenommen, der Grenzwert $\lim_{n\to\infty}a_n = a$ existiere. Dann wird der Abstand $|a_{n+1}-a_n|$ beliebig klein und wir können in der rekursiven Definition $a_{n+1} = \frac{a_n}{3}+1$ links uns rechts den Grenzwert $a$ einsetzen:

$a = \frac{a}{3} + 1$ $\implies a = \frac{3}{2}$

Anmerkung: Bei der Abbildung $\varphi(x)=\frac{x}{3}+1$ handelt es sich um eine sogenannte Kontraktion, da gilt $|\varphi(x)-\varphi(y)| = |\frac{x}{3}+1-\frac{y}{3}-1| \le \frac{1}{3}|x-y|$. Nach dem Banachschen Fixpunktsatz existiert damit der Grenzwert.
